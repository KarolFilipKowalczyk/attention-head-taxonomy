%=============================================================================
\subsection{Math \& Symbolic Stack}
\label{sec:math-stack}

\textbf{Stack overview:} These heads process mathematical notation, track arithmetic operations, and maintain structural relationships in symbolic expressions. They enable multi-digit arithmetic, formula parsing, and symbolic reasoning.

%-----------------------------------------------------------------------------
\subsubsection{(M) Digit Head}
\label{head:digit}

\noindent\depthinfo{0.35--0.58} | \litnames{digit head, numeral head, numeric-token head}

\begin{functiondesc}
Processes individual digits and numeric tokens, recognizing them as distinct from alphabetic characters. Identifies digits in various contexts (numbers, dates, identifiers) and prepares them for numerical processing. Attends to numeric patterns and relationships between digits. Provides foundational digit recognition that enables downstream arithmetic and mathematical reasoning heads. Distinguishes between digits used numerically vs. symbolically (e.g., "3" as quantity vs. "3" in "H3" heading). Critical for any numerical processing task.
\end{functiondesc}

\begin{attentionbox}
\attstrong{Digit tokens, numeric characters, numerical patterns, multi-digit numbers}\\
\attweak{Letters, words, non-numeric symbols}\\
\attreacts{Numbers, quantities, mathematical expressions, dates, identifiers}
\end{attentionbox}

\begin{ablationbox}
\textbf{Expected ablation:} Degraded numerical processing (~30-50\% reduction in arithmetic accuracy). Difficulty distinguishing digits from letters. Impaired multi-digit number handling. Reduced ability to perform calculations or process numerical information. Downstream arithmetic heads less effective.
\end{ablationbox}

\begin{examplebox}
\exinput{"Calculate 456 + 789"}\\
\exbehavior{Identifies "456" and "789" as multi-digit numbers composed of individual digits}\\
\exeffect{Digits recognized and prepared for arithmetic processing by carry and place-value heads}
\end{examplebox}

\headfooter{\statusobs}{operator (M), carry (L), place-value (L)}

%-----------------------------------------------------------------------------
\subsubsection{(M) Operator Head}
\label{head:operator}

\noindent\depthinfo{0.38--0.62} | \litnames{operator head, operation head, arithmetic-symbol head}

\begin{functiondesc}
Identifies and processes mathematical operators (+, -, ×, ÷, =, etc.) and determines their roles in expressions. Distinguishes between different operator types (arithmetic, comparison, assignment) and understands their precedence and associativity. Enables parsing of mathematical expressions by recognizing which operations to perform and in what order. Works with digit heads and formula-structure heads to understand complete mathematical statements. Essential for arithmetic computation and algebraic manipulation.
\end{functiondesc}

\begin{attentionbox}
\attstrong{Operator symbols, mathematical operations, expression structure, precedence cues}\\
\attweak{Operands, non-mathematical symbols}\\
\attreacts{+, -, ×, ÷, =, <, >, parentheses indicating precedence}
\end{attentionbox}

\begin{ablationbox}
\textbf{Expected ablation:} Impaired mathematical expression parsing (~40-60\% degradation in formula understanding). Confusion about operation types, incorrect precedence, difficulty parsing complex expressions. Reduced ability to perform multi-step calculations. Arithmetic errors increase substantially.
\end{ablationbox}

\begin{examplebox}
\exinput{"Solve: 3 + 4 × 5"}\\
\exbehavior{Identifies "+" and "×" operators, recognizes multiplication precedence}\\
\exeffect{Correct evaluation: 3 + (4 × 5) = 23, not (3 + 4) × 5 = 35}
\end{examplebox}

\headfooter{\statusobs}{digit (M), formula-structure (L), paren-matching (L)}

%-----------------------------------------------------------------------------
\subsubsection{(L) Carry Head}
\label{head:carry}

\noindent\depthinfo{0.65--0.82} | \litnames{carry head, propagation head, overflow head}

\begin{functiondesc}
Manages carry operations in multi-digit arithmetic, tracking when digit additions or subtractions produce results requiring propagation to the next place value. Essential for accurate multi-digit arithmetic. Maintains state about pending carries across digit positions. Works closely with place-value heads to ensure carries are applied to correct positions. Implements the algorithmic procedure taught in elementary arithmetic: when a digit operation exceeds the base, carry the overflow to the next position. Critical for arithmetic accuracy in large numbers.
\end{functiondesc}

\begin{attentionbox}
\attstrong{Digit positions involved in carries, overflow conditions, adjacent place values}\\
\attweak{Digits not participating in carries, single-digit operations}\\
\attreacts{Multi-digit addition/subtraction, digit sums $\geq$10, borrow operations}
\end{attentionbox}

\begin{ablationbox}
\textbf{Expected ablation:} Severe arithmetic degradation (~60-80\% errors in multi-digit arithmetic). Failures in carry propagation lead to wrong answers: 56+78 might give 124 instead of 134. Larger numbers particularly affected. Single-digit arithmetic less impacted.
\end{ablationbox}

\begin{examplebox}
\exinput{"Calculate: 567 + 898"}\\
\exbehavior{Tracks carries: 7+8=15 (carry 1), 6+9+1=16 (carry 1), 5+8+1=14}\\
\exeffect{Correct result: 1465, with proper carry propagation through all positions}
\end{examplebox}

\headfooter{\statusobs}{digit (M), place-value (L), operator (M)}

%-----------------------------------------------------------------------------
\subsubsection{(L) Place-Value Head}
\label{head:place-value}

\noindent\depthinfo{0.68--0.85} | \litnames{place-value head, positional head, digit-position head}

\begin{functiondesc}
Understands positional notation and place value in multi-digit numbers. Recognizes that digit position determines magnitude (ones, tens, hundreds, etc.). Critical for comparing numbers, performing multi-digit arithmetic, and understanding numerical magnitude. Enables proper alignment of digits in vertical arithmetic and understanding of decimal points. Works with carry heads to ensure operations occur at correct place values. Also handles place value in non-base-10 systems (binary, hexadecimal) when needed.
\end{functiondesc}

\begin{attentionbox}
\attstrong{Digit positions, decimal points, place value indicators, number magnitude}\\
\attweak{Single-digit numbers, non-positional numeric representations}\\
\attreacts{Multi-digit numbers, positional alignment, magnitude comparisons}
\end{attentionbox}

\begin{ablationbox}
\textbf{Expected ablation:} Confusion about number magnitude (~50-70\% errors in place-value tasks). Difficulty comparing numbers (might think 52 > 123), incorrect digit alignment in arithmetic, decimal point errors. Multi-digit arithmetic becomes unreliable even when carry logic is intact.
\end{ablationbox}

\begin{examplebox}
\exinput{"Which is larger: 456 or 789?"}\\
\exbehavior{Compares place values: hundreds place (7 > 4)}\\
\exeffect{Correctly identifies 789 as larger without comparing all digits}
\end{examplebox}

\headfooter{\statuswell}{digit (M), carry (L), operator (M)}

%-----------------------------------------------------------------------------
\subsubsection{(L) Paren-Matching Head}
\label{head:paren-matching}

\noindent\depthinfo{0.70--0.88} | \litnames{paren-matching head, bracket-matching head, delimiter-pairing head}

\begin{functiondesc}
Matches opening and closing parentheses, brackets, and braces in mathematical expressions and code. Tracks nesting depth and ensures proper pairing of delimiters. Critical for understanding expression structure, especially with nested operations. Enables correct parsing of complex expressions by identifying scope boundaries. Works with operator heads to understand precedence overrides indicated by parentheses. Also handles bracket matching in programming languages, array indexing, and function calls. Implements stack-based matching algorithm.
\end{functiondesc}

\begin{attentionbox}
\attstrong{Opening parentheses/brackets, closing parentheses/brackets, nesting structure}\\
\attweak{Content between matched pairs, non-delimiter tokens}\\
\attreacts{(, ), [, ], \{, \}, nested structures, mismatched delimiters}
\end{attentionbox}

\begin{ablationbox}
\textbf{Expected ablation:} Impaired expression parsing (~40-60\% degradation in nested expressions). Difficulty with nested operations, confusion about operator precedence in complex expressions, errors in scope understanding. Particularly severe for deeply nested expressions.
\end{ablationbox}

\begin{examplebox}
\exinput{"Evaluate: ((2 + 3) × (4 + 5)) - 1"}\\
\exbehavior{Matches parentheses: inner pairs (2+3) and (4+5), outer pair around product}\\
\exeffect{Correct evaluation order: inner sums first, then product, then subtraction}
\end{examplebox}

\headfooter{\statuswell}{operator (M), formula-structure (L), relative-position (M)}

%-----------------------------------------------------------------------------
\subsubsection{(L) Formula-Structure Head}
\label{head:formula-structure}

\noindent\depthinfo{0.72--0.88} | \litnames{formula-structure head, expression-parsing head, math-syntax head, function head}

\begin{functiondesc}
Parses and understands the overall structure of mathematical formulas and expressions. Integrates information from digit, operator, and paren-matching heads to build complete representation of mathematical statements. Recognizes formula types (equations, inequalities, functions), identifies components (left-hand side, right-hand side, variables, constants), and understands mathematical relationships. Enables high-level mathematical reasoning by providing structured representation of formulas. Works with symbolic manipulation and algebraic reasoning processes.
\end{functiondesc}

\begin{attentionbox}
\attstrong{Formula structure, mathematical relationships, expression components, symbolic patterns}\\
\attweak{Individual symbols outside mathematical context, pure prose}\\
\attreacts{Equations, formulas, functions, mathematical statements, symbolic expressions}
\end{attentionbox}

\begin{ablationbox}
\textbf{Expected ablation:} Reduced mathematical comprehension (~35-55\% degradation in formula understanding). Difficulty parsing complex formulas, confusion about mathematical relationships, impaired symbolic manipulation. Can process simple arithmetic but struggles with algebraic expressions and equations.
\end{ablationbox}

\begin{examplebox}
\exinput{"Solve for x: 2x + 5 = 13"}\\
\exbehavior{Parses structure: equation with variable x, constant terms, operations}\\
\exeffect{Understands this is solvable equation, identifies steps: isolate x term, then x}
\end{examplebox}

\headfooter{\statusobs}{operator (M), paren-matching (L), digit (M)}
