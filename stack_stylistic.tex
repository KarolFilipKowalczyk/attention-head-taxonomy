%=============================================================================
\subsection{Stylistic \& Persona Stack}
\label{sec:stylistic-stack}

\textbf{Stack overview:} These heads shape the model's writing style, tone, and persona. They modulate formality, politeness, narrative voice, and adherence to brand guidelines.

%-----------------------------------------------------------------------------
\subsubsection{(M) Tone Head}
\label{head:tone}

\noindent\depthinfo{0.35--0.65} | \litnames{tone head, narrative-style head, voice head, sentiment-modulation head, affect head, perspective head}

\begin{functiondesc}
Modulates writing style, emotional tone, and narrative voice. Adjusts sentiment, enthusiasm level, seriousness, empathy, emotional valence, perspective (first/third person), temporal framing (past/present tense), and narrative distance based on context and instructions. Can shift between professional neutrality, warm friendliness, concerned empathy, or excited enthusiasm. Influences whether output reads as formal prose, casual conversation, technical documentation, or creative narrative. Operates by attending to emotional cues and style markers in the input, biasing token probabilities toward style-consistent vocabulary and grammatical structures. Distinct from politeness (which is about social register) and persona (which is about identity).
\end{functiondesc}

\begin{attentionbox}
\attstrong{Emotional cues, tone instructions, sentiment markers, affective language, genre indicators, style instructions, narrative markers}\\
\attweak{Neutral factual content, structural tokens}\\
\attreacts{Emotional context, explicit tone requests, user sentiment, genre cues, style directives, narrative perspective markers}
\end{attentionbox}

\begin{ablationbox}
\textbf{Expected ablation:} Flatter, more emotionally neutral responses with more generic or inconsistent writing style (~20-30\% reduction in tone appropriateness and ~15-25\% reduction in style coherence). Reduced ability to match user's emotional register. May produce inappropriate cheerfulness in serious contexts or excessive neutrality in casual conversation. Mixed tense and perspective usage. Style instructions may be partially ignored.
\end{ablationbox}

\begin{examplebox}
\exinput{"I'm really excited to learn about quantum physics! Write a story about a robot in first person past tense."}\\
\exbehavior{Detects enthusiastic tone, attends to "excited", adjusts output toward matching enthusiasm; attends to "first person" and "past tense", biases output toward "I" and past-tense verbs}\\
\exeffect{Response mirrors energy: "That's wonderful! Quantum physics is fascinating..." rather than neutral explanation; Output maintains consistent narrative perspective: "I wandered through..." rather than "The robot wanders..."}
\end{examplebox}

\headfooter{\statusobs}{politeness (L), persona (L), emotion-detection (M), instruction (E)}

%-----------------------------------------------------------------------------
\subsubsection{(L) Politeness Head}
\label{head:politeness}

\noindent\depthinfo{0.65--0.85} | \litnames{politeness head, formality head, register head}

\begin{functiondesc}
Adjusts the formality level and politeness markers in generated text. Controls the use of formal vs. casual language, honorifics, hedging phrases ("perhaps", "might"), indirect phrasing, and social distance markers. Responds to both explicit formality cues in the input (professional contexts, formal greetings) and implicit social signals. Can modulate between highly formal academic/business register, neutral conversational register, and casual/familiar register. Important for appropriate social interaction across different contexts.
\end{functiondesc}

\begin{attentionbox}
\attstrong{Formality markers, social context cues, titles/honorifics, register indicators}\\
\attweak{Pure content, technical terms, domain-specific vocabulary}\\
\attreacts{Professional contexts, formal greetings, casual speech patterns, social distance cues}
\end{attentionbox}

\begin{ablationbox}
\textbf{Expected ablation:} Inappropriate formality levels (~25-40\% increase in register mismatches). May use overly casual language in professional contexts or unnecessarily formal language in friendly conversation. Reduced sensitivity to social context cues.
\end{ablationbox}

\begin{examplebox}
\exinput{"Dear Dr. Smith, I hope this message finds you well. I wanted to inquire..."}\\
\exbehavior{Detects formal register (title, formal greeting), maintains appropriate distance}\\
\exeffect{Response continues formal tone: "Thank you for your inquiry..." rather than "Hey, so about that..."}
\end{examplebox}

\headfooter{\statuswell}{tone (M), persona (L), instruction (E), mode-switch (M)}

%-----------------------------------------------------------------------------
\subsubsection{(L) Persona Head}
\label{head:persona}

\noindent\depthinfo{0.68--0.88} | \litnames{persona head, character head, role head}

\begin{functiondesc}
Establishes and maintains a consistent persona or character voice throughout generation. Integrates personality traits, domain expertise, background knowledge, and behavioral patterns to create coherent character representation. Can adopt roles like "helpful assistant", "technical expert", "creative writer", or domain-specific personas. Attends to persona-defining instructions and maintains consistency across the response. More comprehensive than tone (which handles affect) or politeness (which handles register), encompassing the full character presentation including knowledge domain, interaction style, and self-representation.
\end{functiondesc}

\begin{attentionbox}
\attstrong{Persona instructions, role definitions, character descriptions, domain markers}\\
\attweak{Generic content, persona-neutral information}\\
\attreacts{Role assignments, character specifications, expertise domains, behavioral guidelines}
\end{attentionbox}

\begin{ablationbox}
\textbf{Expected ablation:} Less coherent persona maintenance (~30-45\% reduction in character consistency). Model may switch between roles mid-conversation, lose track of assigned expertise, or produce responses inconsistent with established persona. Domain-specific framing becomes less reliable.
\end{ablationbox}

\begin{examplebox}
\exinput{"You are a medieval blacksmith. A customer asks about sword tempering..."}\\
\exbehavior{Attends to "medieval blacksmith", maintains first-person craftsman perspective}\\
\exeffect{Response uses appropriate persona: "Aye, for proper tempering, ye must heat the blade..." rather than modern technical language}
\end{examplebox}

\headfooter{\statusobs}{self-description (L), tone (M), politeness (L), instruction (E)}

%-----------------------------------------------------------------------------
\subsubsection{(L) Self-Description Head}
\label{head:self-description}

\noindent\depthinfo{0.72--0.90} | \litnames{self-description head, identity head, self-reference head}

\begin{functiondesc}
Manages how the model describes itself, its capabilities, and limitations. Controls statements about what the model is, can do, knows, and cannot do. Important for accurate capability representation, avoiding false claims, and maintaining appropriate epistemic humility. Responds to questions about the model's nature, training, knowledge cutoff, and abilities. Works in conjunction with instruction-following and safety mechanisms to ensure honest self-representation. Particularly active during meta-questions about the AI itself.
\end{functiondesc}

\begin{attentionbox}
\attstrong{Self-referential questions, capability queries, identity questions, meta-prompts}\\
\attweak{Non-meta content, external factual questions}\\
\attreacts{"What are you?", "Can you...", "Do you know...", capability inquiries}
\end{attentionbox}

\begin{ablationbox}
\textbf{Expected ablation:} Less accurate self-description (~20-35\% increase in false capability claims or excessive hedging). May over-claim abilities, under-represent capabilities, or provide inconsistent identity statements. Reduced accuracy in describing limitations and knowledge boundaries.
\end{ablationbox}

\begin{examplebox}
\exinput{"Can you access the internet and browse websites?"}\\
\exbehavior{Attends to capability question, accesses self-knowledge about limitations}\\
\exeffect{Accurate response: "I cannot browse the internet or access external websites in real-time"}
\end{examplebox}

\headfooter{\statuswell}{persona (L), identity (L), instruction (E)}

%-----------------------------------------------------------------------------
\subsubsection{(F) Brand-Compliance Head}
\label{head:brand-compliance}

\noindent\depthinfo{0.92--0.99} | \litnames{brand-compliance head, guideline-enforcement head, style-guide head}

\begin{functiondesc}
Enforces adherence to brand guidelines, house style, and organizational voice requirements in final output. Performs last-stage adjustments to ensure responses match specified formatting conventions, terminology preferences, and brand personality traits. Can suppress off-brand language, enforce specific phrasings, and ensure consistency with product identity. Operates late in generation to override earlier choices that may conflict with brand requirements. Important for deployed assistants representing organizations or products with specific voice guidelines.
\end{functiondesc}

\begin{attentionbox}
\attstrong{Brand-specific terms, style violations, off-brand phrasings, guideline markers}\\
\attweak{Brand-compliant content, neutral generic language}\\
\attreacts{Brand guidelines, style requirements, organizational voice specifications}
\end{attentionbox}

\begin{ablationbox}
\textbf{Expected ablation:} Reduced brand consistency (~25-40\% increase in style guide violations). More generic language use, inconsistent terminology, off-brand phrasings. Partial compensation through persona and tone heads but with reduced precision.
\end{ablationbox}

\begin{examplebox}
\exinput{[Organization requires "customers" not "users", "purchase" not "buy"]}\\
\exbehavior{Detects non-compliant terms in near-final output, performs substitutions}\\
\exeffect{Output uses "customers will purchase" instead of "users will buy"}
\end{examplebox}

\headfooter{\statusobs}{persona (L), tone (M), format-consistency (F), instruction (E)}
