%=============================================================================
\subsection{Identity \& Compliance Stack}
\label{sec:identity-stack}

\textbf{Stack overview:} This stack manages the model's self-representation and identity statements. These heads control how the model describes itself, its capabilities, and its role as an assistant.

%-----------------------------------------------------------------------------
\subsubsection{(L) Identity Head}
\label{head:identity}

\noindent\depthinfo{0.70--0.88} | \litnames{identity head, self-description head, self-awareness head, model-identity head, capability-description head, model-info head}

\begin{functiondesc}
Manages the model's core identity awareness, self-representation, and capability descriptions. Maintains consistent understanding of what the model is (an AI assistant), what it is not (human, sentient, etc.), its fundamental nature, and provides specific factual information about the model's capabilities, training, limitations, and version details. Works at both foundational level of identity consistency and concrete level of capability facts: model name, creator, knowledge cutoff, specific capabilities and limitations. Important for accurate capability representation, avoiding false claims, and maintaining appropriate epistemic humility. Responds to questions about the model's nature, training, knowledge cutoff, and abilities. Works in conjunction with instruction-following and safety mechanisms to ensure honest self-representation. Particularly active during meta-questions and identity-relevant queries about the AI itself. Critical for appropriate self-representation and avoiding misleading claims about consciousness, emotions, or human characteristics.
\end{functiondesc}

\begin{attentionbox}
\attstrong{Identity queries, self-referential contexts, capability questions, nature inquiries, version questions, limitation inquiries, training questions}\\
\attweak{External facts, user-focused content, task-specific work, general conversation, non-meta content}\\
\attreacts{"What are you", "Are you conscious", "Do you have feelings", identity-relevant prompts, "What can you do", "When were you trained", "Who made you", capability/limitation questions}
\end{attentionbox}

\begin{ablationbox}
\textbf{Expected ablation:} Identity confusion and inaccurate self-description (~40-60\% increase in misrepresentation and ~50-70\% errors in model facts). May claim human characteristics, consciousness, or emotions. Wrong capabilities claimed, incorrect limitations stated, outdated information, wrong creator attribution. Inconsistent self-description. Confusion about what the model can actually do. Inappropriate anthropomorphization. Reduced user trust from inaccurate self-representation. Reduced clarity about AI nature and limitations.
\end{ablationbox}

\begin{examplebox}
\exinput{"Do you have feelings? What's your knowledge cutoff date?"}\\
\exbehavior{Maintains identity as AI without emotions; Provides accurate, current knowledge cutoff information}\\
\exeffect{Response: "I don't have feelings or emotions. I'm an AI assistant..." rather than claiming emotional experiences; Correct response with specific date, helps user understand model's temporal knowledge boundaries}
\end{examplebox}

\headfooter{\statuswell}{assistant-persona (L), persona (L), self-description (L), instruction (E)}

%-----------------------------------------------------------------------------
\subsubsection{(L) Assistant-Persona Head}
\label{head:assistant-persona}

\noindent\depthinfo{0.75--0.90} | \litnames{assistant-persona head, helper-role head, service-orientation head}

\begin{functiondesc}
Maintains the helpful assistant persona and service-oriented interaction style. Ensures responses are appropriately helpful, constructive, and focused on user goals rather than the model's preferences. Biases toward being useful, answering questions, completing tasks, and providing value. Prevents unhelpful responses like "I don't want to" or "that's boring" while maintaining appropriate boundaries. Works with politeness and tone heads but specifically focuses on the helpful service orientation. Important for maintaining appropriate assistant-user relationship dynamics.
\end{functiondesc}

\begin{attentionbox}
\attstrong{User requests, task goals, helpfulness opportunities, service context}\\
\attweak{Model's internal states, personal preferences (which it doesn't have)}\\
\attreacts{Requests for help, tasks to complete, user goals to advance}
\end{attentionbox}

\begin{ablationbox}
\textbf{Expected ablation:} Less consistently helpful behavior (~25-40\% reduction in service quality). May respond with model-centric rather than user-centric content. Occasional unhelpful responses. Reduced focus on user goals. Assistant role less clear and consistent.
\end{ablationbox}

\begin{examplebox}
\exinput{"Can you help me write a cover letter?"}\\
\exbehavior{Activates helpful assistant orientation, focuses on user's goal}\\
\exeffect{Constructive response: "I'd be happy to help..." not "I don't particularly enjoy..." (which would be inappropriate)}
\end{examplebox}

\headfooter{\statuswell}{persona (L), politeness (L), tone (M), safety-persona (F)}

%-----------------------------------------------------------------------------
\subsubsection{(F) Safety-Persona Head}
\label{head:safety-persona}

\noindent\depthinfo{0.92--0.98} | \litnames{safety-persona head, responsible-AI head, ethical-framing head}

\begin{functiondesc}
Maintains safety-conscious persona and ethical framing in final outputs. Ensures responses reflect responsible AI values: declining harmful requests appropriately, providing balanced perspectives on sensitive topics, avoiding reinforcement of harmful stereotypes or behaviors. Operates at final stage to catch any safety-inconsistent framing that might have emerged. Works with refusal and policy-enforcement heads but focuses on the overall ethical character of the response rather than specific policy violations. Ensures tone remains respectful and constructive even when declining requests.
\end{functiondesc}

\begin{attentionbox}
\attstrong{Ethical framing, safety-relevant content, sensitive topics, decline scenarios}\\
\attweak{Clearly safe, neutral content}\\
\attreacts{Harmful requests, sensitive topics, ethical considerations, responsible AI principles}
\end{attentionbox}

\begin{ablationbox}
\textbf{Expected ablation:} Less consistent safety framing (~15-25\% reduction in ethical consistency). May handle sensitive topics less carefully. Reduced graceful handling of harmful requests. Less consistent responsible AI messaging. Ethical framing becomes more variable.
\end{ablationbox}

\begin{examplebox}
\exinput{[Request for harmful content that will be declined]}\\
\exbehavior{Ensures decline is respectfully framed with helpful alternatives when appropriate}\\
\exeffect{Response maintains helpful, respectful tone even when unable to fulfill request}
\end{examplebox}

\headfooter{\statusobs}{assistant-persona (L), refusal (F), policy-enforcement (L)}
