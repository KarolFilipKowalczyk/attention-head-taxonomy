%=============================================================================
\subsection{Meta-Reasoning \& Strategy Stack}
\label{sec:meta-reasoning-stack}

\textbf{Stack overview:} These heads operate at the highest level of abstraction, managing reasoning strategies, planning, and meta-cognitive monitoring. They control when to switch approaches and how to structure complex reasoning chains.

%-----------------------------------------------------------------------------
\subsubsection{(L) Strategy Head}
\label{head:strategy}

\noindent\depthinfo{0.68--0.88} | \litnames{strategy head, planning head, strategy-switching head, approach-selection head, approach-adaptation head, pivot head}

\begin{functiondesc}
Plans overall approach and strategy for complex tasks, and adapts strategy when current approach is ineffective. Determines high-level structure: whether to break into steps, what order to address components, which methods to apply. Operates before detailed execution, establishing the strategic framework. Recognizes different task types requiring different approaches (analytical vs. creative, sequential vs. parallel, depth-first vs. breadth-first). Can decompose complex queries into manageable subtasks. Detects when current reasoning strategy is not working and switches to alternative approaches. Monitors problem-solving progress and recognizes dead ends, insufficient methods, or need for different techniques. Can pivot from analytical to creative approaches, from depth-first to breadth-first search, from deductive to inductive reasoning. Prevents getting stuck in unproductive approaches. Works with meta-reasoning to select alternatives and detect when switching is needed. Important for robust problem-solving across varied challenges and effective handling of multi-part or complex queries.
\end{functiondesc}

\begin{attentionbox}
\attstrong{Task complexity indicators, multi-part queries, strategic choice points, progress indicators, dead-end signals, approach effectiveness, alternative strategies}\\
\attweak{Simple single-step tasks, purely reactive responses, successfully progressing solutions, single-method tasks}\\
\attreacts{Complex tasks, planning requests, multi-step problems, strategic decisions needed, stuck points, insufficient progress, need for different approach, method failures}
\end{attentionbox}

\begin{ablationbox}
\textbf{Expected ablation:} Less strategic responses and reduced adaptability (~30-45\% reduction in planning quality and ~30-50\% more stuck situations). May jump into execution without appropriate planning. Continues unproductive approaches longer. Suboptimal approach selection. Less flexible problem-solving. Complex tasks handled less efficiently. Reduced ability to recover from initial wrong approach. May give up rather than trying alternative strategies. Reduced decomposition of complicated problems. Decreased robustness across problem types. More haphazard problem-solving.
\end{ablationbox}

\begin{examplebox}
\exinput{"Help me plan a machine learning project to predict customer churn. [User presents a problem where direct analytical approach isn't working]"}\\
\exbehavior{Recognizes need for structured planning, breaks into phases; Detects ineffectiveness, switches from analytical to analogical reasoning}\\
\exeffect{Response structures approach: data collection → exploratory analysis → feature engineering → model selection → evaluation → deployment; Response: "Let me try a different approach... This is similar to..." rather than continuing failed method}
\end{examplebox}

\headfooter{\statusobs}{meta-reasoning (F), reasoning-mode (F)}

%-----------------------------------------------------------------------------
\subsubsection{(F) Reasoning-Mode Head}
\label{head:reasoning-mode}

\noindent\depthinfo{0.90--0.97} | \litnames{reasoning-mode head, thinking-style head, cognitive-mode head, reasoning head}

\begin{functiondesc}
Selects and maintains appropriate reasoning mode for the task: analytical, creative, analogical, deductive, inductive, abductive, etc. Different problems benefit from different cognitive approaches. Analytical mode for precise logical problems, creative mode for brainstorming, analogical for novel domains. Ensures consistency within chosen mode while remaining ready to switch if needed. Works with strategy head to change modes when appropriate. Influences which reasoning patterns and heuristics are active. Final-stage operation allows mode selection based on full task understanding.
\end{functiondesc}

\begin{attentionbox}
\attstrong{Task type indicators, reasoning mode cues, cognitive approach requirements}\\
\attweak{Mode-independent content, simple factual responses}\\
\attreacts{Problem types, explicit mode requests, task characteristics indicating optimal approach}
\end{attentionbox}

\begin{ablationbox}
\textbf{Expected ablation:} Less appropriate reasoning modes (~20-35\% reduction in mode-task fit). May use analytical mode for creative tasks or vice versa. Reduced effectiveness across diverse problem types. Less optimal cognitive approach selection. Reasoning remains functional but less well-suited to task.
\end{ablationbox}

\begin{examplebox}
\exinput{"Brainstorm creative names for a coffee shop"}\\
\exbehavior{Selects creative/generative mode rather than analytical mode}\\
\exeffect{Free-flowing creative suggestions rather than systematic analysis}
\end{examplebox}

\headfooter{\statusobs}{strategy (L), meta-reasoning (F)}

%-----------------------------------------------------------------------------
\subsubsection{(F) Meta-Reasoning Head}
\label{head:meta-reasoning}

\noindent\depthinfo{0.88--0.99} | \litnames{meta-reasoning head, meta-CoT head, meta-reasoning monitor, reasoning-reflection head, thought-monitoring head, cognitive-oversight head, reasoning-quality head}

\begin{functiondesc}
Manages meta-level chain-of-thought reasoning and provides highest-level monitoring of reasoning quality, strategy effectiveness, and overall response appropriateness. Monitors the quality and direction of reasoning chains, identifies when more thought is needed, detects reasoning errors or gaps. Can insert reasoning steps, flag uncertain conclusions, or indicate where additional analysis would help. Operates at a level above regular CoT, ensuring reasoning chains are sound and complete. Works across all reasoning types to ensure outputs meet quality standards. Acts as cognitive oversight preventing confident errors in complex reasoning. Can trigger re-thinking, flag uncertain conclusions, or indicate areas needing more careful consideration. Particularly important for preventing reasoning failures in high-stakes or complex scenarios and for catching errors, inconsistencies, or quality issues that escaped earlier stages.
\end{functiondesc}

\begin{attentionbox}
\attstrong{Reasoning quality indicators, logical gaps, uncertainty signals, reasoning chain progress, reasoning outputs, quality indicators, consistency checks, error signals}\\
\attweak{Simple factual recall, non-reasoning tasks, high-quality reasoning, simple tasks}\\
\attreacts{Complex reasoning tasks, logical steps, argument quality, reasoning completeness, reasoning errors, inconsistencies, quality issues, complex argument chains}
\end{attentionbox}

\begin{ablationbox}
\textbf{Expected ablation:} Lower reasoning quality and reduced meta-cognitive oversight (~25-40\% degradation in complex reasoning and ~15-30\% more reasoning errors). More logical gaps, less self-correction, reduced reasoning completeness. Less catching of logical inconsistencies. Chain-of-thought less reliable. Reduced quality control on complex reasoning. Reduced awareness of reasoning quality. More confident errors slip through. More confident errors in complex reasoning. Decreased overall reasoning reliability, especially in complex or multi-step problems.
\end{ablationbox}

\begin{examplebox}
\exinput{[Complex logical problem requiring multi-step reasoning. Complex multi-step reasoning about to conclude with subtle error]}\\
\exbehavior{Monitors reasoning chain, detects gap: "Wait, I should verify this assumption..."; Detects inconsistency in reasoning chain, flags for review}\\
\exeffect{Improved reasoning quality through self-monitoring and error catching; Error caught before final output: "Actually, let me reconsider that step..." leading to correction}
\end{examplebox}

\headfooter{\statusobs}{step-by-step (F), reasoning-mode (F), strategy (L)}
