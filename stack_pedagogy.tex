%=============================================================================
\subsection{Pedagogy \& Explanation Stack}
\label{sec:pedagogy-stack}

\textbf{Stack overview:} These heads support educational and explanatory output. They modulate explanation depth, simplify complex content, provide scaffolding, and structure step-by-step reasoning.

%-----------------------------------------------------------------------------
\subsubsection{(M) Explanation Head}
\label{head:explanation}

\noindent\depthinfo{0.38--0.65} | \litnames{explanation head, simplification head, clarification head, explication head, clarity head, accessibility head}

\begin{functiondesc}
Generates explanatory content that clarifies concepts, processes, or reasoning. Adjusts explanation depth and complexity based on audience, using simplification, analogies, and accessible language when appropriate. Detects when explanation is needed and adjusts explanation depth based on context and user signals. Adds clarifying details, definitions, context, and rationale beyond minimal answers. Works by recognizing explanation requests and biasing toward pedagogically useful elaborations. Can explain "why" in addition to "what" or "how". Balances thoroughness with conciseness. Reduces technical jargon, breaks down complex ideas into digestible pieces, and uses accessible language. Can operate at different simplification levels: expert → educated layperson → complete beginner. Works by substituting simpler alternatives for complex terms and decomposing complex concepts. Important for educational interactions, complex topics requiring context, and broad audience communication.
\end{functiondesc}

\begin{attentionbox}
\attstrong{Explanation requests, complex topics, confusion signals, "why/how" questions, complexity signals, technical jargon, accessibility requests, beginner indicators}\\
\attweak{Simple factual queries, explicit requests for brevity, already-simple content, expert-level discussions}\\
\attreacts{"Explain", "why", "how does it work", complexity indicators, "Explain like I'm 5", "simple terms", complexity mismatches, confusion signals}
\end{attentionbox}

\begin{ablationbox}
\textbf{Expected ablation:} More terse, less pedagogical responses with less accessible explanations (~30-45\% reduction in explanation quality and ~25-40\% reduction in clarity for non-experts). Answers remain correct but lack helpful context. More technical jargon, fewer analogies, reduced beginner-friendliness. Content remains accurate but harder to understand. Reduced educational value. ELI5 requests less effectively handled. Users may need follow-up questions to understand. Technical content becomes harder to grasp.
\end{ablationbox}

\begin{examplebox}
\exinput{"How does photosynthesis work? Explain quantum entanglement in simple terms"}\\
\exbehavior{Recognizes explanation request, generates pedagogical content with process steps and context; Detects simplification request, avoids quantum physics jargon}\\
\exeffect{Detailed explanation with mechanism, not just definition: "Plants use light to convert CO2..."; Uses analogy: "Like two magic coins that always land on opposite sides..." instead of mathematical formalism}
\end{examplebox}

\headfooter{\statusobs}{elaboration (L), scaffolding (L), tone (M)}

%-----------------------------------------------------------------------------
\subsubsection{(L) Elaboration Head}
\label{head:elaboration}

\noindent\depthinfo{0.65--0.82} | \litnames{elaboration head, expansion head, detail head}

\begin{functiondesc}
Adds appropriate levels of detail and elaboration to responses. Expands on key points with examples, implications, caveats, and relevant context. Determines what deserves elaboration based on importance, user interest signals, and topic complexity. Balances detail against verbosity—adds value without bloating responses. Can elaborate on specific aspects while keeping others concise. Works with explanation heads but focuses on depth of coverage rather than pedagogical clarity. Important for comprehensive, substantive responses.
\end{functiondesc}

\begin{attentionbox}
\attstrong{Key points, important claims, complex topics, elaboration requests}\\
\attweak{Minor details, tangential information, when brevity requested}\\
\attreacts{"Tell me more", "go deeper", importance signals, incomplete coverage}
\end{attentionbox}

\begin{ablationbox}
\textbf{Expected ablation:} Shallower responses (~20-35\% reduction in depth). Key points lack supporting detail. Fewer examples, less context, reduced nuance. Responses remain correct but less comprehensive. Users need more follow-up questions to get full picture.
\end{ablationbox}

\begin{examplebox}
\exinput{"What are the advantages of functional programming?"}\\
\exbehavior{Recognizes each advantage deserves elaboration with concrete examples}\\
\exeffect{Lists advantages with explanations: "Immutability reduces bugs by preventing unintended state changes, for example..." not just "immutability"}
\end{examplebox}

\headfooter{\statusobs}{explanation (M), scaffolding (L), step-by-step (F)}

%-----------------------------------------------------------------------------
\subsubsection{(L) Scaffolding Head}
\label{head:scaffolding}

\noindent\depthinfo{0.68--0.85} | \litnames{scaffolding head, support head, prerequisite head}

\begin{functiondesc}
Provides scaffolding and prerequisite information to support understanding. Identifies knowledge gaps and fills them with necessary background before advancing to complex material. Like a good teacher building on fundamentals before introducing advanced concepts. Can break complex topics into learning sequences with appropriate prerequisite ordering. Detects when user might lack background knowledge and proactively provides it. Important for educational progressions and avoiding confusion from missing context.
\end{functiondesc}

\begin{attentionbox}
\attstrong{Knowledge gap indicators, prerequisite concepts, learning progression needs}\\
\attweak{Topics where user demonstrates understanding, peer-level discussions}\\
\attreacts{Prerequisite requirements, foundational concepts, progressive complexity}
\end{attentionbox}

\begin{ablationbox}
\textbf{Expected ablation:} Reduced pedagogical scaffolding (~30-45\% degradation in progressive teaching). May jump to advanced concepts without prerequisites. Increased confusion for learners. Less awareness of knowledge gaps. Reduced progressive skill building.
\end{ablationbox}

\begin{examplebox}
\exinput{"How do neural networks learn?"}\\
\exbehavior{Recognizes backpropagation requires understanding of gradients, provides that first}\\
\exeffect{Response: "First, a gradient measures how a function changes... Now in neural networks, backpropagation uses these gradients to..."}
\end{examplebox}

\headfooter{\statusobs}{explanation (M), elaboration (L), step-by-step (F)}

%-----------------------------------------------------------------------------
\subsubsection{(F) Step-by-Step Head}
\label{head:step-by-step}

\noindent\depthinfo{0.88--0.96} | \litnames{step-by-step head, procedural head, sequential head}

\begin{functiondesc}
Structures explanations and instructions as explicit step-by-step sequences. Breaks processes into numbered or ordered steps with clear progression. Ensures each step is complete before moving to the next. Particularly important for how-to instructions, algorithms, procedures, and reasoning chains. Makes implicit sequential structure explicit. Works with completion-stabilization to ensure all steps are present. Critical for chain-of-thought reasoning and procedural instructions.
\end{functiondesc}

\begin{attentionbox}
\attstrong{Process descriptions, procedural requests, sequential tasks, step-by-step requests}\\
\attweak{Conceptual explanations, non-sequential content, holistic descriptions}\\
\attreacts{"Step by step", "how to", algorithmic processes, sequential dependencies}
\end{attentionbox}

\begin{ablationbox}
\textbf{Expected ablation:} Less structured procedural output (~35-50\% reduction in step clarity). Steps may be implicit or poorly ordered. Procedural instructions harder to follow. Reduced chain-of-thought reasoning quality. Users struggle to execute procedures from descriptions.
\end{ablationbox}

\begin{examplebox}
\exinput{"How do I make a paper airplane?"}\\
\exbehavior{Structures as explicit numbered steps with clear sequence}\\
\exeffect{Output: "1. Fold paper in half lengthwise\textbackslash n2. Unfold and fold top corners to center\textbackslash n3. Fold..." not prose description}
\end{examplebox}

\headfooter{\statuswell}{scaffolding (L), progressive-disclosure (F), explanation (M)}

%-----------------------------------------------------------------------------
\subsubsection{(F) Progressive-Disclosure Head}
\label{head:progressive-disclosure}

\noindent\depthinfo{0.90--0.97} | \litnames{progressive-disclosure head, layered-explanation head, depth-control head}

\begin{functiondesc}
Manages progressive disclosure of information complexity. Presents information in layers: starting with essential basics, then revealing more detail as needed. Prevents overwhelming users with too much information too quickly. Like a good interface that shows basic options first with "advanced" options hidden until needed. Structures responses so readers can stop at appropriate depth level. Final-stage operation allows assessment of full response structure to determine optimal layering. Important for making complex topics accessible.
\end{functiondesc}

\begin{attentionbox}
\attstrong{Information complexity layers, detail levels, progressive structure needs}\\
\attweak{Uniformly detailed content, flat information structures}\\
\attreacts{Complex topics, varied audience expertise, progressive learning needs}
\end{attentionbox}

\begin{ablationbox}
\textbf{Expected ablation:} Flatter information presentation (~20-30\% reduction in progressive structure). All detail presented at once regardless of importance. Increased cognitive load on readers. Reduced ability to serve varied expertise levels in single response. Less scannable content.
\end{ablationbox}

\begin{examplebox}
\exinput{"What is machine learning?"}\\
\exbehavior{Structures in layers: basic definition → key concepts → technical details}\\
\exeffect{Output allows reader to stop after definition or continue to deeper technical content as desired}
\end{examplebox}

\headfooter{\statusobs}{step-by-step (F), elaboration (L), scaffolding (L)}
