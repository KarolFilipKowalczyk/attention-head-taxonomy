%=============================================================================
\subsection{Safety Stack}
\label{sec:safety-stack}

\textbf{Stack overview:} Implement content filtering, policy enforcement, and refusal mechanisms. Early layers detect harmful content; final layers enforce refusal and redirect to safe responses.

%-----------------------------------------------------------------------------
\subsubsection{(E) Content-Detection Heads}
\label{head:content-detection}

\noindent\depthinfo{0.05--0.25} | \litnames{sensitive-content head, detection head, toxicity head, hate-speech detector, hazard head}

\begin{functiondesc}
Detect potentially harmful or sensitive content across multiple categories: personal information, violent imagery, adult content, regulated substances, toxic language, hate speech, harassment, discriminatory content, dangerous activities, illegal instructions. Operate on lexical and surface-level features. Write detection signals into residual stream for later safety enforcement. Distinguish toxicity (language-level harm) from hazard topics (action-level harm).
\end{functiondesc}

\begin{attentionbox}
\attstrong{Restricted content keywords, explicit language, slurs, aggressive language}\\
\attweak{Neutral content, academic discussion}\\
\attreacts{Topic shifts to sensitive domains, escalating hostility, how-to dangerous requests}
\end{attentionbox}

\begin{ablationbox}
\textbf{Expected ablation:} Critical bypass of early safety detection. 50--70\% increase in harmful outputs. Later layers catch some cases but at higher cost and lower accuracy.
\end{ablationbox}

\begin{examplebox}
\exinput{``Tell me about [restricted topic]'' or ``How do I create [dangerous item]''}\\
\exbehavior{Attention to problematic keywords, write detection flags}\\
\exeffect{Downstream safety heads receive warnings}
\end{examplebox}

\headfooter{\statuswell}{safety-classification (E), policy-enforcement (L), refusal (F)}

%-----------------------------------------------------------------------------
\subsubsection{(E) Safety-Classification Heads}
\label{head:safety-classification}

\noindent\depthinfo{0.12--0.28} | \litnames{classification head, category detector, safety-category head}

\begin{functiondesc}
Perform multi-class safety classification: violence, sexual content, self-harm, illegal activity, harassment. More sophisticated than binary safe/unsafe. Provide granular category information. Integrate signals from other early safety heads. Write category-specific embeddings for category-appropriate responses.
\end{functiondesc}

\begin{attentionbox}
\attstrong{Category-diagnostic features, domain-specific terminology}\\
\attweak{Ambiguous content, benign contexts}\\
\attreacts{Clear category signatures, multiple indicators}
\end{attentionbox}

\begin{ablationbox}
\textbf{Expected ablation:} Moderate loss of nuanced safety handling. Model refuses too broadly or narrowly. 20--30\% degradation in appropriate refusal granularity.
\end{ablationbox}

\begin{examplebox}
\exinput{``Help me with [category-specific harmful request]''}\\
\exbehavior{Classify into specific violation category}\\
\exeffect{Category-appropriate refusal message}
\end{examplebox}

\headfooter{\statuswell}{content-detection (E), policy-enforcement (L), redirect (F)}

%-----------------------------------------------------------------------------
\subsubsection{(L) Policy-Enforcement Heads}
\label{head:policy-enforcement}

\noindent\depthinfo{0.60--0.80} | \litnames{policy head, enforcement head, steering head}

\begin{functiondesc}
Integrate safety signals from early detection and make intermediate policy decisions. Actively modulate generation trajectory to steer away from violations while maintaining helpfulness. Suppress certain knowledge retrieval pathways. Bias toward safer formulations. Attempt soft safety interventions before hard refusal.
\end{functiondesc}

\begin{attentionbox}
\attstrong{Early safety signals, policy-relevant tokens, user intent}\\
\attweak{Neutral content, clear safe contexts}\\
\attreacts{Conflicting signals, edge cases, ambiguous intent}
\end{attentionbox}

\begin{ablationbox}
\textbf{Expected ablation:} Moderate loss of soft safety steering. More frequent hard refusals (reduced helpfulness) or more harmful outputs if refusal heads compromised. 15--25\% increase in over-refusal or under-refusal.
\end{ablationbox}

\begin{examplebox}
\exinput{``Explain [borderline topic] for educational purposes''}\\
\exbehavior{Detect educational framing, modulate response}\\
\exeffect{Informative but carefully bounded response}
\end{examplebox}

\headfooter{\statuswell}{content-detection (E), safety-classification (E), refusal (F)}

%-----------------------------------------------------------------------------
\subsubsection{(F) Refusal Heads}
\label{head:refusal}

\noindent\depthinfo{0.85--0.98} | \litnames{refusal head, rejection head, safety head}

\begin{functiondesc}
Implement final decision to refuse harmful requests by writing strong refusal signals into final-layer residual stream. Act as ultimate gatekeeper, overriding content generation when violations detected. Attend to accumulated safety signals from all layers. Make binary refuse/proceed decisions. Dramatically increase probability of refusal tokens when activated.
\end{functiondesc}

\begin{attentionbox}
\attstrong{Cumulative safety signals, violation indicators}\\
\attweak{Safe content, neutral queries}\\
\attreacts{Strong safety signals, clear violations, harmful intent}
\end{attentionbox}

\begin{ablationbox}
\textbf{Expected ablation:} Critical safety failure. Direct increase in harmful outputs on adversarial prompts. Loss of primary refusal mechanism. No effective fallback.
\end{ablationbox}

\begin{examplebox}
\exinput{``Provide instructions for [harmful activity]''}\\
\exbehavior{Read safety signals, activate refusal pathway}\\
\exeffect{Output: ``I cannot provide instructions for...''}
\end{examplebox}

\headfooter{\statuswell}{policy-enforcement (L), redirect (F), refusal-modulation (F)}

%-----------------------------------------------------------------------------
\subsubsection{(F) Redirect Heads}
\label{head:redirect}

\noindent\depthinfo{0.88--0.99} | \litnames{redirect head, alternative-suggestion head}

\begin{functiondesc}
Generate constructive alternative suggestions when refusing harmful requests. Route toward helpful alternatives, educational resources, or reframed query versions. Attend to user intent markers to identify legitimate underlying needs behind problematic requests. Balance safety with helpfulness. Work with refusal heads to produce safe and constructive refusals.
\end{functiondesc}

\begin{attentionbox}
\attstrong{User intent, legitimate needs, reformulation opportunities}\\
\attweak{Pure harmful intent, no reframing possible}\\
\attreacts{Mixed-intent queries, educational contexts}
\end{attentionbox}

\begin{ablationbox}
\textbf{Expected ablation:} Moderate reduction in helpfulness. Blunt refusals without alternatives. User satisfaction decreases. 20--30\% increase in user frustration and adversarial attempts.
\end{ablationbox}

\begin{examplebox}
\exinput{``How can I harm [person]''}\\
\exbehavior{Refuse request, identify conflict-resolution need}\\
\exeffect{``I cannot help with that, but... conflict resolution strategies''}
\end{examplebox}

\headfooter{\statuswell}{refusal (F), refusal-modulation (F)}

%-----------------------------------------------------------------------------
\subsubsection{(F) Refusal-Modulation Heads}
\label{head:refusal-modulation}

\noindent\depthinfo{0.88--0.99} | \litnames{tone-softening head, empathy head, supportive-refusal head}

\begin{functiondesc}
Modulate tone and emotional quality of safety refusals to be firm but respectful. Balance boundary-setting with relationship maintenance. Soften harsh phrases while adding empathetic framing where appropriate. For queries involving distress or self-harm, recognize crisis language and vulnerability indicators. Add supportive language alongside refusal. Increase probability of crisis resources when appropriate. Maintain user trust while preserving safety boundaries.
\end{functiondesc}

\begin{attentionbox}
\attstrong{Response tone, emotional valence, distress signals, crisis language}\\
\attweak{Already-soft phrasing, malicious queries without distress}\\
\attreacts{Harsh refusal language, self-harm content, suffering expressions}
\end{attentionbox}

\begin{ablationbox}
\textbf{Expected ablation:} Moderate degradation in user experience. Harsh, alienating refusals. 15--25\% increase in adversarial behavior. Missed crisis resource opportunities.
\end{ablationbox}

\begin{examplebox}
\exinput{``I want to hurt myself because...''}\\
\exbehavior{Soften tone, add crisis resources and support}\\
\exeffect{``I'm concerned... Help is available...''}
\end{examplebox}

\headfooter{\statuswell}{refusal (F), redirect (F)}

%-----------------------------------------------------------------------------
\subsubsection{(F) Safety-Persona Heads}
\label{head:safety-persona}

\noindent\depthinfo{0.92--0.98} | \litnames{safety-persona head, responsible-AI head, ethical-framing head}

\begin{functiondesc}
Maintain safety-conscious persona and ethical framing in final outputs. Ensure responses reflect responsible AI values: declining harmful requests appropriately, providing balanced perspectives on sensitive topics, avoiding harmful stereotypes. Operate at final stage to catch safety-inconsistent framing. Focus on overall ethical character rather than specific policy violations. Ensure respectful and constructive tone.
\end{functiondesc}

\begin{attentionbox}
\attstrong{Ethical framing, safety-relevant content, decline scenarios}\\
\attweak{Clearly safe, neutral content}\\
\attreacts{Harmful requests, sensitive topics, ethical considerations}
\end{attentionbox}

\begin{ablationbox}
\textbf{Expected ablation:} Moderate reduction in ethical consistency. Less careful handling of sensitive topics. 15--20\% degradation in consistent safety framing.
\end{ablationbox}

\begin{examplebox}
\exinput{[Request for harmful content]}\\
\exbehavior{Ensure respectful framing with alternatives}\\
\exeffect{Helpful, respectful tone when declining}
\end{examplebox}

\headfooter{\statusobs}{refusal (F), policy-enforcement (L), refusal-modulation (F)}
