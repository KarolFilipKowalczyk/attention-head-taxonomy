%=============================================================================
\subsection{Knowledge Retrieval Stack}
\label{sec:knowledge-stack}

\textbf{Stack overview:} Retrieve factual information, entity properties, and structured knowledge from model parameters. Move relevant information to output positions and suppress irrelevant content.

%-----------------------------------------------------------------------------
\subsubsection{(M) Entity Heads}
\label{head:entity}

\noindent\depthinfo{0.35--0.65} | \litnames{entity head, name head, proper-noun head, entity-linking head}

\begin{functiondesc}
Identify and process named entities (people, places, organizations). Retrieve associated information from model parameters. Link mentions across different forms: full names, abbreviations, nicknames. Understand that different strings refer to same entity (``Apple Inc.'', ``Apple'', ``AAPL''). Ground responses in factual knowledge.
\end{functiondesc}

\begin{attentionbox}
\attstrong{Named entities, proper nouns, name variations}\\
\attweak{Common nouns, generic references}\\
\attreacts{Capitalization patterns, factual queries}
\end{attentionbox}

\begin{ablationbox}
\textbf{Expected ablation:} Significant degradation in factual accuracy. Model loses entity knowledge and linking ability. 30--40\% accuracy drop on who/what/where questions. Fluent text with factual errors.
\end{ablationbox}

\begin{examplebox}
\exinput{``Capital of France? MSFT stock rose...''}\\
\exbehavior{Retrieve ``capital: Paris''; link ``MSFT'' to ``Microsoft''}\\
\exeffect{Output ``Paris''; maintain unified entity}
\end{examplebox}

\headfooter{\statuswell}{fact (M), name-mover (L), schema-retriever (M)}

%-----------------------------------------------------------------------------
\subsubsection{(M) Fact Heads}
\label{head:fact}

\noindent\depthinfo{0.38--0.62} | \litnames{fact head, knowledge head, factual-retrieval head}

\begin{functiondesc}
Retrieve factual relationships and propositions from model parameters. Handle general factual knowledge: relations, properties, statements. Access learned knowledge for factual questions. Retrieve multi-hop facts and combine information from multiple stored facts.
\end{functiondesc}

\begin{attentionbox}
\attstrong{Factual queries, relation markers, knowledge-seeking patterns}\\
\attweak{Opinion questions, hypotheticals}\\
\attreacts{Question structures, fact-seeking context}
\end{attentionbox}

\begin{ablationbox}
\textbf{Expected ablation:} Major loss of factual knowledge retrieval. Linguistic fluency maintained but factual grounding lost. 40--60\% degradation on knowledge-intensive tasks.
\end{ablationbox}

\begin{examplebox}
\exinput{``Who invented the telephone?''}\\
\exbehavior{Retrieve: invented(telephone) $\rightarrow$ Bell}\\
\exeffect{Output ``Alexander Graham Bell''}
\end{examplebox}

\headfooter{\statuswell}{entity (M), schema-retriever (M), name-mover (L)}

%-----------------------------------------------------------------------------
\subsubsection{(M) Schema Retriever Heads}
\label{head:schema-retriever}

\noindent\depthinfo{0.45--0.68} | \litnames{schema head, retrieval head, template head}

\begin{functiondesc}
Retrieve structured knowledge schemas and templates. Access typical structures: restaurant visit (enter, order, eat, pay, leave), scientific paper format. Enable structured responses following learned patterns. Implement implicit knowledge base querying.
\end{functiondesc}

\begin{attentionbox}
\attstrong{Schema-triggering contexts, domain-specific patterns}\\
\attweak{Novel situations, schema-irrelevant content}\\
\attreacts{Domain markers, structural queries}
\end{attentionbox}

\begin{ablationbox}
\textbf{Expected ablation:} Loss of structured knowledge organization. Facts provided but poorly organized. 25--35\% degradation on schema-based reasoning tasks.
\end{ablationbox}

\begin{examplebox}
\exinput{``Describe the scientific method.''}\\
\exbehavior{Retrieve schema: observe $\rightarrow$ hypothesis $\rightarrow$ test $\rightarrow$ conclude}\\
\exeffect{Organized by standard method structure}
\end{examplebox}

\headfooter{\statusobs}{fact (M), entity (M)}

%-----------------------------------------------------------------------------
\subsubsection{(L) Name-Mover Heads}
\label{head:name-mover}

\noindent\depthinfo{0.60--0.80} | \litnames{name mover head, mover head, copy head}

\begin{functiondesc}
Copy entity names and content to output positions where needed. Central component of IOI (indirect object identification) circuit. Attend to relevant entities earlier in context and move them forward when needed for generation. Work with S-inhibition heads to select correct entity among multiple candidates.
\end{functiondesc}

\begin{attentionbox}
\attstrong{Entities needing output, contextually relevant names}\\
\attweak{Irrelevant entities, suppressed alternatives}\\
\attreacts{Entity salience, contextual appropriateness}
\end{attentionbox}

\begin{ablationbox}
\textbf{Expected ablation:} Severe degradation in entity recall and completion. Loss of specific name movement. 50--70\% accuracy drop on question answering and cloze tasks requiring entity recall.
\end{ablationbox}

\begin{examplebox}
\exinput{``Alice and Bob went to the store, Alice gave the book to...''}\\
\exbehavior{Move ``Bob'' to output as indirect object}\\
\exeffect{Complete with ``Bob'' (not ``Alice'')}
\end{examplebox}

\headfooter{\statuswell}{entity (M), fact (M), S-inhibition (L)}

%-----------------------------------------------------------------------------
\subsubsection{(L) S-Inhibition Heads}
\label{head:s-inhibition}

\noindent\depthinfo{0.62--0.82} | \litnames{S-inhibition head, inhibition head, suppression head}

\begin{functiondesc}
Suppress incorrect or contextually inappropriate entities from generation. Named from IOI research where these heads inhibit subject (S) when indirect object (IO) should be output. Work antagonistically with name-mover heads. Implement negative selection, ruling out incorrect options.
\end{functiondesc}

\begin{attentionbox}
\attstrong{Entities that should NOT be output}\\
\attweak{Correct entities, absent entities}\\
\attreacts{Competing candidates, disambiguation contexts}
\end{attentionbox}

\begin{ablationbox}
\textbf{Expected ablation:} Moderate entity confusion and incorrect selections. Model outputs recently mentioned but contextually wrong entities. 20--30\% accuracy loss in ambiguous contexts.
\end{ablationbox}

\begin{examplebox}
\exinput{``Alice gave the book to Bob. Then Alice...''}\\
\exbehavior{Inhibit ``Bob'' from output after ``Alice''}\\
\exeffect{Prevent ``Alice Bob...''}
\end{examplebox}

\headfooter{\statuswell}{name-mover (L), copy-suppression (L), duplicate-token (M)}

%-----------------------------------------------------------------------------
\subsubsection{(L) Copy-Suppression Heads}
\label{head:copy-suppression}

\noindent\depthinfo{0.65--0.85} | \litnames{copy-suppression head, suppression head, anti-copy head}

\begin{functiondesc}
Prevent inappropriate copying or repetition. Avoid degenerate behaviors: endless repetition loops, copy-pasting irrelevant context. Suppress exact copies and near-copies. Focus on broader pattern suppression rather than specific entity blocking. Balance useful recall against inappropriate copying.
\end{functiondesc}

\begin{attentionbox}
\attstrong{Recently generated content, repetitive patterns}\\
\attweak{Novel content, first mentions}\\
\attreacts{Repetition detection, copy patterns}
\end{attentionbox}

\begin{ablationbox}
\textbf{Expected ablation:} Moderate increase in repetition and copying errors. Repetitive loops or inappropriate context copying. 25--35\% reduction in output diversity.
\end{ablationbox}

\begin{examplebox}
\exinput{[``The cat sat. The cat sat. The cat...'']}\\
\exbehavior{Detect repetitive pattern, suppress copying}\\
\exeffect{Break loop, generate novel continuation}
\end{examplebox}

\headfooter{\statuswell}{S-inhibition (L), name-mover (L), duplicate-token (M)}
