%=============================================================================
\subsection{Routing \& Relevance Stack}
\label{sec:routing-stack}

\textbf{Stack overview:} Determine which input parts are relevant to current task and route attention accordingly. Filter information, focus on salient content, manage global context.

%-----------------------------------------------------------------------------
\subsubsection{(M) Topic-Relevance Heads}
\label{head:topic-relevance}

\noindent\depthinfo{0.35--0.60} | \litnames{topic-relevance head, relevance head, salience head, filter head}

\begin{functiondesc}
Identify primary topic and determine which input context parts are relevant to current generation task. Filter irrelevant information while highlighting salient content. Compute relevance scores based on semantic similarity, task alignment, and topical coherence. Maintain topic coherence by attending to topic-establishing phrases and domain indicators.
\end{functiondesc}

\begin{attentionbox}
\attstrong{Task-relevant content, topic indicators, domain markers}\\
\attweak{Off-topic material, unrelated context}\\
\attreacts{Semantic relevance, topical alignment, topic transitions}
\end{attentionbox}

\begin{ablationbox}
\textbf{Expected ablation:} Moderate reduction in focus with increased topic drift. Model distracted by irrelevant content. 20--30\% degradation on long contexts. Responses wander off-topic or miss key details.
\end{ablationbox}

\begin{examplebox}
\exinput{``[Document: cars, climate, history] What caused 2008 financial crisis?''}\\
\exbehavior{Mark financial/economic content relevant, de-emphasize cars/climate}\\
\exeffect{Focus on economic information, ignore unrelated context}
\end{examplebox}

\headfooter{\statuswell}{focus (L), router (L), entity (M)}

%-----------------------------------------------------------------------------
\subsubsection{(L) Focus Heads}
\label{head:focus}

\noindent\depthinfo{0.65--0.80} | \litnames{focus head, attention-routing head, spotlight head}

\begin{functiondesc}
Concentrate attention on most salient elements for current generation step. Implement dynamic focus allocation: suppress less important content, amplify critical information. More selective than topic-relevance heads. Determine exactly which tokens should influence next token prediction. Shift focus as generation proceeds.
\end{functiondesc}

\begin{attentionbox}
\attstrong{Currently salient tokens, query-critical content}\\
\attweak{Background information, low-priority details}\\
\attreacts{Query emphasis, current generation needs}
\end{attentionbox}

\begin{ablationbox}
\textbf{Expected ablation:} Moderate reduction in focus precision. Model gives equal weight to important and peripheral information. 15--25\% degradation on targeted responses. Answers more diffuse, less direct.
\end{ablationbox}

\begin{examplebox}
\exinput{``Among all details, what is the MAIN cause?''}\\
\exbehavior{Attend to ``MAIN cause'', suppress secondary details}\\
\exeffect{Direct answer: primary cause vs. all factors}
\end{examplebox}

\headfooter{\statuswell}{topic-relevance (M), router (L)}

%-----------------------------------------------------------------------------
\subsubsection{(L) Router Heads}
\label{head:router}

\noindent\depthinfo{0.70--0.85} | \litnames{router head, dispatch head, task-routing head}

\begin{functiondesc}
Route query types to appropriate processing strategies or knowledge domains. Act as dispatchers recognizing query type (factual, creative, analytical, procedural). Bias processing toward suitable approaches. Activate different downstream heads based on task classification. Enable dynamic strategy selection based on input characteristics.
\end{functiondesc}

\begin{attentionbox}
\attstrong{Query-type indicators, task markers, domain signals}\\
\attweak{Content details, specific entities}\\
\attreacts{Task classification cues, query structure}
\end{attentionbox}

\begin{ablationbox}
\textbf{Expected ablation:} Moderate reduction in task-appropriate processing. Suboptimal strategy selection. Creative approaches for factual queries or vice versa. 20--30\% degradation on diverse query types.
\end{ablationbox}

\begin{examplebox}
\exinput{``Calculate compound interest vs. Write poem about compound interest''}\\
\exbehavior{Route first to mathematical, second to creative}\\
\exeffect{Calculation vs. literary devices}
\end{examplebox}

\headfooter{\statusobs}{focus (L), mode-switch (M), instruction (E)}

%-----------------------------------------------------------------------------
\subsubsection{(F) Global-Attention Heads}
\label{head:global-attention}

\noindent\depthinfo{0.88--0.96} | \litnames{global-attention head, full-context head, summary-attention head}

\begin{functiondesc}
Maintain broad attention over entire context to integrate global information in final generation. Unlike focused heads, attend widely to ensure complete picture considered before finalization. Catch context elements that earlier focused attention missed. Act as final integration mechanism for coherence checking and global consistency.
\end{functiondesc}

\begin{attentionbox}
\attstrong{All context tokens, document-level information, global constraints}\\
\attweak{Fine-grained local patterns, individual token details}\\
\attreacts{Complete context, document-level coherence}
\end{attentionbox}

\begin{ablationbox}
\textbf{Expected ablation:} Moderate reduction in global coherence. Responses miss relevant information from distant context. 15--25\% increase in locally optimal but globally suboptimal outputs.
\end{ablationbox}

\begin{examplebox}
\exinput{[Long context: ``Keep it under 100 words'']}\\
\exbehavior{Maintain attention on length constraint throughout}\\
\exeffect{Respects word limit despite early mention}
\end{examplebox}

\headfooter{\statusobs}{focus (L), topic-relevance (M), completion-stabilization (F)}

%-----------------------------------------------------------------------------
\subsubsection{(F) Implicit-RAG Routing Heads}
\label{head:implicit-rag}

\noindent\depthinfo{0.90--0.98} | \litnames{implicit-RAG head, knowledge-routing head, rag-routing head}

\begin{functiondesc}
Route attention to knowledge-bearing context portions mimicking retrieval-augmented generation patterns without explicit retrieval. Identify and prioritize factual, knowledge-dense segments that should ground response. Recognize quoted material, factual statements, and authoritative sources. Selectively attend to information that should be treated as retrieved knowledge.
\end{functiondesc}

\begin{attentionbox}
\attstrong{Factual statements, quoted material, authoritative sources}\\
\attweak{Opinions, questions, conversational elements}\\
\attreacts{Citation markers, factual density, authoritative tone}
\end{attentionbox}

\begin{ablationbox}
\textbf{Expected ablation:} Moderate decrease in context utilization. Model relies on parametric knowledge rather than provided information. 20--30\% reduction in grounding to specific context. Less effective use of quoted material.
\end{ablationbox}

\begin{examplebox}
\exinput{``Document: `GDP grew 3.2\% in Q3.' What was growth rate?''}\\
\exbehavior{Strongly attend to quoted factual content}\\
\exeffect{Ground answer: ``3.2\%'' vs. hallucination}
\end{examplebox}

\headfooter{\statusobs}{global-attention (F), fact (M)}
