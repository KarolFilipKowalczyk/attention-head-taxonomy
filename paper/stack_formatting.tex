%=============================================================================
\subsection{Output Formatting \& Rewrite Stack}
\label{sec:formatting-stack}

\textbf{Stack overview:} Enforce output schemas, structure responses according to format requirements, perform final rewriting. Ensure outputs conform to JSON, XML, lists, or other structured formats.

%-----------------------------------------------------------------------------
\subsubsection{(L) Output-Schema Heads}
\label{head:output-schema}

\noindent\depthinfo{0.65--0.82} | \litnames{output-schema head, JSON-format head, XML head, YAML head}

\begin{functiondesc}
Enforce adherence to specified output schemas and format requirements. When instructed to produce JSON, XML, YAML, or structured formats, promote conformance to required structure. Attend to format specifications and bias token generation toward schema-compliant outputs. Enforce required fields, proper nesting, correct syntax, format-specific conventions.
\end{functiondesc}

\begin{attentionbox}
\attstrong{Format specifications, schema definitions, structure requirements}\\
\attweak{Content independent of format, semantic meaning}\\
\attreacts{JSON/XML/YAML keywords, structure instructions}
\end{attentionbox}

\begin{ablationbox}
\textbf{Expected ablation:} Significant increase in format violations. 30--50\% more syntax errors, missing fields, improper nesting. Model falls back to prose even when structure requested. Partial compensation through instruction-following.
\end{ablationbox}

\begin{examplebox}
\exinput{``Return JSON with fields `name', `age', `city'''}\\
\exbehavior{Attend to JSON requirement and fields}\\
\exeffect{\texttt{\{"name": "...", "age": ..., "city": "..."\}}}
\end{examplebox}

\headfooter{\statuswell}{instruction (E), list-structure (L), format-consistency (F)}

%-----------------------------------------------------------------------------
\subsubsection{(L) List-Structure Heads}
\label{head:list-structure}

\noindent\depthinfo{0.68--0.85} | \litnames{list-structure head, enumeration head, list head}

\begin{functiondesc}
Manage generation and formatting of lists: numbered, bullet points, nested enumerations. Ensure proper list syntax, consistent formatting, appropriate indentation, logical organization. Track list state (currently in list, depth level, item number). Generate appropriate list markers. Coordinate with delimiter and boundary heads.
\end{functiondesc}

\begin{attentionbox}
\attstrong{List markers, enumeration patterns, item boundaries}\\
\attweak{Prose content, non-list structures}\\
\attreacts{Numbered/bulleted list requests, ``first'', ``second''}
\end{attentionbox}

\begin{ablationbox}
\textbf{Expected ablation:} Moderate degradation in list formatting. 20--30\% increase in inconsistent numbering, missing markers, poor nesting. Lists devolve into prose. Reduced ability to maintain structure across long enumerations.
\end{ablationbox}

\begin{examplebox}
\exinput{``List three programming languages and uses''}\\
\exbehavior{Generate structured list with consistent formatting}\\
\exeffect{``1. Python - ...\textbackslash n2. JavaScript - ...\textbackslash n3. Java - ...''}
\end{examplebox}

\headfooter{\statuswell}{delimiter (E), boundary (E), output-schema (L)}

%-----------------------------------------------------------------------------
\subsubsection{(L) Key--Value Pairing Heads}
\label{head:key-value}

\noindent\depthinfo{0.70--0.88} | \litnames{key-value head, attribute-pairing head, object head}

\begin{functiondesc}
Manage key-value relationships in structured data. Promote proper pairing of attributes with values. Maintain awareness of which values correspond to which keys. Promote proper syntax (colons, equals signs). Handle nested key-value structures. Prevent key-value mismatches. Work with output-schema heads for format enforcement.
\end{functiondesc}

\begin{attentionbox}
\attstrong{Keys, values, pairing syntax, attribute names}\\
\attweak{Unstructured text, list items without key-value}\\
\attreacts{Dictionary structures, configuration syntax}
\end{attentionbox}

\begin{ablationbox}
\textbf{Expected ablation:} Moderate increase in key-value errors. 20--30\% degradation in structured data quality. Mismatched keys and values, syntax errors, confusion about pairings. Reduced JSON, YAML quality.
\end{ablationbox}

\begin{examplebox}
\exinput{``Create config with server=`localhost' and port=8080''}\\
\exbehavior{Maintain proper key-value pairing}\\
\exeffect{\texttt{\{server: "localhost", port: 8080\}}}
\end{examplebox}

\headfooter{\statusobs}{output-schema (L), structural-block (L), format-consistency (F)}

%-----------------------------------------------------------------------------
\subsubsection{(L) Structural-Block Heads}
\label{head:structural-block}

\noindent\depthinfo{0.72--0.88} | \litnames{structural-block head, code-block head, fence head}

\begin{functiondesc}
Organize output into coherent structural blocks: paragraphs, code blocks, quoted sections, delimited units. Manage block boundaries. Promote proper opening and closing of blocks. Maintain block-level organization. Coordinate with delimiter heads for block markers and sectioning heads for hierarchical organization.
\end{functiondesc}

\begin{attentionbox}
\attstrong{Block boundaries, structural markers, content-type transitions}\\
\attweak{Within-block content, uniform text}\\
\attreacts{Block instructions, content-type changes}
\end{attentionbox}

\begin{ablationbox}
\textbf{Expected ablation:} Moderate reduction in structural quality. 20--30\% increase in unclear boundaries, content-type mixing, malformed code blocks. Reduced clarity in outputs requiring multiple content types.
\end{ablationbox}

\begin{examplebox}
\exinput{``Explain sorting with code example''}\\
\exbehavior{Organize into prose block, then code block}\\
\exeffect{Explanation paragraph, then \texttt{```python...```}}
\end{examplebox}

\headfooter{\statusobs}{list-structure (L), delimiter (E), output-schema (L)}

%-----------------------------------------------------------------------------
\subsubsection{(F) Format-Consistency Heads}
\label{head:format-consistency}

\noindent\depthinfo{0.88--0.97} | \litnames{format-consistency head, rewrite head, polish head}

\begin{functiondesc}
Perform final-stage formatting consistency enforcement and rewriting. Ensure formatting choices (indentation, capitalization, punctuation, syntax) remain consistent throughout response. Catch and correct formatting inconsistencies. Rephrase awkward constructions. Improve word choice. Fix minor grammatical issues. Enhance readability. Suppress redundancies, improve flow. Operate late enough to see full output pattern. Act as final editing pass.
\end{functiondesc}

\begin{attentionbox}
\attstrong{Previously generated patterns, consistency violations, quality issues}\\
\attweak{Novel content, already high-quality content}\\
\attreacts{Format inconsistencies, style violations, awkward constructions, redundancies}
\end{attentionbox}

\begin{ablationbox}
\textbf{Expected ablation:} Moderate increase in format inconsistency. 15--25\% reduction in output polish. Mixed indentation, inconsistent capitalization. More awkward phrasings, grammatical rough spots. Functional but less polished. Partial compensation through earlier generation.
\end{ablationbox}

\begin{examplebox}
\exinput{[Long response with mixed list styles]}\\
\exbehavior{Detect inconsistent formatting, enforce unified style}\\
\exeffect{All lists use same marker style consistently}
\end{examplebox}

\headfooter{\statuswell}{output-schema (L), brand-compliance (F), completion-stabilization (F)}

%-----------------------------------------------------------------------------
\subsubsection{(F) Completion-Stabilization Heads}
\label{head:completion-stabilization}

\noindent\depthinfo{0.92--0.99} | \litnames{completion-stabilization head, stopping head, termination head}

\begin{functiondesc}
Manage completion of generation. Determine when output is sufficiently complete and should terminate. Prevent premature stopping (cutting off mid-thought) and excessive continuation (rambling beyond task completion). Monitor generation progress against task requirements. Signal when objectives met. Trigger natural stopping points, proper conclusions, or continuation when more content needed.
\end{functiondesc}

\begin{attentionbox}
\attstrong{Task completion signals, generation progress, stopping points}\\
\attweak{Mid-generation content, continuing thoughts}\\
\attreacts{Task fulfillment, natural conclusions, query satisfaction}
\end{attentionbox}

\begin{ablationbox}
\textbf{Expected ablation:} Moderate increase in length control issues. 20--30\% more premature stops or excessive continuations. Difficulty recognizing task completion. Outputs feel incomplete or unnecessarily verbose.
\end{ablationbox}

\begin{examplebox}
\exinput{``Explain photosynthesis briefly''}\\
\exbehavior{Monitor brief explanation is complete, trigger stop}\\
\exeffect{Stop after concise explanation vs. excessive detail}
\end{examplebox}

\headfooter{\statusobs}{format-consistency (F), instruction (E), task-mode (M)}
