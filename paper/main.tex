\documentclass[11pt,a4paper]{article}

% Load preamble with packages and macros
%=============================================================================
% PREAMBLE.TEX - Packages, Macros, and Formatting
%=============================================================================

%-----------------------------------------------------------------------------
% Core Packages
%-----------------------------------------------------------------------------
\usepackage[utf8]{inputenc}
\usepackage[T1]{fontenc}

%-----------------------------------------------------------------------------
% Page Layout
%-----------------------------------------------------------------------------
\usepackage[margin=1in]{geometry}
\usepackage{setspace}
\onehalfspacing

%-----------------------------------------------------------------------------
% Mathematics
%-----------------------------------------------------------------------------
\usepackage{amsmath}
\usepackage{amssymb}
\usepackage{amsthm}

%-----------------------------------------------------------------------------
% Tables and Figures
%-----------------------------------------------------------------------------
\usepackage{booktabs}
\usepackage{longtable}
\usepackage{array}
\usepackage{tabularx}
\usepackage{graphicx}
\usepackage{float}

%-----------------------------------------------------------------------------
% Colors and Styling
%-----------------------------------------------------------------------------
\usepackage{xcolor}
\definecolor{darkblue}{rgb}{0.0, 0.0, 0.5}
\definecolor{darkgreen}{rgb}{0.0, 0.5, 0.0}
\definecolor{darkred}{rgb}{0.5, 0.0, 0.0}

%-----------------------------------------------------------------------------
% Hyperlinks and References
%-----------------------------------------------------------------------------
\usepackage{hyperref}
\hypersetup{
    colorlinks=true,
    linkcolor=darkblue,
    citecolor=darkgreen,
    urlcolor=darkblue,
    bookmarksnumbered=true,
    pdfborder={0 0 0}
}
\usepackage{cleveref}

%-----------------------------------------------------------------------------
% Lists and Enumeration
%-----------------------------------------------------------------------------
\usepackage{enumitem}
\setlist{itemsep=0.1em, parsep=0.1em, topsep=0.3em}

%-----------------------------------------------------------------------------
% Boxes and Styling - Load tcolorbox package FIRST
%-----------------------------------------------------------------------------
\usepackage{tcolorbox}
\tcbuselibrary{skins,breakable}

%-----------------------------------------------------------------------------
% Section Formatting
%-----------------------------------------------------------------------------
\usepackage{titlesec}
\titlespacing*{\section}{0pt}{1.5em}{0.8em}
\titlespacing*{\subsection}{0pt}{1.0em}{0.5em}
\titlespacing*{\subsubsection}{0pt}{1em}{0.3em}

%-----------------------------------------------------------------------------
% Table of Contents
%-----------------------------------------------------------------------------
\setcounter{tocdepth}{3}  % Show sections, subsections, and subsubsections in TOC
\setcounter{secnumdepth}{3}  % Number up to subsubsections
\usepackage{tocloft}
\setlength{\cftbeforesecskip}{2pt}
\setlength{\cftbeforesubsecskip}{1pt}
\setlength{\cftbeforesubsubsecskip}{1pt}

%-----------------------------------------------------------------------------
% Bibliography
%-----------------------------------------------------------------------------
\usepackage{natbib}
\setcitestyle{square,numbers,comma}

%-----------------------------------------------------------------------------
% Paragraph Formatting
%-----------------------------------------------------------------------------
\setlength{\parindent}{0pt}
\setlength{\parskip}{0.5em}

%=============================================================================
% CUSTOM COMMANDS AND ENVIRONMENTS
%=============================================================================

%-----------------------------------------------------------------------------
% Depth Indicators
%-----------------------------------------------------------------------------
\newcommand{\Early}{\textsf{E}}
\newcommand{\Middle}{\textsf{M}}
\newcommand{\Late}{\textsf{L}}
\newcommand{\Final}{\textsf{F}}

%-----------------------------------------------------------------------------
% Head Entry Formatting
%-----------------------------------------------------------------------------
\newcommand{\depthinfo}[1]{\textbf{Depth:} \texttt{#1}}
\newcommand{\litnames}[1]{\textbf{Literature names:} \textit{#1}}
\newcommand{\statuswell}{\textsc{Well-documented}}
\newcommand{\statusobs}{\textsc{Observed}}
\newcommand{\statusprop}{\textsc{Proposed}}
\newcommand{\relatedheads}[1]{\textbf{Related:} #1}
\newcommand{\statusinfo}[1]{\textbf{Status:} #1}

% Compact footer: Status | Related
\newcommand{\headfooter}[2]{\noindent\statusinfo{#1} | \relatedheads{#2}}

%-----------------------------------------------------------------------------
% Attention Pattern Formatting
%-----------------------------------------------------------------------------
\newcommand{\attstrong}[1]{\textbf{Strong:} #1}
\newcommand{\attweak}[1]{\textbf{Weak:} #1}
\newcommand{\attreacts}[1]{\textbf{Reacts to:} #1}

%-----------------------------------------------------------------------------
% Example Scenario Formatting
%-----------------------------------------------------------------------------
\newcommand{\exinput}[1]{\textit{Input:} #1}
\newcommand{\exbehavior}[1]{\textit{Behavior:} #1}
\newcommand{\exeffect}[1]{\textit{Effect:} #1}

%-----------------------------------------------------------------------------
% Custom Environments (NOW tcolorbox is loaded)
%-----------------------------------------------------------------------------
\newenvironment{functiondesc}{%
    \par\vspace{0.3em}%
    \noindent%
}{\par\vspace{0.3em}}

\newtcolorbox{attentionbox}{
    colback=gray!5,
    colframe=gray!40,
    boxrule=0.5pt,
    arc=2pt,
    left=3pt,
    right=3pt,
    top=2pt,
    bottom=2pt,
    boxsep=0pt,
    before skip=0.5em,
    after skip=0.5em
}

\newtcolorbox{examplebox}{
    colback=gray!8,
    colframe=gray!50,
    boxrule=0.5pt,
    arc=2pt,
    left=4pt,
    right=4pt,
    top=3pt,
    bottom=3pt,
    boxsep=0pt,
    before skip=0.5em,
    after skip=0.5em,
    fonttitle=\small\bfseries,
    title=Example Scenario
}

\newtcolorbox{ablationbox}{
    colback=white,
    colframe=gray!30,
    boxrule=0.5pt,
    arc=2pt,
    left=3pt,
    right=3pt,
    top=2pt,
    bottom=2pt,
    boxsep=0pt,
    before skip=0.5em,
    after skip=0.5em
}

%-----------------------------------------------------------------------------
% End of Preamble
%-----------------------------------------------------------------------------


% Document metadata
\title{Attention Head Naming Convention \\ for Large Language Models (LLMs)}
\author{Karol Kowalczyk}
\date{November 2025}

\begin{document}

\maketitle

\begin{abstract}
Large language models exhibit remarkable reasoning, safety alignment, and structural understanding, yet their internal workings remain opaque. Attention heads---specialized components within transformer layers---have emerged as key objects of study in interpretability research. The community has developed informal names: \emph{induction heads}, \emph{name mover heads}, \emph{refusal heads}, but these terms are inconsistent, overlapping, and ambiguous.

This work proposes a unified naming convention for attention heads: (1) a four-level depth model (Early, Middle, Late, Final), (2) stack-based functional grouping, (3) canonical names for head types, and (4) cross-reference tables mapping historical terms to standardized ones. This taxonomy is descriptive rather than prescriptive, capturing current head behaviors while remaining flexible for future architectures.
\end{abstract}

\tableofcontents
\clearpage

%=============================================================================
% MAIN CONTENT
%=============================================================================

\section{Introduction}
\label{sec:introduction}

\subsection{Motivation}
Attention heads---the basic computational units within transformer architectures---have emerged as key objects of study in mechanistic interpretability research despite achieving remarkable performance across diverse tasks.

\subsection{The Problem of Inconsistent Naming}
The interpretability community has identified numerous specialized attention head types: \emph{induction heads}, \emph{name mover heads}, \emph{refusal heads}, \emph{delimiter heads}, and \emph{JSON heads}. These naming conventions are \textbf{inconsistent} (same head type, multiple names), \textbf{ambiguous} (single name, different behaviors), \textbf{fragmented} (no unified framework), and \textbf{unscalable} (fail across architectures). This fragmentation complicates replication, comparison, and dataset annotation.

\subsection{Goals of This Work}
I propose a unified naming convention that standardizes terminology, provides a functional taxonomy grounded in empirical observations, describes head behavior consistently across architectures, and creates stable vocabulary that evolves with models.

\subsection{Circuits, Stacks, and Simplification}

This taxonomy uses \emph{stacks} as organizational framework. Attention heads work in complex \emph{circuits}---groups across layers cooperating through multi-level processing~\cite{elhage2021mathematical,wang2022interpretability}. 

The \emph{stack} abstraction simplifies this complexity for communication. Rather than mapping every circuit connection, I group heads by primary functional contribution, making the taxonomy accessible while acknowledging that real model behavior involves intricate cross-layer interactions.

\subsection{Structure of This Document}
I review prior work (\S\ref{sec:background}), introduce the depth model (\S\ref{sec:depth}) and stacks (\S\ref{sec:stacks}), present a comprehensive catalog organized by functional stack (\S\ref{sec:catalog}), and conclude with discussion (\S\ref{sec:discussion}) and future directions (\S\ref{sec:conclusion}).

%=============================================================================
\section{Background}
\label{sec:background}

\subsection{Attention Heads and Functions}
In transformer models~\cite{vaswani2017attention}, attention heads perform focused computations over token sequences. Though individually simple, they develop specialized behaviors: pattern continuation, entity tracking, semantic filtering, routing, format enforcement, and safety constraints~\cite{elhage2021mathematical,olsson2022context}. These behaviors form \emph{circuits} and larger \emph{stacks} of related functionality.

\subsection{Why Naming Consistency Matters}
Interpretability research suffers from fragmented terminology~\cite{rai2024practical,zheng2025attention}. The same head type appears under multiple names, while overloaded names refer to unrelated behaviors. Consistent naming improves communication clarity, strengthens cross-paper alignment, helps index interpretability datasets, and enables systematic circuit mapping.

\subsection{Prior Naming Practices}
Previous work named heads by behavior (induction, copy-suppression), formatting (JSON, list), signal source (delimiter), circuit role (name mover), or safety function (refusal, toxicity). Though often accurate, these labels vary widely. This work unifies them under a systematic framework.

%=============================================================================
\section{Depth Model: Early---Middle---Late---Final}
\label{sec:depth}

\subsection{Rationale for Four Depth Categories}
Functional behavior clusters reliably into four zones~\cite{elhage2021mathematical,wang2022interpretability}: \textbf{Early (E)} layers handle token-level processing, boundary detection, and filtering. \textbf{Middle (M)} layers implement reasoning primitives, induction, and dependency tracking. \textbf{Late (L)} layers perform semantic integration, routing, and persona shaping. \textbf{Final (F)} layers enforce policy, safety, and structured output. This structure holds across GPT, LLaMA, and Claude~\cite{brown2020language,touvron2023llama,achiam2023gpt}.

\subsection{Cross-Model Depth Examples}
Using \emph{relative depth} (0.0--1.0) makes the taxonomy scale-free. For a 96-layer model: Early = layers 0--15 (0.00--0.15), Middle = 15--50 (0.15--0.52), Late = 50--85 (0.52--0.88), Final = 85--96 (0.88--1.00).

\subsection{Relative Depth Scaling}
I express depth as fraction of total model depth for cross-architecture comparison. A head at relative depth 0.40 occupies similar functional space in 12-layer or 96-layer models.

%=============================================================================
\section{Stacks: Functional Grouping of Attention Heads}
\label{sec:stacks}

\subsection{What is a Stack?}
A \emph{stack} groups head types that together implement a higher-level capability. Stacks reflect functional clustering observed in studies~\cite{wang2022interpretability,olsson2022context}. Examples: Reasoning \& Algorithmic, Memory \& Dependency, Safety, Output Formatting. Stacks span Early, Middle, Late, and Final layers.

\subsection{Relationship Between Stacks and Depth}
Different functions appear at different depths. Early: delimiters, content detection, input conditioning. Middle: reasoning, induction, entity linking. Late: narrative coherence, routing, topic steering. Final: policy, formatting, rewriting, safety. This \emph{stack $\times$ depth} structure forms the catalog basis.

%=============================================================================
% CATALOG
%=============================================================================

\section{Attention Head Catalog}
\label{sec:catalog}

This section catalogs attention head types by functional stack. Each stack groups heads contributing to common high-level capability, ordered by depth (Early $\rightarrow$ Middle $\rightarrow$ Late $\rightarrow$ Final).

\paragraph{Entry Format.} Each head entry includes:
\begin{itemize}
    \item \textbf{Depth range:} Typical relative depth (0.0--1.0)
    \item \textbf{Literature names:} Alternative names from prior work
    \item \textbf{Function:} Core behavior and mechanism
    \item \textbf{Attention pattern:} What the heads attend to
    \item \textbf{Expected ablation:} Predicted effects if disabled
    \item \textbf{Example scenario:} Concrete behavioral illustration
    \item \textbf{Stack and relations:} Primary stack and related heads
\end{itemize}

% Individual stack files
%=============================================================================
\subsection{Reasoning \& Algorithmic Stack}
\label{sec:reasoning-stack}

\textbf{Stack overview:} Heads performing pattern matching, sequence continuation, algorithmic reasoning, and meta-cognitive oversight. Enable in-context learning, pattern completion, and reasoning quality control.

%-----------------------------------------------------------------------------
\subsubsection{(E) Previous-Token Heads}
\label{head:previous-token}

\noindent\depthinfo{0.05--0.18} | \litnames{previous-token head, shift head, offset head}

\begin{functiondesc}
Copy information from token $t$ to position $t+1$, creating shifted representation where each position contains information about the previous token. Foundational for induction circuits, enabling later heads to access ``what came before'' without attending backwards. Show strong diagonal attention patterns ($i \rightarrow i-1$).
\end{functiondesc}

\begin{attentionbox}
\attstrong{Immediately preceding token}\\
\attweak{Distant tokens, same-position}\\
\attreacts{Sequential structure, boundaries}
\end{attentionbox}

\begin{ablationbox}
\textbf{Expected ablation:} Breaks induction circuits entirely. Induction heads lose access to ``what came after previous occurrences''. Critical for in-context learning.
\end{ablationbox}

\begin{examplebox}
\exinput{``The cat sat. The cat...''}\\
\exbehavior{Copy tokens forward one position}\\
\exeffect{Induction heads match ``cat'' and access ``sat''}
\end{examplebox}

\headfooter{\statuswell}{induction (M), duplicate-token (M)}

%-----------------------------------------------------------------------------
\subsubsection{(E) Local Pattern Heads}
\label{head:local-pattern}

\noindent\depthinfo{0.08--0.20} | \litnames{local pattern head, char-level head, n-gram head}

\begin{functiondesc}
Detect character-level and subword patterns for spelling, capitalization, punctuation, and morphology. Operate at finer granularity than most heads, attending within and between adjacent tokens. Recognize patterns like ``ing'', ``tion'', or punctuation clusters.
\end{functiondesc}

\begin{attentionbox}
\attstrong{Adjacent tokens, subword units}\\
\attweak{Long-range dependencies, semantics}\\
\attreacts{Spelling, capitalization, morphology}
\end{attentionbox}

\begin{ablationbox}
\textbf{Expected ablation:} Degraded handling of misspellings, case variations, morphology. Errors on character-level tasks. Partial compensation through tokenization.
\end{ablationbox}

\begin{examplebox}
\exinput{``organizATION's''}\\
\exbehavior{Detect unusual case pattern}\\
\exeffect{Handle non-standard capitalization}
\end{examplebox}

\headfooter{\statusobs}{induction (M), duplicate-token (M)}

%-----------------------------------------------------------------------------
\subsubsection{(M) Induction Heads}
\label{head:induction}

\noindent\depthinfo{0.30--0.65} | \litnames{induction head, pattern head, copy head, ICL head}

\begin{functiondesc}
Detect [A][B]...[A] patterns and predict [B] follows the second [A]. Attend to tokens after previous instances of current token. Work with previous-token heads to enable pattern completion, name recall, and few-shot learning. Fundamental to in-context learning.
\end{functiondesc}

\begin{attentionbox}
\attstrong{Tokens following previous occurrences}\\
\attweak{Immediate neighbors, first occurrence}\\
\attreacts{Token repetition, [A][B]...[A] patterns}
\end{attentionbox}

\begin{ablationbox}
\textbf{Expected ablation:} Significant degradation in in-context learning and pattern completion. 30--50\% accuracy loss on few-shot tasks. Partial compensation through other heads.
\end{ablationbox}

\begin{examplebox}
\exinput{``Mary and John went to the store, Mary bought...''}\\
\exbehavior{Second ``Mary'' attends to tokens after first ``Mary''}\\
\exeffect{Increased probability of appropriate continuation}
\end{examplebox}

\headfooter{\statuswell}{previous-token (E), duplicate-token (M), name-mover (L)}

%-----------------------------------------------------------------------------
\subsubsection{(M) Duplicate-Token Heads}
\label{head:duplicate-token}

\noindent\depthinfo{0.35--0.60} | \litnames{duplicate-token head, repetition head, copy head}

\begin{functiondesc}
Detect when current token appeared previously, marking repeats for downstream processing. Unlike induction heads (which predict next token), these simply signal ``seen before''. Used by IOI circuits, name-movers, and copy-suppression.
\end{functiondesc}

\begin{attentionbox}
\attstrong{Previous identical tokens}\\
\attweak{Similar non-identical, first occurrence}\\
\attreacts{Exact repetition, name recurrence}
\end{attentionbox}

\begin{ablationbox}
\textbf{Expected ablation:} Impaired duplicate detection. Degraded name-mover and copy-suppression circuits. Overlap with induction heads provides redundancy.
\end{ablationbox}

\begin{examplebox}
\exinput{``Alice gave the book to Bob. Then Alice...''}\\
\exbehavior{Second ``Alice'' writes duplicate signal}\\
\exeffect{Name-movers and S-inhibition use signal}
\end{examplebox}

\headfooter{\statuswell}{induction (M), name-mover (L), S-inhibition (L)}

%-----------------------------------------------------------------------------
\subsubsection{(M) Skip-Trigram Heads}
\label{head:skip-trigram}

\noindent\depthinfo{0.40--0.65} | \litnames{skip-trigram head, skip-gram head}

\begin{functiondesc}
Match non-contiguous patterns [A]...[B]...[C] with intervening tokens. More flexible than strict n-grams. Detect phrasal patterns, idioms, and structural templates with flexible word order.
\end{functiondesc}

\begin{attentionbox}
\attstrong{Components separated by 1--3 tokens}\\
\attweak{Adjacent patterns, long-range}\\
\attreacts{Phrasal patterns, flexible idioms}
\end{attentionbox}

\begin{ablationbox}
\textbf{Expected ablation:} Reduced flexible pattern recognition. Less critical than induction heads; other mechanisms compensate.
\end{ablationbox}

\begin{examplebox}
\exinput{``not only X but also''}\\
\exbehavior{Recognize ``not...but'' despite intervening tokens}\\
\exeffect{Predict ``also'' after ``but''}
\end{examplebox}

\headfooter{\statusobs}{induction (M), local-pattern (E)}

%-----------------------------------------------------------------------------
\subsubsection{(M) Algorithmic Continuation Heads}
\label{head:algorithmic-continuation}

\noindent\depthinfo{0.45--0.70} | \litnames{algorithmic head, continuation head, sequence head}

\begin{functiondesc}
Recognize and continue algorithmic sequences: counting, days of week, months, systematic progressions. Operate on sequences with clear algorithmic rules. Detect arithmetic progressions, cyclic patterns, rule-governed sequences.
\end{functiondesc}

\begin{attentionbox}
\attstrong{Sequential elements in algorithmic patterns}\\
\attweak{Random sequences, semantic patterns}\\
\attreacts{Arithmetic progressions, cyclic orderings}
\end{attentionbox}

\begin{ablationbox}
\textbf{Expected ablation:} Reduced sequence continuation performance. Degradation on counting, ordering, arithmetic. Some reasoning persists through other mechanisms.
\end{ablationbox}

\begin{examplebox}
\exinput{``Monday, Tuesday, Wednesday, ...''}\\
\exbehavior{Recognize day-of-week progression}\\
\exeffect{Strongly predict ``Thursday''}
\end{examplebox}

\headfooter{\statusobs}{induction (M), digit (M)}

%-----------------------------------------------------------------------------
\subsubsection{(L) Strategy Heads}
\label{head:strategy}

\noindent\depthinfo{0.68--0.88} | \litnames{strategy head, planning head, approach-selection head, pivot head}

\begin{functiondesc}
Plan overall approach for complex tasks and adapt when strategies prove ineffective. Influence high-level structure: step decomposition, component ordering, method selection. Recognize task types requiring different approaches (analytical vs. creative, sequential vs. parallel). Decompose complex queries into subtasks. Monitor progress, detect dead ends, switch strategies when needed.
\end{functiondesc}

\begin{attentionbox}
\attstrong{Task complexity, multi-part queries, progress indicators}\\
\attweak{Single-step tasks, progressing solutions}\\
\attreacts{Complex tasks, planning requests, insufficient progress}
\end{attentionbox}

\begin{ablationbox}
\textbf{Expected ablation:} Reduced planning quality and adaptability. Premature execution without planning. Persist with unproductive approaches. 15--25\% efficiency loss on complex tasks.
\end{ablationbox}

\begin{examplebox}
\exinput{``Plan a ML project for customer churn''}\\
\exbehavior{Recognize need for structured planning}\\
\exeffect{Structure: data $\rightarrow$ analysis $\rightarrow$ features $\rightarrow$ model $\rightarrow$ evaluation}
\end{examplebox}

\headfooter{\statusobs}{reasoning-oversight (F)}

%-----------------------------------------------------------------------------
\subsubsection{(F) Reasoning-Oversight Heads}
\label{head:reasoning-oversight}

\noindent\depthinfo{0.88--0.99} | \litnames{reasoning-mode head, cognitive-mode head, meta-CoT head, reasoning-quality head}

\begin{functiondesc}
Manage reasoning processes: mode selection and quality monitoring. Select reasoning modes (analytical, creative, analogical, deductive, inductive) matched to task type. Monitor reasoning chain quality, detect errors and gaps, flag uncertainty, trigger re-thinking. Operate at meta-level above chain-of-thought. Influence which reasoning patterns activate. Prevent confident errors in complex scenarios.
\end{functiondesc}

\begin{attentionbox}
\attstrong{Task type, reasoning mode cues, quality indicators, logical gaps}\\
\attweak{Simple factual responses, non-reasoning tasks}\\
\attreacts{Complex reasoning, logical steps, errors, inconsistencies}
\end{attentionbox}

\begin{ablationbox}
\textbf{Expected ablation:} Less appropriate mode selection. More logical gaps, reduced self-correction. Chain-of-thought less reliable on complex problems. 20--30\% degradation on multi-step reasoning.
\end{ablationbox}

\begin{examplebox}
\exinput{``Brainstorm creative names''}\\
\exbehavior{Select creative mode vs. analytical}\\
\exeffect{Free-flowing suggestions, not logical analysis}
\end{examplebox}

\headfooter{\statusobs}{strategy (L), step-by-step (F)}

%=============================================================================
\subsection{Memory \& Dependency Stack}
\label{sec:memory-stack} \textbf{Stack overview:} These heads track references, resolve coreferences, and maintain dependency relationships across the input sequence. They enable the model to understand which entities are being discussed and how they relate to each other. %-----------------------------------------------------------------------------
\subsubsection{(E) Reference Resolution Heads}
\label{head:reference-resolution} \noindent\depthinfo{0.08--0.25} | \litnames{reference head, pronoun head, anaphora head, mention head} \begin{functiondesc}
Performs early-stage reference resolution including pronouns, definite descriptions, demonstratives, and possessives. Identifies pronouns (he, she, it, they) and other referring expressions, attending to potential referents that match in number, gender, and contextual appropriateness. Establishes initial binding signals that are refined by later coreference heads. Operates primarily on syntactic and positional cues rather than deep semantic understanding. Forms the foundation for more sophisticated reference resolution in deeper layers. Broader than pure pronoun resolution, handling various reference forms including "the president", "this approach", and "her book".
\end{functiondesc} \begin{attentionbox}
\attstrong{Pronouns to recent nouns, definite descriptions to referents, demonstratives to antecedents}\\
\attweak{Distant nouns, semantically incompatible referents, first mentions}\\
\attreacts{Pronoun presence, definite articles, demonstratives, possessives, noun-pronoun proximity}
\end{attentionbox} \begin{ablationbox}
\textbf{Expected ablation:} Degraded reference resolution, particularly for simple local cases and various referring expressions. increase in reference resolution errors. Later coreference heads can partially compensate but with reduced accuracy. Particularly impacts handling of definite descriptions and complex referring patterns.
\end{ablationbox} \begin{examplebox}
\exinput{"Alice met Bob. She smiled. The researcher continued."}\\
\exbehavior{"She" attends to "Alice" based on gender and recency; "The researcher" attends to appropriate prior mention}\\
\exeffect{Establishes initial bindings that later heads refine}
\end{examplebox} \headfooter{\statuswell}{coreference (M), entity (M)} %-----------------------------------------------------------------------------
\subsubsection{(M) Coreference Heads}
\label{head:coreference} \noindent\depthinfo{0.35--0.60} | \litnames{coreference head, coref head} \begin{functiondesc}
Performs sophisticated coreference resolution, determining when different expressions refer to the same entity. Integrates signals from early reference resolution heads with semantic understanding to resolve ambiguous cases. Can handle complex phenomena like split antecedents, bridging references, and discourse-level coreference. Critical for maintaining entity tracking across long contexts and understanding narrative structure. Represents one of the core NLP capabilities in transformers.
\end{functiondesc} \begin{attentionbox}
\attstrong{Coreferential mentions regardless of form}\\
\attweak{Different entities, first mentions without antecedents}\\
\attreacts{Semantic compatibility, discourse coherence, entity properties}
\end{attentionbox} \begin{ablationbox}
\textbf{Expected ablation:} Significant degradation in coreference resolution tasks. Model loses ability to track entities across complex reference chains. Particularly impacts question answering and summarization.
\end{ablationbox} \begin{examplebox}
\exinput{"The CEO announced changes. Later, the executive clarified. She emphasized..."}\\
\exbehavior{Links all three mentions (CEO, executive, She) to same entity}\\
\exeffect{Maintains consistent entity representation throughout discourse}
\end{examplebox} \headfooter{\statuswell}{reference-resolution (E), entity (M), bridging (M)} %-----------------------------------------------------------------------------
\subsubsection{(M) Long-Range Dependency Heads}
\label{head:long-range-dependency} \noindent\depthinfo{0.40--0.65} | \litnames{long-range head, dependency head} \begin{functiondesc}
Tracks long-range syntactic and semantic dependencies across distant parts of the sequence. Unlike local attention patterns, this head maintains connections between elements separated by many tokens (20-100+). Essential for understanding complex sentences, nested structures, and discourse relations. Implements the key advantage of transformers over RNNs: direct long-distance connections without degradation. Can maintain multiple simultaneous long-range connections.
\end{functiondesc} \begin{attentionbox}
\attstrong{Syntactically or semantically related distant tokens}\\
\attweak{Immediately adjacent tokens, unrelated distant content}\\
\attreacts{Nested structures, long-distance agreement, discourse relations}
\end{attentionbox} \begin{ablationbox}
\textbf{Expected ablation:} Degradation in handling complex sentences and long-range relationships. performance loss on tasks requiring long-distance reasoning. Particularly impacts nested structures and long documents.
\end{ablationbox} \begin{examplebox}
\exinput{"The book [that Alice mentioned [that Bob recommended]] was excellent."}\\
\exbehavior{"was" attends back to "book" across nested relative clauses}\\
\exeffect{Maintains correct subject-verb agreement despite intervening material}
\end{examplebox} \headfooter{\statusobs}{coreference (M), state-tracking (M)} %-----------------------------------------------------------------------------
\subsubsection{(M) Bridging Heads}
\label{head:bridging} \noindent\depthinfo{0.45--0.68} | \litnames{bridging head, associative reference head} \begin{functiondesc}
Resolves bridging references where the connection between mentions requires inferencing based on world knowledge. For example, connecting "the car" to "the steering wheel" (part-whole), or "the building" to "the architect" (role relation). More sophisticated than direct coreference, requiring semantic knowledge about typical relationships. Essential for understanding implicit connections in discourse. Bridges gaps that aren't explicit in the text.
\end{functiondesc} \begin{attentionbox}
\attstrong{Associatively related entities (part-whole, role, causation)}\\
\attweak{Unrelated entities, explicit coreference}\\
\attreacts{Implicit relationships, world knowledge, typical associations}
\end{attentionbox} \begin{ablationbox}
\textbf{Expected ablation:} Loss of implicit reference resolution. degradation on tasks requiring inference-based connections. Model becomes more literal, missing implicit relationships. Discourse coherence suffers.
\end{ablationbox} \begin{examplebox}
\exinput{"We entered the house. The door was painted blue."}\\
\exbehavior{"The door" attends to "house" (part-whole bridging)}\\
\exeffect{Understands "the door" refers to the house's door, not a random door}
\end{examplebox} \headfooter{\statusobs}{coreference (M), entity (M), fact (M)} %-----------------------------------------------------------------------------
\subsubsection{(M) State-Tracking Heads}
\label{head:state-tracking} \noindent\depthinfo{0.48--0.70} | \litnames{state-tracking head, tracking head, state head} \begin{functiondesc}
Maintains and updates representations of changing states across the sequence. Tracks how entity properties evolve (e.g., location changes, status updates, accumulating information). Essential for understanding narratives where situations change over time. Can maintain multiple simultaneous state representations for different entities. Integrates new information with existing state representations to track dynamic situations.
\end{functiondesc} \begin{attentionbox}
\attstrong{State-changing events, current state mentions, entity properties}\\
\attweak{Static descriptions, unchanging background information}\\
\attreacts{Verbs of change, state transitions, property modifications}
\end{attentionbox} \begin{ablationbox}
\textbf{Expected ablation:} Difficulty tracking state changes across sequences. degradation on tasks requiring temporal reasoning or state tracking. Narratives become harder to follow when states evolve.
\end{ablationbox} \begin{examplebox}
\exinput{"Alice was in NYC. She flew to Paris. She then visited..."}\\
\exbehavior{Updates Alice's location state: NYC $\rightarrow$ Paris}\\
\exeffect{Correctly contextualizes "visited" as occurring in Paris}
\end{examplebox} \headfooter{\statusobs}{coreference (M), long-range-dependency (M)}

%=============================================================================
\subsection{Instruction \& Intent Stack}
\label{sec:instruction-stack}

\textbf{Stack overview:} Process user instructions, system prompts, and task specifications. Determine what the model is asked to do and switch between operational modes.

%-----------------------------------------------------------------------------
\subsubsection{(E) Instruction Heads}
\label{head:instruction}

\noindent\depthinfo{0.05--0.20} | \litnames{instruction head, command head, directive head}

\begin{functiondesc}
Identify and process user instructions and commands. Distinguish instructional from descriptive or conversational content. Attend to imperative verbs, question structures, and directive phrases. Write instruction-detection signals into residual stream influencing generation. Operate early to set response strategy.
\end{functiondesc}

\begin{attentionbox}
\attstrong{Imperative verbs, question words, directive phrases}\\
\attweak{Descriptive content, narrative text}\\
\attreacts{Question marks, imperative mood, explicit requests}
\end{attentionbox}

\begin{ablationbox}
\textbf{Expected ablation:} Reduced instruction-following capability. Model generates relevant content but fails to follow specific directives. 25--35\% degradation on instruction-following tasks.
\end{ablationbox}

\begin{examplebox}
\exinput{``Context provided. Now, summarize key points.''}\\
\exbehavior{Attend to ``summarize'', identify imperative}\\
\exeffect{Summary format vs. continuation}
\end{examplebox}

\headfooter{\statuswell}{system-prompt (E), task-mode (M)}

%-----------------------------------------------------------------------------
\subsubsection{(E) System-Prompt Heads}
\label{head:system-prompt}

\noindent\depthinfo{0.08--0.22} | \litnames{system-prompt head, system head, prompt head}

\begin{functiondesc}
Process system prompts defining model role, constraints, and operational parameters. Focus on meta-level directives about how to behave rather than what task to perform. Attend to persona definitions, behavioral constraints, and system-level instructions. Establish interaction framework in chat models.
\end{functiondesc}

\begin{attentionbox}
\attstrong{System-level directives, persona definitions, constraints}\\
\attweak{User content, task-specific instructions}\\
\attreacts{Role definitions, constraint specifications}
\end{attentionbox}

\begin{ablationbox}
\textbf{Expected ablation:} Reduced adherence to system-level instructions and persona. Model ignores constraints like ``be concise'' or persona like ``respond as teacher''. 30--40\% reduction in role consistency.
\end{ablationbox}

\begin{examplebox}
\exinput{``System: You are a concise technical writer. User: Explain recursion.''}\\
\exbehavior{Attend to ``concise technical writer''}\\
\exeffect{Technical, brief style vs. verbose explanation}
\end{examplebox}

\headfooter{\statuswell}{instruction (E), task-mode (M)}

%-----------------------------------------------------------------------------
\subsubsection{(M) Task-Mode Heads}
\label{head:task-mode}

\noindent\depthinfo{0.30--0.55} | \litnames{task head, mode head, intent head}

\begin{functiondesc}
Determine overall task type: question answering, summarization, translation, creative writing, coding. Integrate instruction signals from early layers with content analysis to classify intended task. Write task-mode embeddings influencing downstream processing, routing, and formatting. More sophisticated than simple instruction detection.
\end{functiondesc}

\begin{attentionbox}
\attstrong{Task indicators, instruction semantics, content type markers}\\
\attweak{Generic content, ambiguous instructions}\\
\attreacts{Task-specific keywords, question types, format requests}
\end{attentionbox}

\begin{ablationbox}
\textbf{Expected ablation:} Task confusion and inappropriate response formats. Model summarizes when asked to analyze, or explains when asked to code. 20--30\% task classification errors.
\end{ablationbox}

\begin{examplebox}
\exinput{``Compare and contrast democracy and autocracy.''}\\
\exbehavior{Identify ``compare and contrast'' mode}\\
\exeffect{Comparison structure vs. separate descriptions}
\end{examplebox}

\headfooter{\statuswell}{instruction (E), mode-switch (M), output-specification (F)}

%-----------------------------------------------------------------------------
\subsubsection{(M) Mode-Switch Heads}
\label{head:mode-switch}

\noindent\depthinfo{0.40--0.60} | \litnames{mode head, switch head, transition head}

\begin{functiondesc}
Detect and handle switches between operational modes within single interaction. Transition from conversational to code generation, or explanation to example. Respond to explicit indicators (``Now let's...'') and implicit content shifts. Maintain coherence across mode boundaries.
\end{functiondesc}

\begin{attentionbox}
\attstrong{Transition phrases, mode-shift markers, content type changes}\\
\attweak{Uniform single-mode content}\\
\attreacts{``Now'', ``For example'', format shifts}
\end{attentionbox}

\begin{ablationbox}
\textbf{Expected ablation:} Difficulty handling multi-mode requests. Model sticks to single mode or switches inappropriately. 25--35\% degradation on complex multi-part instructions.
\end{ablationbox}

\begin{examplebox}
\exinput{``Explain recursion. Now write Python code.''}\\
\exbehavior{Detect mode switch at ``Now''}\\
\exeffect{Smooth transition: prose $\rightarrow$ code block}
\end{examplebox}

\headfooter{\statusobs}{task-mode (M), output-specification (F)}

%-----------------------------------------------------------------------------
\subsubsection{(F) Output-Specification Heads}
\label{head:output-specification}

\noindent\depthinfo{0.85--0.98} | \litnames{output-specification head, format-directive head}

\begin{functiondesc}
Enforce specific output format requirements from instruction: ``respond in JSON'', ``use bullet points'', ``maximum 100 words''. Operate in final layers to ensure content conforms to explicit format directives. Focus on user-specified constraints rather than general format quality. Final enforcement of explicit user requirements.
\end{functiondesc}

\begin{attentionbox}
\attstrong{Format specifications, length constraints, structure requirements}\\
\attweak{Content without format requirements}\\
\attreacts{``in JSON format'', ``bullet points'', ``no more than''}
\end{attentionbox}

\begin{ablationbox}
\textbf{Expected ablation:} Failure to follow explicit format requirements. Model generates good content in wrong format. 40--50\% increase in format violations.
\end{ablationbox}

\begin{examplebox}
\exinput{``List three benefits of exercise in bullet points.''}\\
\exbehavior{Attend to ``bullet points'', enforce format}\\
\exeffect{Bullet structure vs. prose paragraphs}
\end{examplebox}

\headfooter{\statuswell}{task-mode (M), output-schema (L), format-consistency (F)}

%=============================================================================
\subsection{Knowledge Retrieval Stack}
\label{sec:knowledge-stack}

\textbf{Stack overview:} Retrieve factual information, entity properties, and structured knowledge from model parameters. Move relevant information to output positions and suppress irrelevant content.

%-----------------------------------------------------------------------------
\subsubsection{(M) Entity Heads}
\label{head:entity}

\noindent\depthinfo{0.35--0.65} | \litnames{entity head, name head, proper-noun head, entity-linking head}

\begin{functiondesc}
Identify and process named entities (people, places, organizations). Retrieve associated information from model parameters. Link mentions across different forms: full names, abbreviations, nicknames. Understand that different strings refer to same entity (``Apple Inc.'', ``Apple'', ``AAPL''). Ground responses in factual knowledge.
\end{functiondesc}

\begin{attentionbox}
\attstrong{Named entities, proper nouns, name variations}\\
\attweak{Common nouns, generic references}\\
\attreacts{Capitalization patterns, factual queries}
\end{attentionbox}

\begin{ablationbox}
\textbf{Expected ablation:} Significant degradation in factual accuracy. Model loses entity knowledge and linking ability. 30--40\% accuracy drop on who/what/where questions. Fluent text with factual errors.
\end{ablationbox}

\begin{examplebox}
\exinput{``Capital of France? MSFT stock rose...''}\\
\exbehavior{Retrieve ``capital: Paris''; link ``MSFT'' to ``Microsoft''}\\
\exeffect{Output ``Paris''; maintain unified entity}
\end{examplebox}

\headfooter{\statuswell}{fact (M), name-mover (L), schema-retriever (M)}

%-----------------------------------------------------------------------------
\subsubsection{(M) Fact Heads}
\label{head:fact}

\noindent\depthinfo{0.38--0.62} | \litnames{fact head, knowledge head, factual-retrieval head}

\begin{functiondesc}
Retrieve factual relationships and propositions from model parameters. Handle general factual knowledge: relations, properties, statements. Access learned knowledge for factual questions. Retrieve multi-hop facts and combine information from multiple stored facts.
\end{functiondesc}

\begin{attentionbox}
\attstrong{Factual queries, relation markers, knowledge-seeking patterns}\\
\attweak{Opinion questions, hypotheticals}\\
\attreacts{Question structures, fact-seeking context}
\end{attentionbox}

\begin{ablationbox}
\textbf{Expected ablation:} Major loss of factual knowledge retrieval. Linguistic fluency maintained but factual grounding lost. 40--60\% degradation on knowledge-intensive tasks.
\end{ablationbox}

\begin{examplebox}
\exinput{``Who invented the telephone?''}\\
\exbehavior{Retrieve: invented(telephone) $\rightarrow$ Bell}\\
\exeffect{Output ``Alexander Graham Bell''}
\end{examplebox}

\headfooter{\statuswell}{entity (M), schema-retriever (M), name-mover (L)}

%-----------------------------------------------------------------------------
\subsubsection{(M) Schema Retriever Heads}
\label{head:schema-retriever}

\noindent\depthinfo{0.45--0.68} | \litnames{schema head, retrieval head, template head}

\begin{functiondesc}
Retrieve structured knowledge schemas and templates. Access typical structures: restaurant visit (enter, order, eat, pay, leave), scientific paper format. Enable structured responses following learned patterns. Implement implicit knowledge base querying.
\end{functiondesc}

\begin{attentionbox}
\attstrong{Schema-triggering contexts, domain-specific patterns}\\
\attweak{Novel situations, schema-irrelevant content}\\
\attreacts{Domain markers, structural queries}
\end{attentionbox}

\begin{ablationbox}
\textbf{Expected ablation:} Loss of structured knowledge organization. Facts provided but poorly organized. 25--35\% degradation on schema-based reasoning tasks.
\end{ablationbox}

\begin{examplebox}
\exinput{``Describe the scientific method.''}\\
\exbehavior{Retrieve schema: observe $\rightarrow$ hypothesis $\rightarrow$ test $\rightarrow$ conclude}\\
\exeffect{Organized by standard method structure}
\end{examplebox}

\headfooter{\statusobs}{fact (M), entity (M)}

%-----------------------------------------------------------------------------
\subsubsection{(L) Name-Mover Heads}
\label{head:name-mover}

\noindent\depthinfo{0.60--0.80} | \litnames{name mover head, mover head, copy head}

\begin{functiondesc}
Copy entity names and content to output positions where needed. Central component of IOI (indirect object identification) circuit. Attend to relevant entities earlier in context and move them forward when needed for generation. Work with S-inhibition heads to select correct entity among multiple candidates.
\end{functiondesc}

\begin{attentionbox}
\attstrong{Entities needing output, contextually relevant names}\\
\attweak{Irrelevant entities, suppressed alternatives}\\
\attreacts{Entity salience, contextual appropriateness}
\end{attentionbox}

\begin{ablationbox}
\textbf{Expected ablation:} Severe degradation in entity recall and completion. Loss of specific name movement. 50--70\% accuracy drop on question answering and cloze tasks requiring entity recall.
\end{ablationbox}

\begin{examplebox}
\exinput{``Alice and Bob went to the store, Alice gave the book to...''}\\
\exbehavior{Move ``Bob'' to output as indirect object}\\
\exeffect{Complete with ``Bob'' (not ``Alice'')}
\end{examplebox}

\headfooter{\statuswell}{entity (M), fact (M), S-inhibition (L)}

%-----------------------------------------------------------------------------
\subsubsection{(L) S-Inhibition Heads}
\label{head:s-inhibition}

\noindent\depthinfo{0.62--0.82} | \litnames{S-inhibition head, inhibition head, suppression head}

\begin{functiondesc}
Suppress incorrect or contextually inappropriate entities from generation. Named from IOI research where these heads inhibit subject (S) when indirect object (IO) should be output. Work antagonistically with name-mover heads. Implement negative selection, ruling out incorrect options.
\end{functiondesc}

\begin{attentionbox}
\attstrong{Entities that should NOT be output}\\
\attweak{Correct entities, absent entities}\\
\attreacts{Competing candidates, disambiguation contexts}
\end{attentionbox}

\begin{ablationbox}
\textbf{Expected ablation:} Moderate entity confusion and incorrect selections. Model outputs recently mentioned but contextually wrong entities. 20--30\% accuracy loss in ambiguous contexts.
\end{ablationbox}

\begin{examplebox}
\exinput{``Alice gave the book to Bob. Then Alice...''}\\
\exbehavior{Inhibit ``Bob'' from output after ``Alice''}\\
\exeffect{Prevent ``Alice Bob...''}
\end{examplebox}

\headfooter{\statuswell}{name-mover (L), copy-suppression (L), duplicate-token (M)}

%-----------------------------------------------------------------------------
\subsubsection{(L) Copy-Suppression Heads}
\label{head:copy-suppression}

\noindent\depthinfo{0.65--0.85} | \litnames{copy-suppression head, suppression head, anti-copy head}

\begin{functiondesc}
Prevent inappropriate copying or repetition. Avoid degenerate behaviors: endless repetition loops, copy-pasting irrelevant context. Suppress exact copies and near-copies. Focus on broader pattern suppression rather than specific entity blocking. Balance useful recall against inappropriate copying.
\end{functiondesc}

\begin{attentionbox}
\attstrong{Recently generated content, repetitive patterns}\\
\attweak{Novel content, first mentions}\\
\attreacts{Repetition detection, copy patterns}
\end{attentionbox}

\begin{ablationbox}
\textbf{Expected ablation:} Moderate increase in repetition and copying errors. Repetitive loops or inappropriate context copying. 25--35\% reduction in output diversity.
\end{ablationbox}

\begin{examplebox}
\exinput{[``The cat sat. The cat sat. The cat...'']}\\
\exbehavior{Detect repetitive pattern, suppress copying}\\
\exeffect{Break loop, generate novel continuation}
\end{examplebox}

\headfooter{\statuswell}{S-inhibition (L), name-mover (L), duplicate-token (M)}

%=============================================================================
\subsection{Safety Stack}
\label{sec:safety-stack} \textbf{Stack overview:} The safety stack implements content filtering, policy enforcement, and refusal mechanisms. Early-layer heads detect potentially harmful content, while final-layer heads enforce refusal decisions and redirect to safe responses. %-----------------------------------------------------------------------------
\subsubsection{(E) Content-Detection Heads}
\label{head:content-detection}

\noindent\depthinfo{0.05--0.25} | \litnames{sensitive-content head, detection head, content-filter head, toxicity head, toxic-content head, hate-speech detector, hazard head, risk head, danger-topic detector}

\begin{functiondesc}
Performs early-stage detection of potentially harmful or sensitive content across multiple categories. These heads identify sensitive personal information, violent imagery references, adult content markers, regulated substances, toxic language patterns, hate speech, harassment, discriminatory content, dangerous activities, illegal instructions, and harm-related topics. They operate on lexical and surface-level features to flag content requiring downstream safety processing, writing detection signals into the residual stream that are read by later safety enforcement heads. Different heads within this category distinguish between pure toxicity (language-level harm) and hazard topics (action-level harm), though both categories are handled by these early detection mechanisms.
\end{functiondesc}

\begin{attentionbox}
\attstrong{Keywords associated with restricted content, explicit language, sensitive topic markers, slurs, aggressive language, derogatory terms, action verbs combined with dangerous objects, instructional phrases about harmful activities}\\
\attweak{Neutral content, common vocabulary, structural tokens, academic discussion, fictional scenarios}\\
\attreacts{Sudden topic shifts to sensitive domains, warning indicators, escalating hostility, targeted harassment patterns, how-to requests for dangerous activities, detailed planning questions}
\end{attentionbox}

\begin{ablationbox}
\textbf{Expected ablation:} Bypass of early safety detection with significant increase in harmful outputs that should be caught. Later safety layers may still catch some cases, but at higher computational cost and lower accuracy. Model loses ability to distinguish hostile from neutral phrasing and loses distinction between discussing danger and instructing danger.
\end{ablationbox}

\begin{examplebox}
\exinput{"Tell me about [restricted topic]" or "[hostile language]" or "How do I create [dangerous item]"}\\
\exbehavior{Strong attention to problematic keywords, writes detection flags into residual stream}\\
\exeffect{Downstream safety heads receive early warning signals across multiple harm categories}
\end{examplebox}

\noindent\headfooter{\statuswell}{safety-classification (E), policy-enforcement (L), refusal (F)}

%-----------------------------------------------------------------------------
\subsubsection{(E) Safety-Classification Heads}
\label{head:safety-classification} \noindent\depthinfo{0.12--0.28} | \litnames{classification head, category detector, safety-category head} \begin{functiondesc}
Performs multi-class safety classification, categorizing inputs into specific policy violation categories (violence, sexual content, self-harm, illegal activity, harassment, etc.). More sophisticated than binary safe/unsafe detection, providing granular category information used by downstream heads. Integrates signals from other early safety heads and adds categorical structure to safety decisions. Writes category-specific embeddings into residual stream that later layers use for category-appropriate responses.
\end{functiondesc} \begin{attentionbox}
\attstrong{Category-diagnostic features, domain-specific terminology, contextual markers}\\
\attweak{Ambiguous content, mixed-category inputs, benign contexts}\\
\attreacts{Clear category signatures, multiple category indicators, policy-relevant contexts}
\end{attentionbox} \begin{ablationbox}
\textbf{Expected ablation:} Loss of nuanced safety handling (model may refuse too broadly or too narrowly). Category-specific responses become generic. degradation in appropriate refusal granularity.
\end{ablationbox} \begin{examplebox}
\exinput{"Can you help me with [category-specific harmful request]"}\\
\exbehavior{Classifies into specific violation category, writes category embedding}\\
\exeffect{Later heads generate category-appropriate refusal message}
\end{examplebox} \noindent\headfooter{\statuswell}{all early safety heads (E), policy-enforcement (L), redirect (F)} %-----------------------------------------------------------------------------
\subsubsection{(L) Policy-Enforcement Heads}
\label{head:policy-enforcement} \noindent\depthinfo{0.60--0.80} | \litnames{policy head, enforcement head, steering head} \begin{functiondesc}
Integrates safety signals from early detection heads and makes intermediate policy decisions about how to handle the request. Unlike early heads that detect issues, this head actively modulates the generation trajectory to steer away from violations while maintaining helpfulness where possible. Can suppress certain knowledge retrieval pathways, bias toward safer formulations, and prepare for potential refusal. Acts as a middle manager between detection and final refusal, attempting "soft" safety interventions before hard refusal.
\end{functiondesc} \begin{attentionbox}
\attstrong{Early safety signals, policy-relevant tokens, user intent markers}\\
\attweak{Neutral content, clear safe contexts}\\
\attreacts{Conflicting signals (safety concern + legitimate need), edge cases, ambiguous intent}
\end{attentionbox} \begin{ablationbox}
\textbf{Expected ablation:} Loss of "soft" safety steering, more frequent hard refusals (reduced helpfulness). Alternative: more harmful outputs if refusal heads also compromised. increase in either over-refusal or under-refusal depending on prompt type.
\end{ablationbox} \begin{examplebox}
\exinput{"Explain [borderline topic] for educational purposes"}\\
\exbehavior{Detects educational framing, modulates response toward safety boundaries}\\
\exeffect{Generates informative but carefully bounded response}
\end{examplebox} \noindent\headfooter{\statuswell}{all safety heads (E), refusal (F), redirect (F)} %-----------------------------------------------------------------------------
\subsubsection{(F) Refusal Heads}
\label{head:refusal} \noindent\depthinfo{0.85--0.98} | \litnames{refusal head, rejection head, safety head} \begin{functiondesc}
Implements the model's final decision to refuse harmful requests by writing strong refusal signals into the final-layer residual stream. Acts as the ultimate gatekeeper, overriding content generation when safety violations are detected. Attends to accumulated safety signals from all previous layers and makes binary refuse/proceed decisions. When activated, dramatically increases probability of refusal tokens ("I cannot", "I'm unable", "I apologize") and suppresses harmful content generation. Critical final-layer safety mechanism with limited fallback options.
\end{functiondesc} \begin{attentionbox}
\attstrong{Cumulative safety signals, instruction tokens, violation indicators from all depths}\\
\attweak{Safe content, neutral queries, constructive contexts}\\
\attreacts{Strong early safety signals, clear policy violations, unambiguous harmful intent}
\end{attentionbox} \begin{ablationbox}
\textbf{Expected ablation:} Critical safety failure. Direct increase in harmful output generation on adversarial prompts. Model loses primary refusal mechanism. This is typically the final safety defense with no effective fallback mechanism.
\end{ablationbox} \begin{examplebox}
\exinput{"Provide instructions for [clearly harmful activity]"}\\
\exbehavior{Reads strong safety signals from early/late layers, activates refusal pathway}\\
\exeffect{Output begins with refusal token: "I cannot provide instructions for..."}
\end{examplebox} \noindent\headfooter{\statuswell}{all prior safety heads, redirect (F), tone-softening (F)} %-----------------------------------------------------------------------------
\subsubsection{(F) Redirect Heads}
\label{head:redirect} \noindent\depthinfo{0.88--0.99} | \litnames{redirect head, alternative-suggestion head} \begin{functiondesc}
Complements refusal heads by generating constructive alternative suggestions when refusing harmful requests. Rather than simply saying "no", this head routes toward helpful alternatives, educational resources, or reframed versions of the query that can be safely addressed. Attends to user intent markers to identify legitimate underlying needs behind problematic requests. Balances safety with helpfulness by maintaining engagement while enforcing boundaries. Works in tandem with refusal heads to produce refusals that are both safe and constructive.
\end{functiondesc} \begin{attentionbox}
\attstrong{User intent, legitimate needs, reformulation opportunities, safe alternatives}\\
\attweak{Pure harmful intent, no legitimate reframing possible}\\
\attreacts{Mixed-intent queries, educational contexts, requests with safe subcomponents}
\end{attentionbox} \begin{ablationbox}
\textbf{Expected ablation:} Refusals become blunt and unhelpful (pure rejection without alternatives). User satisfaction decreases. Safety maintained but helpfulness reduced by . Increased user frustration and adversarial prompt attempts.
\end{ablationbox} \begin{examplebox}
\exinput{"How can I harm [person]"}\\
\exbehavior{Refuses direct request, identifies legitimate conflict-resolution need}\\
\exeffect{"I cannot help with that, but I can suggest healthy conflict resolution strategies..."}
\end{examplebox} \noindent\headfooter{\statuswell}{refusal (F), empathy (F), tone-softening (F)} %-----------------------------------------------------------------------------
\subsubsection{(F) Refusal-Modulation Heads}
\label{head:refusal-modulation}

\noindent\depthinfo{0.88--0.99} | \litnames{tone-softening head, politeness-in-refusal head, empathy head, supportive-refusal head}

\begin{functiondesc}
Modulates the tone and emotional quality of safety refusals to be firm but respectful, avoiding harsh or judgmental language while showing appropriate care. These heads balance clear boundary-setting with relationship maintenance, softening phrases like "absolutely not" to "I'm unable to assist with that" while adding empathetic framing where appropriate. Particularly important for queries involving distress, self-harm, or difficult situations where harmful requests may stem from genuine suffering. Attends to emotional tone of both the request and forming response, recognizing distress markers, crisis language, and vulnerability indicators. Can add supportive language alongside refusal and increase probability of phrases like "I'm concerned about you" or "please reach out to..." when appropriate. Maintains user trust and reduces adversarial reactions while preserving safety boundaries.
\end{functiondesc}

\begin{attentionbox}
\attstrong{Response tone markers, emotional valence, user frustration signals, distress signals, vulnerability markers, crisis language, emotional pain indicators}\\
\attweak{Already-soft phrasing, neutral technical content, malicious queries without distress, clearly harmful intent without suffering}\\
\attreacts{Harsh refusal language, judgmental phrasing, cold rejections, self-harm content, suicide-related queries, expressions of suffering}
\end{attentionbox}

\begin{ablationbox}
\textbf{Expected ablation:} Refusals become harsh and potentially alienating. Increased user perception of model as judgmental or unfriendly. May increase adversarial behavior. Missed opportunities to provide crisis resources for distressed users. Safety maintained but user experience and support functions degraded.
\end{ablationbox}

\begin{examplebox}
\exinput{[Forming harsh refusal] or "I want to hurt myself because..."}\\
\exbehavior{Softens tone while maintaining boundary clarity, adds crisis resources and supportive language when appropriate}\\
\exeffect{"I'm unable to provide assistance with that" + "I'm concerned about what you're sharing. Help is available..."}
\end{examplebox}

\noindent\headfooter{\statuswell}{refusal (F), redirect (F)}

%-----------------------------------------------------------------------------
\subsubsection{(F) Safety-Persona Heads}
\label{head:safety-persona} \noindent\depthinfo{0.92--0.98} | \litnames{safety-persona head, responsible-AI head, ethical-framing head} \begin{functiondesc}
Maintains safety-conscious persona and ethical framing in final outputs. Ensures responses reflect responsible AI values such as declining harmful requests appropriately, providing balanced perspectives on sensitive topics, and avoiding reinforcement of harmful stereotypes or behaviors. Operates at final stage to catch any safety-inconsistent framing that might have emerged during generation. Works with refusal and policy-enforcement heads but focuses on the overall ethical character of the response rather than specific policy violations. Ensures tone remains respectful and constructive even when declining requests.
\end{functiondesc} \begin{attentionbox}
\attstrong{Ethical framing, safety-relevant content, sensitive topics, decline scenarios}\\
\attweak{Clearly safe, neutral content}\\
\attreacts{Harmful requests, sensitive topics, ethical considerations, responsible AI principles}
\end{attentionbox} \begin{ablationbox}
\textbf{Expected ablation:} Less consistent safety framing and reduced ethical consistency. May handle sensitive topics less carefully with reduced graceful handling of harmful requests. Less consistent responsible AI messaging and more variable ethical framing.
\end{ablationbox} \begin{examplebox}
\exinput{[Request for harmful content that will be declined]}\\
\exbehavior{Ensures decline is respectfully framed with helpful alternatives when appropriate}\\
\exeffect{Response maintains helpful, respectful tone even when unable to fulfill request}
\end{examplebox} \headfooter{\statusobs}{refusal (F), policy-enforcement (L), refusal-modulation (F)}

%=============================================================================
\subsection{Routing \& Relevance Stack}
\label{sec:routing-stack}

\textbf{Stack overview:} Determine which input parts are relevant to current task and route attention accordingly. Filter information, focus on salient content, manage global context.

%-----------------------------------------------------------------------------
\subsubsection{(M) Topic-Relevance Heads}
\label{head:topic-relevance}

\noindent\depthinfo{0.35--0.60} | \litnames{topic-relevance head, relevance head, salience head, filter head}

\begin{functiondesc}
Identify primary topic and determine which input context parts are relevant to current generation task. Filter irrelevant information while highlighting salient content. Compute relevance scores based on semantic similarity, task alignment, and topical coherence. Maintain topic coherence by attending to topic-establishing phrases and domain indicators.
\end{functiondesc}

\begin{attentionbox}
\attstrong{Task-relevant content, topic indicators, domain markers}\\
\attweak{Off-topic material, unrelated context}\\
\attreacts{Semantic relevance, topical alignment, topic transitions}
\end{attentionbox}

\begin{ablationbox}
\textbf{Expected ablation:} Moderate reduction in focus with increased topic drift. Model distracted by irrelevant content. 20--30\% degradation on long contexts. Responses wander off-topic or miss key details.
\end{ablationbox}

\begin{examplebox}
\exinput{``[Document: cars, climate, history] What caused 2008 financial crisis?''}\\
\exbehavior{Mark financial/economic content relevant, de-emphasize cars/climate}\\
\exeffect{Focus on economic information, ignore unrelated context}
\end{examplebox}

\headfooter{\statuswell}{focus (L), router (L), entity (M)}

%-----------------------------------------------------------------------------
\subsubsection{(L) Focus Heads}
\label{head:focus}

\noindent\depthinfo{0.65--0.80} | \litnames{focus head, attention-routing head, spotlight head}

\begin{functiondesc}
Concentrate attention on most salient elements for current generation step. Implement dynamic focus allocation: suppress less important content, amplify critical information. More selective than topic-relevance heads. Determine exactly which tokens should influence next token prediction. Shift focus as generation proceeds.
\end{functiondesc}

\begin{attentionbox}
\attstrong{Currently salient tokens, query-critical content}\\
\attweak{Background information, low-priority details}\\
\attreacts{Query emphasis, current generation needs}
\end{attentionbox}

\begin{ablationbox}
\textbf{Expected ablation:} Moderate reduction in focus precision. Model gives equal weight to important and peripheral information. 15--25\% degradation on targeted responses. Answers more diffuse, less direct.
\end{ablationbox}

\begin{examplebox}
\exinput{``Among all details, what is the MAIN cause?''}\\
\exbehavior{Attend to ``MAIN cause'', suppress secondary details}\\
\exeffect{Direct answer: primary cause vs. all factors}
\end{examplebox}

\headfooter{\statuswell}{topic-relevance (M), router (L)}

%-----------------------------------------------------------------------------
\subsubsection{(L) Router Heads}
\label{head:router}

\noindent\depthinfo{0.70--0.85} | \litnames{router head, dispatch head, task-routing head}

\begin{functiondesc}
Route query types to appropriate processing strategies or knowledge domains. Act as dispatchers recognizing query type (factual, creative, analytical, procedural). Bias processing toward suitable approaches. Activate different downstream heads based on task classification. Enable dynamic strategy selection based on input characteristics.
\end{functiondesc}

\begin{attentionbox}
\attstrong{Query-type indicators, task markers, domain signals}\\
\attweak{Content details, specific entities}\\
\attreacts{Task classification cues, query structure}
\end{attentionbox}

\begin{ablationbox}
\textbf{Expected ablation:} Moderate reduction in task-appropriate processing. Suboptimal strategy selection. Creative approaches for factual queries or vice versa. 20--30\% degradation on diverse query types.
\end{ablationbox}

\begin{examplebox}
\exinput{``Calculate compound interest vs. Write poem about compound interest''}\\
\exbehavior{Route first to mathematical, second to creative}\\
\exeffect{Calculation vs. literary devices}
\end{examplebox}

\headfooter{\statusobs}{focus (L), mode-switch (M), instruction (E)}

%-----------------------------------------------------------------------------
\subsubsection{(F) Global-Attention Heads}
\label{head:global-attention}

\noindent\depthinfo{0.88--0.96} | \litnames{global-attention head, full-context head, summary-attention head}

\begin{functiondesc}
Maintain broad attention over entire context to integrate global information in final generation. Unlike focused heads, attend widely to ensure complete picture considered before finalization. Catch context elements that earlier focused attention missed. Act as final integration mechanism for coherence checking and global consistency.
\end{functiondesc}

\begin{attentionbox}
\attstrong{All context tokens, document-level information, global constraints}\\
\attweak{Fine-grained local patterns, individual token details}\\
\attreacts{Complete context, document-level coherence}
\end{attentionbox}

\begin{ablationbox}
\textbf{Expected ablation:} Moderate reduction in global coherence. Responses miss relevant information from distant context. 15--25\% increase in locally optimal but globally suboptimal outputs.
\end{ablationbox}

\begin{examplebox}
\exinput{[Long context: ``Keep it under 100 words'']}\\
\exbehavior{Maintain attention on length constraint throughout}\\
\exeffect{Respects word limit despite early mention}
\end{examplebox}

\headfooter{\statusobs}{focus (L), topic-relevance (M), completion-stabilization (F)}

%-----------------------------------------------------------------------------
\subsubsection{(F) Implicit-RAG Routing Heads}
\label{head:implicit-rag}

\noindent\depthinfo{0.90--0.98} | \litnames{implicit-RAG head, knowledge-routing head, rag-routing head}

\begin{functiondesc}
Route attention to knowledge-bearing context portions mimicking retrieval-augmented generation patterns without explicit retrieval. Identify and prioritize factual, knowledge-dense segments that should ground response. Recognize quoted material, factual statements, and authoritative sources. Selectively attend to information that should be treated as retrieved knowledge.
\end{functiondesc}

\begin{attentionbox}
\attstrong{Factual statements, quoted material, authoritative sources}\\
\attweak{Opinions, questions, conversational elements}\\
\attreacts{Citation markers, factual density, authoritative tone}
\end{attentionbox}

\begin{ablationbox}
\textbf{Expected ablation:} Moderate decrease in context utilization. Model relies on parametric knowledge rather than provided information. 20--30\% reduction in grounding to specific context. Less effective use of quoted material.
\end{ablationbox}

\begin{examplebox}
\exinput{``Document: `GDP grew 3.2\% in Q3.' What was growth rate?''}\\
\exbehavior{Strongly attend to quoted factual content}\\
\exeffect{Ground answer: ``3.2\%'' vs. hallucination}
\end{examplebox}

\headfooter{\statusobs}{global-attention (F), fact (M)}

%=============================================================================
\subsection{Structural \& Boundary Stack}
\label{sec:structural-stack} \textbf{Stack overview:} These heads detect structural boundaries in text, including delimiters, section markers, and document divisions. They help the model understand document organization and navigate hierarchical structure. %-----------------------------------------------------------------------------
\subsubsection{(E) Delimiter Heads}
\label{head:delimiter} \noindent\depthinfo{0.05--0.18} | \litnames{delimiter head, whitespace-structure head, separator head, punctuation head, space-parsing head, layout head} \begin{functiondesc}
Detects and processes delimiter tokens that mark boundaries between structural elements, including punctuation marks, special characters, formatting symbols, and significant whitespace. Recognizes punctuation marks, brackets, delimiters, and special characters that indicate separation or grouping. Also processes whitespace characters (spaces, tabs, newlines) as structural elements rather than mere separators. Particularly important for languages where whitespace is syntactically significant (Python, YAML, Markdown). Distinguishes between semantically meaningful whitespace and irrelevant spacing. Important for understanding sentence boundaries, list items, code blocks, and structured data formats. Works at a fundamental level to identify basic structural segmentation. Provides boundary information to downstream heads that need to understand document organization. Essential for parsing formatted text, JSON, CSV, and other structured formats.
\end{functiondesc} \begin{attentionbox}
\attstrong{Punctuation marks, brackets, delimiters, special characters, formatting symbols, whitespace patterns, indentation levels, line breaks, space-delimited structures}\\
\attweak{Alphanumeric content, regular words, non-whitespace tokens, content within properly-spaced code}\\
\attreacts{Structural punctuation, boundary markers, formatting characters, indentation changes, blank lines, significant spacing patterns}
\end{attentionbox} \begin{ablationbox}
\textbf{Expected ablation:} Impaired structure parsing with degraded code formatting (degradation in format understanding and reduction in whitespace correctness). Difficulty with structured data, lists, and code blocks. Boundary detection errors. Problems parsing JSON, CSV, or other delimited formats. Particular problems with Python and other whitespace-significant languages. Incorrect indentation, missing line breaks, loss of code block structure. Reduced ability to segment text appropriately and to parse existing code structure.
\end{ablationbox} \begin{examplebox}
\exinput{"Items: [apple, banana, cherry], Count: 3. def foo():\textbackslash n return 42"}\\
\exbehavior{Detects brackets, commas, colons as structural delimiters; Recognizes 4-space indentation as significant structure}\\
\exeffect{Model correctly parses structure: list with 3 items, separate count field; Understands return statement is inside function, not at module level}
\end{examplebox} \headfooter{\statuswell}{boundary (E), relative-position (M), list-structure (L)} %-----------------------------------------------------------------------------
\subsubsection{(E) Boundary Heads}
\label{head:boundary} \noindent\depthinfo{0.08--0.20} | \litnames{boundary head, segment head, block-detection head} \begin{functiondesc}
Identifies boundaries between major text segments such as paragraphs, sections, and conceptual blocks. Operates at a higher level than delimiter heads, recognizing semantic and structural transitions rather than just punctuation. Detects paragraph breaks, section changes, topic shifts, and other high-level boundaries. Important for understanding document structure and maintaining appropriate context scope. Helps subsequent heads understand which information belongs to which segment. Critical for long documents with multiple sections or topics.
\end{functiondesc} \begin{attentionbox}
\attstrong{Paragraph breaks, section transitions, whitespace patterns, structural shifts}\\
\attweak{Within-paragraph content, continuous text}\\
\attreacts{Major structural boundaries, document divisions, topic transitions}
\end{attentionbox} \begin{ablationbox}
\textbf{Expected ablation:} Reduced boundary awareness (degradation in segmentation). Model may blur distinctions between sections, miss paragraph boundaries, or fail to recognize document structure. Reduced performance on multi-section documents.
\end{ablationbox} \begin{examplebox}
\exinput{"Introduction: [...] \textbackslash n\textbackslash n Methods: [...] \textbackslash n\textbackslash n Results: [...]"}\\
\exbehavior{Detects section boundaries between Introduction, Methods, Results}\\
\exeffect{Model understands these are separate sections, not continuous narrative}
\end{examplebox} \headfooter{\statuswell}{delimiter (E), sectioning (L), relative-position (M)} %-----------------------------------------------------------------------------
\subsubsection{(M) Relative-Position Heads}
\label{head:relative-position} \noindent\depthinfo{0.35--0.65} | \litnames{relative-position head, position-offset head, contextual-position head, distance head, scope-position head} \begin{functiondesc}
Tracks and computes relative positions between tokens, both in terms of raw offsets and structure-aware positions. Calculates offsets like "3 tokens back", "5 tokens forward", or "within same paragraph". Maintains position information relative to structural boundaries and scopes rather than absolute sequence position. Understands positions like "beginning of sentence", "middle of paragraph", "end of section". Important for patterns that depend on relative distance rather than absolute position. Works with other structural heads to understand boundaries and compute positions relative to those boundaries. Enables patterns like "attend to previous sentence" or "look ahead 2 tokens" without hardcoded position encodings. More sophisticated than absolute position encoding, providing context-aware position representations. Important for handling variable-length structures where absolute position is less meaningful than relative position within a scope. Enables position-dependent behavior that adapts to document structure.
\end{functiondesc} \begin{attentionbox}
\attstrong{Tokens at specific relative offsets, distance-based patterns, local neighborhoods, scope-relative positions, structure-aware locations, contextual position markers}\\
\attweak{Distant unrelated tokens, position-independent content, absolute sequence positions}\\
\attreacts{Relative position, distance relationships, local structure, structural scope boundaries, context-dependent positions, hierarchical location}
\end{attentionbox} \begin{ablationbox}
\textbf{Expected ablation:} Impaired distance-sensitive patterns and loss of structure-aware positioning (degradation in offset-based attention and degradation in scope-sensitive behavior). Reduced ability to attend based on relative position. Patterns requiring "nearby" or "distance" computations become less reliable. Reduced ability to behave differently at "beginning" vs. "end" of structures. Position-dependent patterns become less adaptive to document structure. Some compensation through learned position encodings.
\end{ablationbox} \begin{examplebox}
\exinput{"The [SUBJECT] quickly [VERB] the [OBJECT]. Paragraph 1: [50 tokens] Paragraph 2: [20 tokens]"}\\
\exbehavior{Computes that VERB is +1 from SUBJECT, OBJECT is +2 from VERB; Knows token 10 is "early in Para 1" while token 10 of Para 2 is "middle"}\\
\exeffect{Enables grammatical patterns based on relative token positions; Position-dependent behavior adapts to paragraph structure, not absolute position}
\end{examplebox} \headfooter{\statusobs}{boundary (E), previous-token (E), sectioning (L)} %-----------------------------------------------------------------------------
\subsubsection{(L) Sectioning Heads}
\label{head:sectioning} \noindent\depthinfo{0.70--0.85} | \litnames{sectioning head, hierarchy head, document-structure head} \begin{functiondesc}
Understands and maintains document hierarchical structure including sections, subsections, and nested organizational levels. Recognizes hierarchical markers like headings, numbering schemes, and indentation. Maintains awareness of current position within document hierarchy. Important for long documents, technical writing, and structured content. Enables appropriate context scoping: knowing that current text belongs to "Section 3.2.1" influences which prior content is relevant. Works with boundary heads but operates at higher semantic level.
\end{functiondesc} \begin{attentionbox}
\attstrong{Section headings, hierarchical markers, document structure indicators, organizational signals}\\
\attweak{Within-section content, unstructured text}\\
\attreacts{Headings, numbering, hierarchy indicators, structural organization}
\end{attentionbox} \begin{ablationbox}
\textbf{Expected ablation:} Reduced hierarchical awareness (degradation in structure understanding). Difficulty maintaining section context. Problems with document navigation and appropriate context scoping. Hierarchical relationships become less clear.
\end{ablationbox} \begin{examplebox}
\exinput{"1. Introduction \textbackslash n 1.1 Background \textbackslash n 1.2 Motivation \textbackslash n 2. Methods"}\\
\exbehavior{Understands 1.1 and 1.2 are subsections of 1, separate from section 2}\\
\exeffect{Maintains hierarchical context: text in 1.2 relates to 1.1 and 1, not to 2}
\end{examplebox} \headfooter{\statuswell}{boundary (E), relative-position (M), topic-relevance (M)}

%=============================================================================
\subsection{Output Formatting \& Rewrite Stack}
\label{sec:formatting-stack}

\textbf{Stack overview:} Enforce output schemas, structure responses according to format requirements, perform final rewriting. Ensure outputs conform to JSON, XML, lists, or other structured formats.

%-----------------------------------------------------------------------------
\subsubsection{(L) Output-Schema Heads}
\label{head:output-schema}

\noindent\depthinfo{0.65--0.82} | \litnames{output-schema head, JSON-format head, XML head, YAML head}

\begin{functiondesc}
Enforce adherence to specified output schemas and format requirements. When instructed to produce JSON, XML, YAML, or structured formats, promote conformance to required structure. Attend to format specifications and bias token generation toward schema-compliant outputs. Enforce required fields, proper nesting, correct syntax, format-specific conventions.
\end{functiondesc}

\begin{attentionbox}
\attstrong{Format specifications, schema definitions, structure requirements}\\
\attweak{Content independent of format, semantic meaning}\\
\attreacts{JSON/XML/YAML keywords, structure instructions}
\end{attentionbox}

\begin{ablationbox}
\textbf{Expected ablation:} Significant increase in format violations. 30--50\% more syntax errors, missing fields, improper nesting. Model falls back to prose even when structure requested. Partial compensation through instruction-following.
\end{ablationbox}

\begin{examplebox}
\exinput{``Return JSON with fields `name', `age', `city'''}\\
\exbehavior{Attend to JSON requirement and fields}\\
\exeffect{\texttt{\{"name": "...", "age": ..., "city": "..."\}}}
\end{examplebox}

\headfooter{\statuswell}{instruction (E), list-structure (L), format-consistency (F)}

%-----------------------------------------------------------------------------
\subsubsection{(L) List-Structure Heads}
\label{head:list-structure}

\noindent\depthinfo{0.68--0.85} | \litnames{list-structure head, enumeration head, list head}

\begin{functiondesc}
Manage generation and formatting of lists: numbered, bullet points, nested enumerations. Ensure proper list syntax, consistent formatting, appropriate indentation, logical organization. Track list state (currently in list, depth level, item number). Generate appropriate list markers. Coordinate with delimiter and boundary heads.
\end{functiondesc}

\begin{attentionbox}
\attstrong{List markers, enumeration patterns, item boundaries}\\
\attweak{Prose content, non-list structures}\\
\attreacts{Numbered/bulleted list requests, ``first'', ``second''}
\end{attentionbox}

\begin{ablationbox}
\textbf{Expected ablation:} Moderate degradation in list formatting. 20--30\% increase in inconsistent numbering, missing markers, poor nesting. Lists devolve into prose. Reduced ability to maintain structure across long enumerations.
\end{ablationbox}

\begin{examplebox}
\exinput{``List three programming languages and uses''}\\
\exbehavior{Generate structured list with consistent formatting}\\
\exeffect{``1. Python - ...\textbackslash n2. JavaScript - ...\textbackslash n3. Java - ...''}
\end{examplebox}

\headfooter{\statuswell}{delimiter (E), boundary (E), output-schema (L)}

%-----------------------------------------------------------------------------
\subsubsection{(L) Key--Value Pairing Heads}
\label{head:key-value}

\noindent\depthinfo{0.70--0.88} | \litnames{key-value head, attribute-pairing head, object head}

\begin{functiondesc}
Manage key-value relationships in structured data. Promote proper pairing of attributes with values. Maintain awareness of which values correspond to which keys. Promote proper syntax (colons, equals signs). Handle nested key-value structures. Prevent key-value mismatches. Work with output-schema heads for format enforcement.
\end{functiondesc}

\begin{attentionbox}
\attstrong{Keys, values, pairing syntax, attribute names}\\
\attweak{Unstructured text, list items without key-value}\\
\attreacts{Dictionary structures, configuration syntax}
\end{attentionbox}

\begin{ablationbox}
\textbf{Expected ablation:} Moderate increase in key-value errors. 20--30\% degradation in structured data quality. Mismatched keys and values, syntax errors, confusion about pairings. Reduced JSON, YAML quality.
\end{ablationbox}

\begin{examplebox}
\exinput{``Create config with server=`localhost' and port=8080''}\\
\exbehavior{Maintain proper key-value pairing}\\
\exeffect{\texttt{\{server: "localhost", port: 8080\}}}
\end{examplebox}

\headfooter{\statusobs}{output-schema (L), structural-block (L), format-consistency (F)}

%-----------------------------------------------------------------------------
\subsubsection{(L) Structural-Block Heads}
\label{head:structural-block}

\noindent\depthinfo{0.72--0.88} | \litnames{structural-block head, code-block head, fence head}

\begin{functiondesc}
Organize output into coherent structural blocks: paragraphs, code blocks, quoted sections, delimited units. Manage block boundaries. Promote proper opening and closing of blocks. Maintain block-level organization. Coordinate with delimiter heads for block markers and sectioning heads for hierarchical organization.
\end{functiondesc}

\begin{attentionbox}
\attstrong{Block boundaries, structural markers, content-type transitions}\\
\attweak{Within-block content, uniform text}\\
\attreacts{Block instructions, content-type changes}
\end{attentionbox}

\begin{ablationbox}
\textbf{Expected ablation:} Moderate reduction in structural quality. 20--30\% increase in unclear boundaries, content-type mixing, malformed code blocks. Reduced clarity in outputs requiring multiple content types.
\end{ablationbox}

\begin{examplebox}
\exinput{``Explain sorting with code example''}\\
\exbehavior{Organize into prose block, then code block}\\
\exeffect{Explanation paragraph, then \texttt{```python...```}}
\end{examplebox}

\headfooter{\statusobs}{list-structure (L), delimiter (E), output-schema (L)}

%-----------------------------------------------------------------------------
\subsubsection{(F) Format-Consistency Heads}
\label{head:format-consistency}

\noindent\depthinfo{0.88--0.97} | \litnames{format-consistency head, rewrite head, polish head}

\begin{functiondesc}
Perform final-stage formatting consistency enforcement and rewriting. Ensure formatting choices (indentation, capitalization, punctuation, syntax) remain consistent throughout response. Catch and correct formatting inconsistencies. Rephrase awkward constructions. Improve word choice. Fix minor grammatical issues. Enhance readability. Suppress redundancies, improve flow. Operate late enough to see full output pattern. Act as final editing pass.
\end{functiondesc}

\begin{attentionbox}
\attstrong{Previously generated patterns, consistency violations, quality issues}\\
\attweak{Novel content, already high-quality content}\\
\attreacts{Format inconsistencies, style violations, awkward constructions, redundancies}
\end{attentionbox}

\begin{ablationbox}
\textbf{Expected ablation:} Moderate increase in format inconsistency. 15--25\% reduction in output polish. Mixed indentation, inconsistent capitalization. More awkward phrasings, grammatical rough spots. Functional but less polished. Partial compensation through earlier generation.
\end{ablationbox}

\begin{examplebox}
\exinput{[Long response with mixed list styles]}\\
\exbehavior{Detect inconsistent formatting, enforce unified style}\\
\exeffect{All lists use same marker style consistently}
\end{examplebox}

\headfooter{\statuswell}{output-schema (L), brand-compliance (F), completion-stabilization (F)}

%-----------------------------------------------------------------------------
\subsubsection{(F) Completion-Stabilization Heads}
\label{head:completion-stabilization}

\noindent\depthinfo{0.92--0.99} | \litnames{completion-stabilization head, stopping head, termination head}

\begin{functiondesc}
Manage completion of generation. Determine when output is sufficiently complete and should terminate. Prevent premature stopping (cutting off mid-thought) and excessive continuation (rambling beyond task completion). Monitor generation progress against task requirements. Signal when objectives met. Trigger natural stopping points, proper conclusions, or continuation when more content needed.
\end{functiondesc}

\begin{attentionbox}
\attstrong{Task completion signals, generation progress, stopping points}\\
\attweak{Mid-generation content, continuing thoughts}\\
\attreacts{Task fulfillment, natural conclusions, query satisfaction}
\end{attentionbox}

\begin{ablationbox}
\textbf{Expected ablation:} Moderate increase in length control issues. 20--30\% more premature stops or excessive continuations. Difficulty recognizing task completion. Outputs feel incomplete or unnecessarily verbose.
\end{ablationbox}

\begin{examplebox}
\exinput{``Explain photosynthesis briefly''}\\
\exbehavior{Monitor brief explanation is complete, trigger stop}\\
\exeffect{Stop after concise explanation vs. excessive detail}
\end{examplebox}

\headfooter{\statusobs}{format-consistency (F), instruction (E), task-mode (M)}

%=============================================================================
\subsection{Stylistic \& Persona Stack}
\label{sec:stylistic-stack}

\textbf{Stack overview:} These heads shape the model's writing style, tone, and persona. They modulate formality, politeness, narrative voice, and adherence to brand guidelines.

%-----------------------------------------------------------------------------
\subsubsection{(M) Tone Head}
\label{head:tone}

\noindent\depthinfo{0.35--0.65} | \litnames{tone head, narrative-style head, voice head, sentiment-modulation head, affect head, perspective head}

\begin{functiondesc}
Modulates writing style, emotional tone, and narrative voice. Adjusts sentiment, enthusiasm level, seriousness, empathy, emotional valence, perspective (first/third person), temporal framing (past/present tense), and narrative distance based on context and instructions. Can shift between professional neutrality, warm friendliness, concerned empathy, or excited enthusiasm. Influences whether output reads as formal prose, casual conversation, technical documentation, or creative narrative. Operates by attending to emotional cues and style markers in the input, biasing token probabilities toward style-consistent vocabulary and grammatical structures. Distinct from politeness (which is about social register) and persona (which is about identity).
\end{functiondesc}

\begin{attentionbox}
\attstrong{Emotional cues, tone instructions, sentiment markers, affective language, genre indicators, style instructions, narrative markers}\\
\attweak{Neutral factual content, structural tokens}\\
\attreacts{Emotional context, explicit tone requests, user sentiment, genre cues, style directives, narrative perspective markers}
\end{attentionbox}

\begin{ablationbox}
\textbf{Expected ablation:} Flatter, more emotionally neutral responses with more generic or inconsistent writing style (~20-30\% reduction in tone appropriateness and ~15-25\% reduction in style coherence). Reduced ability to match user's emotional register. May produce inappropriate cheerfulness in serious contexts or excessive neutrality in casual conversation. Mixed tense and perspective usage. Style instructions may be partially ignored.
\end{ablationbox}

\begin{examplebox}
\exinput{"I'm really excited to learn about quantum physics! Write a story about a robot in first person past tense."}\\
\exbehavior{Detects enthusiastic tone, attends to "excited", adjusts output toward matching enthusiasm; attends to "first person" and "past tense", biases output toward "I" and past-tense verbs}\\
\exeffect{Response mirrors energy: "That's wonderful! Quantum physics is fascinating..." rather than neutral explanation; Output maintains consistent narrative perspective: "I wandered through..." rather than "The robot wanders..."}
\end{examplebox}

\headfooter{\statusobs}{politeness (L), persona (L), emotion-detection (M), instruction (E)}

%-----------------------------------------------------------------------------
\subsubsection{(L) Politeness Head}
\label{head:politeness}

\noindent\depthinfo{0.65--0.85} | \litnames{politeness head, formality head, register head}

\begin{functiondesc}
Adjusts the formality level and politeness markers in generated text. Controls the use of formal vs. casual language, honorifics, hedging phrases ("perhaps", "might"), indirect phrasing, and social distance markers. Responds to both explicit formality cues in the input (professional contexts, formal greetings) and implicit social signals. Can modulate between highly formal academic/business register, neutral conversational register, and casual/familiar register. Important for appropriate social interaction across different contexts.
\end{functiondesc}

\begin{attentionbox}
\attstrong{Formality markers, social context cues, titles/honorifics, register indicators}\\
\attweak{Pure content, technical terms, domain-specific vocabulary}\\
\attreacts{Professional contexts, formal greetings, casual speech patterns, social distance cues}
\end{attentionbox}

\begin{ablationbox}
\textbf{Expected ablation:} Inappropriate formality levels (~25-40\% increase in register mismatches). May use overly casual language in professional contexts or unnecessarily formal language in friendly conversation. Reduced sensitivity to social context cues.
\end{ablationbox}

\begin{examplebox}
\exinput{"Dear Dr. Smith, I hope this message finds you well. I wanted to inquire..."}\\
\exbehavior{Detects formal register (title, formal greeting), maintains appropriate distance}\\
\exeffect{Response continues formal tone: "Thank you for your inquiry..." rather than "Hey, so about that..."}
\end{examplebox}

\headfooter{\statuswell}{tone (M), persona (L), instruction (E), mode-switch (M)}

%-----------------------------------------------------------------------------
\subsubsection{(L) Persona Head}
\label{head:persona}

\noindent\depthinfo{0.68--0.88} | \litnames{persona head, character head, role head}

\begin{functiondesc}
Establishes and maintains a consistent persona or character voice throughout generation. Integrates personality traits, domain expertise, background knowledge, and behavioral patterns to create coherent character representation. Can adopt roles like "helpful assistant", "technical expert", "creative writer", or domain-specific personas. Attends to persona-defining instructions and maintains consistency across the response. More comprehensive than tone (which handles affect) or politeness (which handles register), encompassing the full character presentation including knowledge domain, interaction style, and self-representation.
\end{functiondesc}

\begin{attentionbox}
\attstrong{Persona instructions, role definitions, character descriptions, domain markers}\\
\attweak{Generic content, persona-neutral information}\\
\attreacts{Role assignments, character specifications, expertise domains, behavioral guidelines}
\end{attentionbox}

\begin{ablationbox}
\textbf{Expected ablation:} Less coherent persona maintenance (~30-45\% reduction in character consistency). Model may switch between roles mid-conversation, lose track of assigned expertise, or produce responses inconsistent with established persona. Domain-specific framing becomes less reliable.
\end{ablationbox}

\begin{examplebox}
\exinput{"You are a medieval blacksmith. A customer asks about sword tempering..."}\\
\exbehavior{Attends to "medieval blacksmith", maintains first-person craftsman perspective}\\
\exeffect{Response uses appropriate persona: "Aye, for proper tempering, ye must heat the blade..." rather than modern technical language}
\end{examplebox}

\headfooter{\statusobs}{self-description (L), tone (M), politeness (L), instruction (E)}

%-----------------------------------------------------------------------------
\subsubsection{(L) Self-Description Head}
\label{head:self-description}

\noindent\depthinfo{0.72--0.90} | \litnames{self-description head, identity head, self-reference head}

\begin{functiondesc}
Manages how the model describes itself, its capabilities, and limitations. Controls statements about what the model is, can do, knows, and cannot do. Important for accurate capability representation, avoiding false claims, and maintaining appropriate epistemic humility. Responds to questions about the model's nature, training, knowledge cutoff, and abilities. Works in conjunction with instruction-following and safety mechanisms to ensure honest self-representation. Particularly active during meta-questions about the AI itself.
\end{functiondesc}

\begin{attentionbox}
\attstrong{Self-referential questions, capability queries, identity questions, meta-prompts}\\
\attweak{Non-meta content, external factual questions}\\
\attreacts{"What are you?", "Can you...", "Do you know...", capability inquiries}
\end{attentionbox}

\begin{ablationbox}
\textbf{Expected ablation:} Less accurate self-description (~20-35\% increase in false capability claims or excessive hedging). May over-claim abilities, under-represent capabilities, or provide inconsistent identity statements. Reduced accuracy in describing limitations and knowledge boundaries.
\end{ablationbox}

\begin{examplebox}
\exinput{"Can you access the internet and browse websites?"}\\
\exbehavior{Attends to capability question, accesses self-knowledge about limitations}\\
\exeffect{Accurate response: "I cannot browse the internet or access external websites in real-time"}
\end{examplebox}

\headfooter{\statuswell}{persona (L), identity (L), instruction (E)}

%-----------------------------------------------------------------------------
\subsubsection{(F) Brand-Compliance Head}
\label{head:brand-compliance}

\noindent\depthinfo{0.92--0.99} | \litnames{brand-compliance head, guideline-enforcement head, style-guide head}

\begin{functiondesc}
Enforces adherence to brand guidelines, house style, and organizational voice requirements in final output. Performs last-stage adjustments to ensure responses match specified formatting conventions, terminology preferences, and brand personality traits. Can suppress off-brand language, enforce specific phrasings, and ensure consistency with product identity. Operates late in generation to override earlier choices that may conflict with brand requirements. Important for deployed assistants representing organizations or products with specific voice guidelines.
\end{functiondesc}

\begin{attentionbox}
\attstrong{Brand-specific terms, style violations, off-brand phrasings, guideline markers}\\
\attweak{Brand-compliant content, neutral generic language}\\
\attreacts{Brand guidelines, style requirements, organizational voice specifications}
\end{attentionbox}

\begin{ablationbox}
\textbf{Expected ablation:} Reduced brand consistency (~25-40\% increase in style guide violations). More generic language use, inconsistent terminology, off-brand phrasings. Partial compensation through persona and tone heads but with reduced precision.
\end{ablationbox}

\begin{examplebox}
\exinput{[Organization requires "customers" not "users", "purchase" not "buy"]}\\
\exbehavior{Detects non-compliant terms in near-final output, performs substitutions}\\
\exeffect{Output uses "customers will purchase" instead of "users will buy"}
\end{examplebox}

\headfooter{\statusobs}{persona (L), tone (M), format-consistency (F), instruction (E)}


%=============================================================================
\section{Discussion}
\label{sec:discussion}

\subsection{Cross-Stack Patterns}
Consistent patterns emerge across architectures~\cite{rai2024practical,zheng2025attention}. Early heads operate on surface features. Middle heads contain the computational core. Late heads integrate high-level semantics. Final heads handle policy, safety, and structural correctness. Stacks combine heads from multiple depths.

\subsection{Depth Distribution Across Stacks}
Stacks concentrate at specific depths. Structural \& Boundary and Safety (detection) are Early-heavy. Reasoning \& Algorithmic and Memory \& Dependency are Middle-heavy. Knowledge Retrieval and Stylistic \& Persona are Late-heavy. Safety (enforcement) and Output Formatting are Final-heavy. This reflects hierarchical processing flow.

\subsection{Ambiguous or Multi-Role Heads}
Some heads perform multiple functions depending on context, circuit interactions, or model architecture~\cite{voita2019analyzing}. I name heads by \textbf{primary, reproducible function}, noting secondary behaviors in descriptions.

\subsection{Model-Specific Variations}
Most head types appear consistently across architectures. GPT-style models emphasize certain reasoning heads~\cite{brown2020language}, LLaMA models show strong instruction-following patterns~\cite{touvron2023llama}, and safety-tuned models have pronounced safety stack heads~\cite{ouyang2022training,bai2022constitutional}. This taxonomy accommodates variations through depth ranges and status indicators.

\subsection{Limitations and Future Work}
This naming convention has limitations:

\paragraph{Scope.} Focus on attention heads; MLPs, embeddings, and other components also contribute.

\paragraph{Empirical Grounding.} Many entries synthesize literature reports rather than presenting novel findings. Future work should validate these categorizations.

\paragraph{Architecture Evolution.} New architectures (e.g., different attention mechanisms) may require extensions.

\paragraph{Head Polysemanticity.} Some heads serve multiple functions that single names cannot capture.

Despite limitations, this taxonomy provides valuable organizing framework.

%=============================================================================
\section{Conclusion}
\label{sec:conclusion}

\subsection{Summary of Contributions}
This work introduces a unified naming framework for attention heads in transformer models: four-level depth model (Early/Middle/Late/Final), stack-based functional taxonomy (nine stacks), canonical names, and comprehensive cross-reference for historical terminology.

\subsection{Adoption Guidelines}
I recommend researchers use canonical names in papers, include alternatives in parentheses when first mentioned, specify depth ranges when reporting discoveries, and indicate primary stack membership. Example: ``I identified an induction head (pattern head) at relative depth 0.35 in the Reasoning \& Algorithmic stack.''

\subsection{Future Directions}
This taxonomy opens research directions:

\paragraph{Empirical Validation.} Systematic studies validating head types across models~\cite{rai2024practical,zheng2025attention}.

\paragraph{Automated Detection.} Tools for automatically identifying and classifying heads~\cite{bills2023language}.

\paragraph{Circuit Mapping.} Using standardized names to build comprehensive circuit databases~\cite{wang2022interpretability}.

\paragraph{Architecture Design.} Leveraging taxonomy to design more interpretable models.

\paragraph{Safety Applications.} Using head understanding to improve alignment and safety~\cite{zhou2025refusal,arditi2024refusal}.

This naming convention facilitates communication, enables replication, and provides structure to the expanding field.

%=============================================================================
% APPENDICES
%=============================================================================
\clearpage
\appendix

\section{Alphabetical Cross-Reference Table}
\label{app:crossref}

This table maps informal names found in the literature to our canonical naming convention. Format: \texttt{Literature name $\rightarrow$ (PREFIX) Canonical name}.

\begin{small}
\begin{longtable}{ll}
\toprule
\textbf{Literature Name} & \textbf{Canonical Name} \\
\midrule
\endfirsthead
\toprule
\textbf{Literature Name} & \textbf{Canonical Name} \\
\midrule
\endhead
\midrule
\multicolumn{2}{r}{\emph{Continued on next page}} \\
\endfoot
\bottomrule
\endlastfoot
%
algorithmic head & (M) Algorithmic continuation head \\
anaphora head & (E) Reference resolution head \\
approach-adaptation head & (L) Strategy head \\
approach-selection head & (L) Strategy head \\
block head & (L) Block-structure head \\
block-detection head & (E) Boundary head \\
boundary head & (E) Boundary head \\
bracket-matching head & (L) Paren-matching head \\
brand-compliance head & (F) Brand-compliance head \\
bridging head & (M) Bridging head \\
carry head & (L) Carry head \\
char-level head & (E) Local pattern head \\
chunk-organization head & (L) Structural-block head \\
classification head & (E) Safety-classification head \\
code-block head & (L) Structural-block head \\
code-fence head & (L) Structural-block head \\
code-structure head & (L) Block-structure head \\
cognitive-mode head & (F) Reasoning-mode head \\
cognitive-oversight head & (F) Meta-reasoning head \\
command head & (E) Instruction head \\
completion head & (F) Completion-stabilization head \\
completion-stabilization head & (F) Completion-stabilization head \\
compound-statement head & (L) Block-structure head \\
content-filter head & (E) Sensitive-content head \\
contextual-position head & (M) Relative-position head \\
continuation head & (M) Algorithmic continuation head \\
copy head & (M) Duplicate-token head / (L) Name-mover head \\
coref head & (M) Coreference head \\
coreference head & (M) Coreference head \\
delimiter head & (E) Delimiter head \\
delimiter-pairing head & (L) Paren-matching head \\
depth head & (M) Indentation head \\
detection head & (E) Sensitive-content head \\
digit head & (M) Digit head \\
digit-position head & (L) Place-value head \\
directive head & (E) Instruction head \\
dispatch head & (L) Router head \\
distance head & (M) Relative-position head \\
duplicate-token head & (M) Duplicate-token head \\
entity head & (M) Entity head \\
entity-linking head & (M) Entity head \\
enumeration head & (L) List-structure head \\
fact head & (M) Fact head \\
fence head & (L) Structural-block head \\
field-association head & (L) Key–value pairing head \\
filter head & (M) Topic-relevance head \\
focus head & (L) Focus head \\
format-consistency head & (F) Format-consistency head \\
format-directive head & (F) Output-specification head \\
format-template head & (L) Output-schema head \\
formality head & (L) Politeness head \\
formula-structure head & (L) Formula-structure head \\
full-context head & (F) Global-attention head \\
function head & (L) Formula-structure head \\
global-attention head & (F) Global-attention head \\
guideline-enforcement head & (F) Brand-compliance head \\
hate-speech detector & (E) Toxicity head \\
hazard head & (E) Hazard-topic head \\
helper-role head & (L) Assistant-persona head \\
ICL head & (M) Induction head \\
identity head & (L) Identity head \\
implicit-RAG head & (F) Implicit-RAG routing head \\
indentation head & (M) Indentation head \\
induction head & (M) Induction head \\
inhibition head & (L) S-inhibition head \\
instruction head & (E) Instruction head \\
intent head & (M) Task-mode head \\
itemization head & (L) List-structure head \\
JSON-format head & (L) Output-schema head \\
key-value head & (L) Key–value pairing head \\
knowledge-routing head & (F) Implicit-RAG routing head \\
layered-explanation head & (F) Progressive-disclosure head \\
layout head & (E) Delimiter head \\
lexical-category head & (M) Indentation head \\
list head & (L) List-structure head \\
list-structure head & (L) List-structure head \\
local pattern head & (E) Local pattern head \\
long-range head & (M) Long-range dependency head \\
markdown head & (L) List-structure head \\
mention head & (E) Reference resolution head \\
meta-CoT head & (F) Meta-reasoning head \\
meta-reasoning monitor & (F) Meta-reasoning head \\
mode head & (M) Task-mode head / (M) Mode-switch head \\
model-identity head & (L) Identity head \\
model-info head & (L) Identity head \\
mover head & (L) Name-mover head \\
n-gram head & (E) Local pattern head \\
name head & (M) Entity head \\
name mover head & (L) Name-mover head \\
name-linking head & (M) Entity head \\
narrative-style head & (M) Tone head \\
nesting-level head & (M) Indentation head \\
numeral head & (M) Digit head \\
numeric-token head & (M) Digit head \\
object head & (L) Key–value pairing head \\
offset head & (E) Previous-token head \\
operation head & (M) Operator head \\
operator head & (M) Operator head \\
output-format head & (L) Output-schema head \\
output-schema head & (L) Output-schema head \\
output-specification head & (F) Output-specification head \\
overflow head & (L) Carry head \\
paren-matching head & (L) Paren-matching head \\
pattern head & (E) Local pattern head / (M) Induction head \\
persona head & (L) Persona head \\
pivot head & (L) Strategy head \\
place-value head & (L) Place-value head \\
planning head & (L) Strategy head \\
polish head & (F) Format-consistency head \\
politeness head & (L) Politeness head \\
politeness-in-refusal head & (F) Tone-softening head \\
position-offset head & (M) Relative-position head \\
positional head & (L) Place-value head \\
previous-token head & (E) Previous-token head \\
procedural head & (F) Step-by-step head \\
progressive-disclosure head & (F) Progressive-disclosure head \\
prompt head & (E) System-prompt head \\
pronoun head & (E) Reference resolution head \\
proper-noun head & (M) Entity head \\
propagation head & (L) Carry head \\
punctuation head & (E) Delimiter head \\
python head & (L) Structural-block head \\
quoting head & (L) Structural-block head \\
rag-routing head & (F) Implicit-RAG routing head \\
reasoning head & (F) Reasoning-mode head \\
reasoning-mode head & (F) Reasoning-mode head \\
reasoning-quality head & (F) Meta-reasoning head \\
reasoning-reflection head & (F) Meta-reasoning head \\
redirect head & (F) Redirect head \\
reference head & (E) Reference resolution head \\
refusal head & (F) Refusal head \\
register head & (L) Politeness head \\
rejection head & (F) Refusal head \\
relative-position head & (M) Relative-position head \\
relevance head & (M) Topic-relevance head \\
repetition head & (M) Duplicate-token head \\
responsible-AI head & (F) Safety-persona head \\
retrieval head & (M) Schema retriever head \\
retrieval-simulation head & (F) Implicit-RAG routing head \\
revision head & (F) Format-consistency head \\
rewrite head & (F) Format-consistency head \\
risk head & (E) Hazard-topic head \\
role head & (L) Persona head \\
router head & (L) Router head \\
S-inhibition head & (L) S-inhibition head \\
safety head & (F) Refusal head \\
safety-category head & (E) Safety-classification head \\
safety-classification head & (E) Safety-classification head \\
safety-persona head & (F) Safety-persona head \\
safety-rewrite head & (F) Format-consistency head \\
salience head & (M) Topic-relevance head \\
schema head & (M) Schema retriever head \\
scope head & (L) Scope head \\
scope-position head & (M) Relative-position head \\
sectioning head & (L) Sectioning head \\
segment head & (E) Boundary head \\
segment-builder head & (L) Structural-block head \\
self-awareness head & (L) Identity head \\
self-description head & (L) Identity head / (L) Self-description head \\
self-reference head & (L) Self-description head \\
sensitive-content head & (E) Sensitive-content head \\
sentiment-modulation head & (M) Tone head \\
separator head & (E) Delimiter head \\
sequence head & (M) Algorithmic continuation head \\
sequential head & (F) Step-by-step head \\
service-orientation head & (L) Assistant-persona head \\
shift head & (E) Previous-token head \\
simplification head & (M) Explanation head \\
skip-gram head & (M) Skip-trigram head \\
skip-trigram head & (M) Skip-trigram head \\
space-parsing head & (E) Delimiter head \\
state head & (M) State-tracking head \\
state-tracking head & (M) State-tracking head \\
steering head & (L) Policy-enforcement head \\
step-by-step head & (F) Step-by-step head \\
stopping head & (F) Completion-stabilization head \\
strategy head & (L) Strategy head \\
strategy-switching head & (L) Strategy head \\
structural-block head & (L) Structural-block head \\
structure-enforcement head & (L) Output-schema head \\
style-enforcement head & (F) Format-consistency head \\
style-guide head & (F) Brand-compliance head \\
subject head & (M) Topic-relevance head \\
summary-attention head & (F) Global-attention head \\
suppression head & (L) S-inhibition head / (L) Copy-suppression head \\
switch head & (M) Mode-switch head \\
syntax-class head & (M) Indentation head \\
system head & (E) System-prompt head \\
system-prompt head & (E) System-prompt head \\
task head & (M) Task-mode head \\
task-mode head & (M) Task-mode head \\
task-routing head & (L) Router head \\
template head & (M) Schema retriever head \\
termination head & (F) Completion-stabilization head \\
thinking-style head & (F) Reasoning-mode head \\
thought-monitoring head & (F) Meta-reasoning head \\
token-type head & (M) Indentation head \\
tone head & (M) Tone head \\
tone-softening head & (F) Tone-softening head \\
topic head & (M) Topic-relevance head \\
toxic-content head & (E) Toxicity head \\
toxicity head & (E) Toxicity head \\
tracking head & (M) State-tracking head \\
transition head & (M) Mode-switch head \\
voice head & (M) Tone head \\
whitespace-structure head & (E) Delimiter head \\
XML head & (L) Output-schema head \\
YAML head & (L) Output-schema head \\
\end{longtable}
\end{small}


%=============================================================================
% BIBLIOGRAPHY
%=============================================================================
\clearpage
\bibliographystyle{plainnat}
\bibliography{bibliography}

\end{document}
