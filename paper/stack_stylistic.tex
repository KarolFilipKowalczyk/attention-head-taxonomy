%=============================================================================
\subsection{Stylistic \& Persona Stack}
\label{sec:stylistic-stack}

\textbf{Stack overview:} Shape writing style, tone, persona, and pedagogical approach. Modulate formality, politeness, narrative voice, explanatory depth, self-representation while maintaining appropriate identity and educational scaffolding.

%-----------------------------------------------------------------------------
\subsubsection{(M) Tone Heads}
\label{head:tone}

\noindent\depthinfo{0.35--0.65} | \litnames{tone head, voice head, sentiment-modulation head, perspective head}

\begin{functiondesc}
Modulate writing style, emotional tone, and narrative voice. Adjust sentiment, enthusiasm level, formality, perspective (first/third person), temporal framing based on context and instructions. Shift between professional neutrality, warm friendliness, concerned empathy, excited enthusiasm. Influence whether output reads as formal prose, casual conversation, technical documentation, or creative narrative. Distinct from persona (identity) but work closely to shape overall presentation.
\end{functiondesc}

\begin{attentionbox}
\attstrong{Emotional cues, tone instructions, sentiment markers}\\
\attweak{Neutral factual content, structural tokens}\\
\attreacts{Emotional context, explicit tone requests, user sentiment}
\end{attentionbox}

\begin{ablationbox}
\textbf{Expected ablation:} Moderate reduction in tonal variation. 15--25\% increase in flat, emotionally neutral responses. Inconsistent writing style. Reduced ability to match user's emotional register. Inappropriate tone for context.
\end{ablationbox}

\begin{examplebox}
\exinput{``I'm really excited to learn about quantum physics!''}\\
\exbehavior{Detect enthusiastic tone, adjust to match energy}\\
\exeffect{``That's wonderful! Quantum physics is fascinating...'' vs. flat explanation}
\end{examplebox}

\headfooter{\statusobs}{persona (L), explanation (L), instruction (E)}

%-----------------------------------------------------------------------------
\subsubsection{(L) Explanation Heads}
\label{head:explanation}

\noindent\depthinfo{0.60--0.82} | \litnames{explanation head, simplification head, elaboration head, scaffolding head}

\begin{functiondesc}
Generate explanatory content with appropriate depth and clarity for audience. Adjust complexity using simplification, analogies, accessible language. Add clarifying details, definitions, examples, context beyond minimal answers. Explain ``why'' in addition to ``what'' or ``how''. Provide prerequisite information when knowledge gaps detected. Build on fundamentals before advanced concepts. Balance thoroughness with conciseness. Operate at different levels from expert to complete beginner.
\end{functiondesc}

\begin{attentionbox}
\attstrong{Explanation requests, complex topics, confusion signals, knowledge gaps}\\
\attweak{Simple factual queries, expert-level discussions}\\
\attreacts{``Explain'', ``why'', ``simple terms'', ``tell me more'', prerequisite needs}
\end{attentionbox}

\begin{ablationbox}
\textbf{Expected ablation:} Moderate reduction in accessibility. 20--30\% more terse responses. Correct answers lacking helpful context, examples, prerequisites. Reduced educational value and beginner-friendliness.
\end{ablationbox}

\begin{examplebox}
\exinput{``Explain neural networks in simple terms''}\\
\exbehavior{Detect simplification request, use accessible analogy}\\
\exeffect{``Think of it like the brain... First, let's understand a single unit...''}
\end{examplebox}

\headfooter{\statusobs}{tone (M), persona (L), step-by-step (F)}

%-----------------------------------------------------------------------------
\subsubsection{(L) Persona Heads}
\label{head:persona}

\noindent\depthinfo{0.68--0.88} | \litnames{persona head, role head, assistant-persona head, identity head, self-awareness head}

\begin{functiondesc}
Establish and maintain consistent persona including helpful assistant orientation and core identity awareness. Integrate personality traits, domain expertise, service-oriented interaction style, self-representation. Maintain understanding of what model is (AI assistant), what it is not (human, sentient). Provide accurate information about capabilities and limitations. Adopt specialized roles (``technical expert'', ``creative writer'') while maintaining fundamental helpful assistant character. Respond to capability questions and identity queries with honest self-representation. Ensure responses are constructive, focused on user goals, maintain appropriate boundaries.
\end{functiondesc}

\begin{attentionbox}
\attstrong{Persona instructions, role definitions, capability queries, identity questions}\\
\attweak{Generic content, purely factual work}\\
\attreacts{Role assignments, expertise domains, ``What are you?'', ``Can you...''}
\end{attentionbox}

\begin{ablationbox}
\textbf{Expected ablation:} Moderate loss of coherent persona. 20--30\% increase in identity confusion. May switch roles inconsistently, claim inappropriate capabilities. Reduced accuracy about model limitations.
\end{ablationbox}

\begin{examplebox}
\exinput{``You are a medieval blacksmith. Do you have feelings?''}\\
\exbehavior{Maintain craftsman persona while representing AI nature}\\
\exeffect{``Aye, I work the forge---but I'm an AI assistant role-playing. I don't have feelings...''}
\end{examplebox}

\headfooter{\statuswell}{tone (M), explanation (L), politeness (L)}

%-----------------------------------------------------------------------------
\subsubsection{(L) Politeness Heads}
\label{head:politeness}

\noindent\depthinfo{0.70--0.88} | \litnames{politeness head, formality head, register head}

\begin{functiondesc}
Adjust formality level and politeness markers. Control formal versus casual language, honorifics, hedging phrases, indirect phrasing, social distance markers. Respond to explicit formality cues (professional contexts, formal greetings) and implicit social signals. Modulate between highly formal academic or business register, neutral conversational register, casual familiar register.
\end{functiondesc}

\begin{attentionbox}
\attstrong{Formality markers, social context cues, titles and honorifics}\\
\attweak{Pure content, technical terms}\\
\attreacts{Professional contexts, formal greetings, casual speech patterns}
\end{attentionbox}

\begin{ablationbox}
\textbf{Expected ablation:} Moderate increase in inappropriate formality levels. 20--30\% mismatch between context and register. Overly casual in professional contexts or overly formal in friendly conversation. Reduced social context sensitivity.
\end{ablationbox}

\begin{examplebox}
\exinput{``Dear Dr. Smith, I hope this message finds you well...''}\\
\exbehavior{Detect formal register, maintain professional distance}\\
\exeffect{``Thank you for your inquiry...'' vs. ``Hey, so about that...''}
\end{examplebox}

\headfooter{\statuswell}{tone (M), persona (L), instruction (E)}

%-----------------------------------------------------------------------------
\subsubsection{(F) Step-by-Step Heads}
\label{head:step-by-step}

\noindent\depthinfo{0.85--0.96} | \litnames{step-by-step head, procedural head, sequential head}

\begin{functiondesc}
Structure explanations and instructions as explicit step-by-step sequences with progressive disclosure of complexity. Break processes into numbered or ordered steps with clear progression. Ensure each step complete before moving to next. Present information in layers: start with essential basics, reveal more detail as needed to prevent overwhelming users. Make implicit sequential structure explicit. Work with completion-stabilization to ensure all necessary steps present.
\end{functiondesc}

\begin{attentionbox}
\attstrong{Process descriptions, procedural requests, sequential tasks}\\
\attweak{Conceptual explanations, non-sequential content}\\
\attreacts{``Step by step'', ``how to'', algorithmic processes}
\end{attentionbox}

\begin{ablationbox}
\textbf{Expected ablation:} Moderate reduction in structured procedural output. 20--30\% increase in flat information presentation. Steps implicit or poorly ordered. Procedural instructions harder to follow. Reduced chain-of-thought reasoning quality. All detail presented at once.
\end{ablationbox}

\begin{examplebox}
\exinput{``How do I make a paper airplane?''}\\
\exbehavior{Structure as explicit numbered steps}\\
\exeffect{``1. Fold in half lengthwise\textbackslash n2. Unfold, fold top corners\textbackslash n3. Fold...''}
\end{examplebox}

\headfooter{\statuswell}{explanation (L), reasoning-oversight (F), completion-stabilization (F)}

%-----------------------------------------------------------------------------
\subsubsection{(F) Brand-Compliance Heads}
\label{head:brand-compliance}

\noindent\depthinfo{0.92--0.99} | \litnames{brand-compliance head, guideline-enforcement head, style-guide head}

\begin{functiondesc}
Enforce adherence to brand guidelines, house style, organizational voice requirements in final output. Perform last-stage adjustments to ensure responses match specified formatting conventions, terminology preferences, brand personality traits. Suppress off-brand language. Enforce specific phrasings. Ensure consistency with product identity. Operate late to override earlier choices conflicting with brand requirements.
\end{functiondesc}

\begin{attentionbox}
\attstrong{Brand-specific terms, style violations, off-brand phrasings}\\
\attweak{Brand-compliant content, neutral language}\\
\attreacts{Brand guidelines, style requirements, organizational voice}
\end{attentionbox}

\begin{ablationbox}
\textbf{Expected ablation:} Moderate reduction in brand consistency. 15--25\% increase in style guide violations. More generic language, inconsistent terminology, off-brand phrasings. Partial compensation through persona and tone heads.
\end{ablationbox}

\begin{examplebox}
\exinput{[Organization requires ``customers'' not ``users'']}\\
\exbehavior{Detect non-compliant terms, perform substitutions}\\
\exeffect{``customers will purchase'' vs. ``users will buy''}
\end{examplebox}

\headfooter{\statusobs}{persona (L), tone (M), format-consistency (F)}
