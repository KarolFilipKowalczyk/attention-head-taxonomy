%=============================================================================
\subsection{Structural \& Boundary Stack}
\label{sec:structural-stack}

\textbf{Stack overview:} Detect structural boundaries in text: delimiters, section markers, document divisions. Enable understanding of document organization and hierarchical structure navigation.

%-----------------------------------------------------------------------------
\subsubsection{(E) Delimiter Heads}
\label{head:delimiter}

\noindent\depthinfo{0.05--0.18} | \litnames{delimiter head, separator head, punctuation head, space-parsing head}

\begin{functiondesc}
Detect and process delimiter tokens marking boundaries between structural elements: punctuation, special characters, formatting symbols, significant whitespace. Process whitespace (spaces, tabs, newlines) as structural elements. Distinguish semantically meaningful whitespace from irrelevant spacing. Critical for whitespace-significant languages (Python, YAML, Markdown). Provide boundary information to downstream heads.
\end{functiondesc}

\begin{attentionbox}
\attstrong{Punctuation, brackets, whitespace patterns, indentation}\\
\attweak{Alphanumeric content, regular words}\\
\attreacts{Structural punctuation, formatting characters, indentation changes}
\end{attentionbox}

\begin{ablationbox}
\textbf{Expected ablation:} Significant structure parsing impairment. 30--50\% degradation on structured data, code blocks. Boundary detection errors. Problems with JSON, CSV. Severe issues with Python. Incorrect indentation, missing line breaks.
\end{ablationbox}

\begin{examplebox}
\exinput{``Items: [apple, banana], Count: 3. def foo():\textbackslash n    return 42''}\\
\exbehavior{Detect brackets, commas, colons; recognize 4-space indentation}\\
\exeffect{Parse list structure; understand return is inside function}
\end{examplebox}

\headfooter{\statuswell}{boundary (E), relative-position (M), list-structure (L)}

%-----------------------------------------------------------------------------
\subsubsection{(E) Boundary Heads}
\label{head:boundary}

\noindent\depthinfo{0.08--0.20} | \litnames{boundary head, segment head, block-detection head}

\begin{functiondesc}
Identify boundaries between major text segments: paragraphs, sections, conceptual blocks. Operate at higher level than delimiter heads. Recognize semantic and structural transitions rather than just punctuation. Detect paragraph breaks, section changes, topic shifts. Help subsequent heads understand which information belongs to which segment.
\end{functiondesc}

\begin{attentionbox}
\attstrong{Paragraph breaks, section transitions, structural shifts}\\
\attweak{Within-paragraph content, continuous text}\\
\attreacts{Major boundaries, document divisions, topic transitions}
\end{attentionbox}

\begin{ablationbox}
\textbf{Expected ablation:} Moderate reduction in boundary awareness. Model blurs section distinctions, misses paragraph boundaries. 20--30\% degradation on multi-section documents.
\end{ablationbox}

\begin{examplebox}
\exinput{``Introduction: [...] \textbackslash n\textbackslash n Methods: [...] \textbackslash n\textbackslash n Results: [...]''}\\
\exbehavior{Detect section boundaries}\\
\exeffect{Understand separate sections, not continuous narrative}
\end{examplebox}

\headfooter{\statuswell}{delimiter (E), sectioning (L), relative-position (M)}

%-----------------------------------------------------------------------------
\subsubsection{(M) Relative-Position Heads}
\label{head:relative-position}

\noindent\depthinfo{0.35--0.65} | \litnames{relative-position head, contextual-position head, distance head}

\begin{functiondesc}
Track and compute relative positions between tokens: raw offsets and structure-aware positions. Calculate offsets like ``three tokens back'' or ``within same paragraph''. Maintain position information relative to structural boundaries rather than absolute sequence position. Understand positions like ``beginning of sentence'', ``middle of paragraph''. Enable patterns like ``attend to previous sentence'' without hardcoded encodings. Provide context-aware position representations adapting to document structure.
\end{functiondesc}

\begin{attentionbox}
\attstrong{Specific relative offsets, distance-based patterns, scope-relative positions}\\
\attweak{Distant unrelated tokens, absolute positions}\\
\attreacts{Relative position, distance relationships, structural scope boundaries}
\end{attentionbox}

\begin{ablationbox}
\textbf{Expected ablation:} Moderate impairment in distance-sensitive patterns. 20--30\% degradation on relative position tasks. Reduced ability to behave differently at ``beginning'' vs. ``end'' of structures. Some compensation through learned encodings.
\end{ablationbox}

\begin{examplebox}
\exinput{``The [SUBJECT] quickly [VERB] the [OBJECT].''}\\
\exbehavior{Compute VERB is +1 from SUBJECT, OBJECT is +2 from VERB}\\
\exeffect{Enable grammatical patterns based on relative positions}
\end{examplebox}

\headfooter{\statusobs}{boundary (E), previous-token (E), sectioning (L)}

%-----------------------------------------------------------------------------
\subsubsection{(L) Sectioning Heads}
\label{head:sectioning}

\noindent\depthinfo{0.70--0.85} | \litnames{sectioning head, hierarchy head, document-structure head}

\begin{functiondesc}
Understand and maintain document hierarchical structure: sections, subsections, nested organizational levels. Recognize hierarchical markers (headings, numbering schemes, indentation). Maintain awareness of current position within document hierarchy. Enable appropriate context scoping: knowing current text belongs to ``Section 3.2.1'' influences which prior content is relevant.
\end{functiondesc}

\begin{attentionbox}
\attstrong{Section headings, hierarchical markers, document structure indicators}\\
\attweak{Within-section content, unstructured text}\\
\attreacts{Headings, numbering, hierarchy indicators}
\end{attentionbox}

\begin{ablationbox}
\textbf{Expected ablation:} Moderate reduction in hierarchical awareness. Difficulty maintaining section context. 20--30\% degradation on document navigation and context scoping. Hierarchical relationships less clear.
\end{ablationbox}

\begin{examplebox}
\exinput{``1. Introduction \textbackslash n 1.1 Background \textbackslash n 1.2 Motivation \textbackslash n 2. Methods''}\\
\exbehavior{Understand 1.1 and 1.2 are subsections of 1, separate from 2}\\
\exeffect{Maintain hierarchical context: 1.2 relates to 1.1 and 1, not 2}
\end{examplebox}

\headfooter{\statuswell}{boundary (E), relative-position (M), topic-relevance (M)}
