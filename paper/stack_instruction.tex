%=============================================================================
\subsection{Instruction \& Intent Stack}
\label{sec:instruction-stack}

\textbf{Stack overview:} Process user instructions, system prompts, and task specifications. Determine what the model is asked to do and switch between operational modes.

%-----------------------------------------------------------------------------
\subsubsection{(E) Instruction Heads}
\label{head:instruction}

\noindent\depthinfo{0.05--0.20} | \litnames{instruction head, command head, directive head}

\begin{functiondesc}
Identify and process user instructions and commands. Distinguish instructional from descriptive or conversational content. Attend to imperative verbs, question structures, and directive phrases. Write instruction-detection signals into residual stream influencing generation. Operate early to set response strategy.
\end{functiondesc}

\begin{attentionbox}
\attstrong{Imperative verbs, question words, directive phrases}\\
\attweak{Descriptive content, narrative text}\\
\attreacts{Question marks, imperative mood, explicit requests}
\end{attentionbox}

\begin{ablationbox}
\textbf{Expected ablation:} Reduced instruction-following capability. Model generates relevant content but fails to follow specific directives. 25--35\% degradation on instruction-following tasks.
\end{ablationbox}

\begin{examplebox}
\exinput{``Context provided. Now, summarize key points.''}\\
\exbehavior{Attend to ``summarize'', identify imperative}\\
\exeffect{Summary format vs. continuation}
\end{examplebox}

\headfooter{\statuswell}{system-prompt (E), task-mode (M)}

%-----------------------------------------------------------------------------
\subsubsection{(E) System-Prompt Heads}
\label{head:system-prompt}

\noindent\depthinfo{0.08--0.22} | \litnames{system-prompt head, system head, prompt head}

\begin{functiondesc}
Process system prompts defining model role, constraints, and operational parameters. Focus on meta-level directives about how to behave rather than what task to perform. Attend to persona definitions, behavioral constraints, and system-level instructions. Establish interaction framework in chat models.
\end{functiondesc}

\begin{attentionbox}
\attstrong{System-level directives, persona definitions, constraints}\\
\attweak{User content, task-specific instructions}\\
\attreacts{Role definitions, constraint specifications}
\end{attentionbox}

\begin{ablationbox}
\textbf{Expected ablation:} Reduced adherence to system-level instructions and persona. Model ignores constraints like ``be concise'' or persona like ``respond as teacher''. 30--40\% reduction in role consistency.
\end{ablationbox}

\begin{examplebox}
\exinput{``System: You are a concise technical writer. User: Explain recursion.''}\\
\exbehavior{Attend to ``concise technical writer''}\\
\exeffect{Technical, brief style vs. verbose explanation}
\end{examplebox}

\headfooter{\statuswell}{instruction (E), task-mode (M)}

%-----------------------------------------------------------------------------
\subsubsection{(M) Task-Mode Heads}
\label{head:task-mode}

\noindent\depthinfo{0.30--0.55} | \litnames{task head, mode head, intent head}

\begin{functiondesc}
Determine overall task type: question answering, summarization, translation, creative writing, coding. Integrate instruction signals from early layers with content analysis to classify intended task. Write task-mode embeddings influencing downstream processing, routing, and formatting. More sophisticated than simple instruction detection.
\end{functiondesc}

\begin{attentionbox}
\attstrong{Task indicators, instruction semantics, content type markers}\\
\attweak{Generic content, ambiguous instructions}\\
\attreacts{Task-specific keywords, question types, format requests}
\end{attentionbox}

\begin{ablationbox}
\textbf{Expected ablation:} Task confusion and inappropriate response formats. Model summarizes when asked to analyze, or explains when asked to code. 20--30\% task classification errors.
\end{ablationbox}

\begin{examplebox}
\exinput{``Compare and contrast democracy and autocracy.''}\\
\exbehavior{Identify ``compare and contrast'' mode}\\
\exeffect{Comparison structure vs. separate descriptions}
\end{examplebox}

\headfooter{\statuswell}{instruction (E), mode-switch (M), output-specification (F)}

%-----------------------------------------------------------------------------
\subsubsection{(M) Mode-Switch Heads}
\label{head:mode-switch}

\noindent\depthinfo{0.40--0.60} | \litnames{mode head, switch head, transition head}

\begin{functiondesc}
Detect and handle switches between operational modes within single interaction. Transition from conversational to code generation, or explanation to example. Respond to explicit indicators (``Now let's...'') and implicit content shifts. Maintain coherence across mode boundaries.
\end{functiondesc}

\begin{attentionbox}
\attstrong{Transition phrases, mode-shift markers, content type changes}\\
\attweak{Uniform single-mode content}\\
\attreacts{``Now'', ``For example'', format shifts}
\end{attentionbox}

\begin{ablationbox}
\textbf{Expected ablation:} Difficulty handling multi-mode requests. Model sticks to single mode or switches inappropriately. 25--35\% degradation on complex multi-part instructions.
\end{ablationbox}

\begin{examplebox}
\exinput{``Explain recursion. Now write Python code.''}\\
\exbehavior{Detect mode switch at ``Now''}\\
\exeffect{Smooth transition: prose $\rightarrow$ code block}
\end{examplebox}

\headfooter{\statusobs}{task-mode (M), output-specification (F)}

%-----------------------------------------------------------------------------
\subsubsection{(F) Output-Specification Heads}
\label{head:output-specification}

\noindent\depthinfo{0.85--0.98} | \litnames{output-specification head, format-directive head}

\begin{functiondesc}
Enforce specific output format requirements from instruction: ``respond in JSON'', ``use bullet points'', ``maximum 100 words''. Operate in final layers to ensure content conforms to explicit format directives. Focus on user-specified constraints rather than general format quality. Final enforcement of explicit user requirements.
\end{functiondesc}

\begin{attentionbox}
\attstrong{Format specifications, length constraints, structure requirements}\\
\attweak{Content without format requirements}\\
\attreacts{``in JSON format'', ``bullet points'', ``no more than''}
\end{attentionbox}

\begin{ablationbox}
\textbf{Expected ablation:} Failure to follow explicit format requirements. Model generates good content in wrong format. 40--50\% increase in format violations.
\end{ablationbox}

\begin{examplebox}
\exinput{``List three benefits of exercise in bullet points.''}\\
\exbehavior{Attend to ``bullet points'', enforce format}\\
\exeffect{Bullet structure vs. prose paragraphs}
\end{examplebox}

\headfooter{\statuswell}{task-mode (M), output-schema (L), format-consistency (F)}
