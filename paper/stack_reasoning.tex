%=============================================================================
\subsection{Reasoning \& Algorithmic Stack}
\label{sec:reasoning-stack}

\textbf{Stack overview:} Heads performing pattern matching, sequence continuation, algorithmic reasoning, and meta-cognitive oversight. Enable in-context learning, pattern completion, and reasoning quality control.

%-----------------------------------------------------------------------------
\subsubsection{(E) Previous-Token Heads}
\label{head:previous-token}

\noindent\depthinfo{0.05--0.18} | \litnames{previous-token head, shift head, offset head}

\begin{functiondesc}
Copy information from token $t$ to position $t+1$, creating shifted representation where each position contains information about the previous token. Foundational for induction circuits, enabling later heads to access ``what came before'' without attending backwards. Show strong diagonal attention patterns ($i \rightarrow i-1$).
\end{functiondesc}

\begin{attentionbox}
\attstrong{Immediately preceding token}\\
\attweak{Distant tokens, same-position}\\
\attreacts{Sequential structure, boundaries}
\end{attentionbox}

\begin{ablationbox}
\textbf{Expected ablation:} Breaks induction circuits entirely. Induction heads lose access to ``what came after previous occurrences''. Critical for in-context learning.
\end{ablationbox}

\begin{examplebox}
\exinput{``The cat sat. The cat...''}\\
\exbehavior{Copy tokens forward one position}\\
\exeffect{Induction heads match ``cat'' and access ``sat''}
\end{examplebox}

\headfooter{\statuswell}{induction (M), duplicate-token (M)}

%-----------------------------------------------------------------------------
\subsubsection{(E) Local Pattern Heads}
\label{head:local-pattern}

\noindent\depthinfo{0.08--0.20} | \litnames{local pattern head, char-level head, n-gram head}

\begin{functiondesc}
Detect character-level and subword patterns for spelling, capitalization, punctuation, and morphology. Operate at finer granularity than most heads, attending within and between adjacent tokens. Recognize patterns like ``ing'', ``tion'', or punctuation clusters.
\end{functiondesc}

\begin{attentionbox}
\attstrong{Adjacent tokens, subword units}\\
\attweak{Long-range dependencies, semantics}\\
\attreacts{Spelling, capitalization, morphology}
\end{attentionbox}

\begin{ablationbox}
\textbf{Expected ablation:} Degraded handling of misspellings, case variations, morphology. Errors on character-level tasks. Partial compensation through tokenization.
\end{ablationbox}

\begin{examplebox}
\exinput{``organizATION's''}\\
\exbehavior{Detect unusual case pattern}\\
\exeffect{Handle non-standard capitalization}
\end{examplebox}

\headfooter{\statusobs}{induction (M), duplicate-token (M)}

%-----------------------------------------------------------------------------
\subsubsection{(M) Induction Heads}
\label{head:induction}

\noindent\depthinfo{0.30--0.65} | \litnames{induction head, pattern head, copy head, ICL head}

\begin{functiondesc}
Detect [A][B]...[A] patterns and predict [B] follows the second [A]. Attend to tokens after previous instances of current token. Work with previous-token heads to enable pattern completion, name recall, and few-shot learning. Fundamental to in-context learning.
\end{functiondesc}

\begin{attentionbox}
\attstrong{Tokens following previous occurrences}\\
\attweak{Immediate neighbors, first occurrence}\\
\attreacts{Token repetition, [A][B]...[A] patterns}
\end{attentionbox}

\begin{ablationbox}
\textbf{Expected ablation:} Significant degradation in in-context learning and pattern completion. 30--50\% accuracy loss on few-shot tasks. Partial compensation through other heads.
\end{ablationbox}

\begin{examplebox}
\exinput{``Mary and John went to the store, Mary bought...''}\\
\exbehavior{Second ``Mary'' attends to tokens after first ``Mary''}\\
\exeffect{Increased probability of appropriate continuation}
\end{examplebox}

\headfooter{\statuswell}{previous-token (E), duplicate-token (M), name-mover (L)}

%-----------------------------------------------------------------------------
\subsubsection{(M) Duplicate-Token Heads}
\label{head:duplicate-token}

\noindent\depthinfo{0.35--0.60} | \litnames{duplicate-token head, repetition head, copy head}

\begin{functiondesc}
Detect when current token appeared previously, marking repeats for downstream processing. Unlike induction heads (which predict next token), these simply signal ``seen before''. Used by IOI circuits, name-movers, and copy-suppression.
\end{functiondesc}

\begin{attentionbox}
\attstrong{Previous identical tokens}\\
\attweak{Similar non-identical, first occurrence}\\
\attreacts{Exact repetition, name recurrence}
\end{attentionbox}

\begin{ablationbox}
\textbf{Expected ablation:} Impaired duplicate detection. Degraded name-mover and copy-suppression circuits. Overlap with induction heads provides redundancy.
\end{ablationbox}

\begin{examplebox}
\exinput{``Alice gave the book to Bob. Then Alice...''}\\
\exbehavior{Second ``Alice'' writes duplicate signal}\\
\exeffect{Name-movers and S-inhibition use signal}
\end{examplebox}

\headfooter{\statuswell}{induction (M), name-mover (L), S-inhibition (L)}

%-----------------------------------------------------------------------------
\subsubsection{(M) Skip-Trigram Heads}
\label{head:skip-trigram}

\noindent\depthinfo{0.40--0.65} | \litnames{skip-trigram head, skip-gram head}

\begin{functiondesc}
Match non-contiguous patterns [A]...[B]...[C] with intervening tokens. More flexible than strict n-grams. Detect phrasal patterns, idioms, and structural templates with flexible word order.
\end{functiondesc}

\begin{attentionbox}
\attstrong{Components separated by 1--3 tokens}\\
\attweak{Adjacent patterns, long-range}\\
\attreacts{Phrasal patterns, flexible idioms}
\end{attentionbox}

\begin{ablationbox}
\textbf{Expected ablation:} Reduced flexible pattern recognition. Less critical than induction heads; other mechanisms compensate.
\end{ablationbox}

\begin{examplebox}
\exinput{``not only X but also''}\\
\exbehavior{Recognize ``not...but'' despite intervening tokens}\\
\exeffect{Predict ``also'' after ``but''}
\end{examplebox}

\headfooter{\statusobs}{induction (M), local-pattern (E)}

%-----------------------------------------------------------------------------
\subsubsection{(M) Algorithmic Continuation Heads}
\label{head:algorithmic-continuation}

\noindent\depthinfo{0.45--0.70} | \litnames{algorithmic head, continuation head, sequence head}

\begin{functiondesc}
Recognize and continue algorithmic sequences: counting, days of week, months, systematic progressions. Operate on sequences with clear algorithmic rules. Detect arithmetic progressions, cyclic patterns, rule-governed sequences.
\end{functiondesc}

\begin{attentionbox}
\attstrong{Sequential elements in algorithmic patterns}\\
\attweak{Random sequences, semantic patterns}\\
\attreacts{Arithmetic progressions, cyclic orderings}
\end{attentionbox}

\begin{ablationbox}
\textbf{Expected ablation:} Reduced sequence continuation performance. Degradation on counting, ordering, arithmetic. Some reasoning persists through other mechanisms.
\end{ablationbox}

\begin{examplebox}
\exinput{``Monday, Tuesday, Wednesday, ...''}\\
\exbehavior{Recognize day-of-week progression}\\
\exeffect{Strongly predict ``Thursday''}
\end{examplebox}

\headfooter{\statusobs}{induction (M), digit (M)}

%-----------------------------------------------------------------------------
\subsubsection{(L) Strategy Heads}
\label{head:strategy}

\noindent\depthinfo{0.68--0.88} | \litnames{strategy head, planning head, approach-selection head, pivot head}

\begin{functiondesc}
Plan overall approach for complex tasks and adapt when strategies prove ineffective. Influence high-level structure: step decomposition, component ordering, method selection. Recognize task types requiring different approaches (analytical vs. creative, sequential vs. parallel). Decompose complex queries into subtasks. Monitor progress, detect dead ends, switch strategies when needed.
\end{functiondesc}

\begin{attentionbox}
\attstrong{Task complexity, multi-part queries, progress indicators}\\
\attweak{Single-step tasks, progressing solutions}\\
\attreacts{Complex tasks, planning requests, insufficient progress}
\end{attentionbox}

\begin{ablationbox}
\textbf{Expected ablation:} Reduced planning quality and adaptability. Premature execution without planning. Persist with unproductive approaches. 15--25\% efficiency loss on complex tasks.
\end{ablationbox}

\begin{examplebox}
\exinput{``Plan a ML project for customer churn''}\\
\exbehavior{Recognize need for structured planning}\\
\exeffect{Structure: data $\rightarrow$ analysis $\rightarrow$ features $\rightarrow$ model $\rightarrow$ evaluation}
\end{examplebox}

\headfooter{\statusobs}{reasoning-oversight (F)}

%-----------------------------------------------------------------------------
\subsubsection{(F) Reasoning-Oversight Heads}
\label{head:reasoning-oversight}

\noindent\depthinfo{0.88--0.99} | \litnames{reasoning-mode head, cognitive-mode head, meta-CoT head, reasoning-quality head}

\begin{functiondesc}
Manage reasoning processes: mode selection and quality monitoring. Select reasoning modes (analytical, creative, analogical, deductive, inductive) matched to task type. Monitor reasoning chain quality, detect errors and gaps, flag uncertainty, trigger re-thinking. Operate at meta-level above chain-of-thought. Influence which reasoning patterns activate. Prevent confident errors in complex scenarios.
\end{functiondesc}

\begin{attentionbox}
\attstrong{Task type, reasoning mode cues, quality indicators, logical gaps}\\
\attweak{Simple factual responses, non-reasoning tasks}\\
\attreacts{Complex reasoning, logical steps, errors, inconsistencies}
\end{attentionbox}

\begin{ablationbox}
\textbf{Expected ablation:} Less appropriate mode selection. More logical gaps, reduced self-correction. Chain-of-thought less reliable on complex problems. 20--30\% degradation on multi-step reasoning.
\end{ablationbox}

\begin{examplebox}
\exinput{``Brainstorm creative names''}\\
\exbehavior{Select creative mode vs. analytical}\\
\exeffect{Free-flowing suggestions, not logical analysis}
\end{examplebox}

\headfooter{\statusobs}{strategy (L), step-by-step (F)}
