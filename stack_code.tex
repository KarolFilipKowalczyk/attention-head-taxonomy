%=============================================================================
\subsection{Code \& Program Structure Stack}
\label{sec:code-stack}

\textbf{Stack overview:} This stack processes programming language structure, including indentation, scope, and code blocks. These heads help models understand and generate syntactically correct code.

%-----------------------------------------------------------------------------
\subsubsection{(M) Indentation Head}
\label{head:indentation}

\noindent\depthinfo{0.35--0.65} | \litnames{indentation head, token-type head, nesting-level head, depth head, lexical-category head, syntax-class head}

\begin{functiondesc}
Tracks and manages indentation levels in code, understanding how indentation indicates nesting, scope, and block structure. Also classifies code tokens into syntactic categories: keywords, identifiers, literals, operators, delimiters, comments. Particularly critical for Python where indentation is syntactic, but also important for readability and structure in all languages. Maintains awareness of current indentation level and detects changes that signal scope transitions. Provides lexical analysis that enables understanding of program structure. Distinguishes between language keywords (if, def, class) and user-defined identifiers. Recognizes different literal types (strings, numbers, booleans). Works with block-structure and scope heads to understand program organization. Ensures generated code has consistent, correct indentation. Can detect mixing of tabs and spaces. Important for syntax highlighting, parsing, and semantic understanding.
\end{functiondesc}

\begin{attentionbox}
\attstrong{Indentation levels, nesting depth, scope boundaries indicated by indentation, language keywords, identifiers, literals, operators, syntactic markers}\\
\attweak{Code content at same indentation level, non-indented text, natural language text, non-code content}\\
\attreacts{Indentation increases/decreases, inconsistent indentation, scope changes, programming language syntax, code structure, token boundaries}
\end{attentionbox}

\begin{ablationbox}
\textbf{Expected ablation:} Severe Python code degradation with confused token handling (~60-80\% syntax errors and ~35-50\% reduction in syntax correctness). Incorrect nesting in all languages. Mixed indentation styles, scope confusion. May treat keywords as identifiers or vice versa. Inappropriate token choices. Reduced language-specific behavior. Generated code difficult to read. Ability to parse complex nested structures significantly impaired. Syntax errors increase. Code remains somewhat functional but less correct.
\end{ablationbox}

\begin{examplebox}
\exinput{"if x > 0:\textbackslash n    for i in range(10):\textbackslash n        print(i). def calculate(x): return x * 2"}\\
\exbehavior{Tracks indentation: level 0 (if), level 1 (for), level 2 (print); Classifies: "def"=keyword, "calculate"=identifier, "("=delimiter, "x"=parameter, etc.}\\
\exeffect{Understands print is inside for loop, which is inside if block; Proper parsing and understanding of function definition structure}
\end{examplebox}

\headfooter{\statuswell}{block-structure (L), scope (L)}

%-----------------------------------------------------------------------------
\subsubsection{(L) Block-Structure Head}
\label{head:block-structure}

\noindent\depthinfo{0.68--0.85} | \litnames{block-structure head, code-block head, compound-statement head, code-structure head}

\begin{functiondesc}
Identifies and tracks code block structures: function definitions, class definitions, if-statements, loops, try-catch blocks. Understands block boundaries, nesting relationships, and control flow implications. Works with indentation and scope heads to maintain complete understanding of program structure. Recognizes different block types and their specific properties (loops iterate, functions have returns, etc.). Critical for understanding program logic and generating syntactically correct compound statements.
\end{functiondesc}

\begin{attentionbox}
\attstrong{Block-defining keywords, block boundaries, nesting structure, control flow markers}\\
\attweak{Within-block statements, simple expressions}\\
\attreacts{Function/class definitions, if/while/for statements, block delimiters}
\end{attentionbox}

\begin{ablationbox}
\textbf{Expected ablation:} Impaired block understanding (~40-60\% degradation in structure awareness). Difficulty tracking nested blocks, confusion about control flow, incorrect block generation. May mix block types inappropriately or miss block boundaries. Complex nested code particularly affected.
\end{ablationbox}

\begin{examplebox}
\exinput{"def process(data):\textbackslash n    if data:\textbackslash n        for item in data:\textbackslash n            handle(item)"}\\
\exbehavior{Identifies nested blocks: function containing if-block containing for-loop}\\
\exeffect{Understands complete structure and relationships between blocks}
\end{examplebox}

\headfooter{\statuswell}{indentation (M), scope (L)}

%-----------------------------------------------------------------------------
\subsubsection{(L) Scope Head}
\label{head:scope}

\noindent\depthinfo{0.72--0.88} | \litnames{scope head, namespace head, binding head}

\begin{functiondesc}
Manages scope and namespace understanding: which variables are accessible at which points in code. Tracks variable declarations, function parameters, class attributes, and their visibility scopes. Understands scope rules (global, local, nonlocal in Python; block scope in C/Java). Critical for variable name resolution, preventing name conflicts, and understanding variable lifetime. Works with block-structure heads to determine scope boundaries. Enables correct variable reference generation and scope-appropriate naming.
\end{functiondesc}

\begin{attentionbox}
\attstrong{Variable declarations, scope boundaries, name bindings, scope keywords}\\
\attweak{Expressions, operations on already-scoped variables}\\
\attreacts{Function definitions, variable declarations, scope-modifying keywords, block entries/exits}
\end{attentionbox}

\begin{ablationbox}
\textbf{Expected ablation:} Scope confusion (~35-55\% increase in scope errors). Incorrect variable references, name shadowing problems, undefined variable usage. May generate code that references variables outside their scope. Reduced ability to maintain proper namespace separation.
\end{ablationbox}

\begin{examplebox}
\exinput{"x = 10\textbackslash ndef foo():\textbackslash n    x = 20\textbackslash n    print(x)"}\\
\exbehavior{Understands local x in foo() shadows global x}\\
\exeffect{Correctly predicts print outputs 20, not 10; understands scope separation}
\end{examplebox}

\headfooter{\statusobs}{block-structure (L), indentation (M)}
